\biblebook{Kejadian}

\begin{biblechapter} % Kejadian 1
\verseWithHeading{Allah menciptakan langit dan bumi serta isinya} Pada mulanya Allah menciptakan langit dan bumi.
\verse Bumi belum berbentuk dan kosong; gelap gulita menutupi samudera raya, dan Roh Allah melayang-layang di atas permukaan air.
\verse Berfirmanlah Allah: "Jadilah terang." Lalu terang itu jadi.
\verse Allah melihat bahwa terang itu baik, lalu dipisahkan-Nyalah terang itu dari gelap.
\verse Dan Allah menamai terang itu siang, dan gelap itu malam. Jadilah petang dan jadilah pagi, itulah hari pertama.
\verse Berfirmanlah Allah: "Jadilah cakrawala di tengah segala air untuk memisahkan air dari air."
\verse Maka Allah menjadikan cakrawala dan Ia memisahkan air yang ada di bawah cakrawala itu dari air yang ada di atasnya. Dan jadilah demikian.
\verse Lalu Allah menamai cakrawala itu langit. Jadilah petang dan jadilah pagi, itulah hari kedua.
\verse Berfirmanlah Allah: "Hendaklah segala air yang di bawah langit berkumpul pada satu tempat, sehingga kelihatan yang kering." Dan jadilah demikian.
\verse Lalu Allah menamai yang kering itu darat, dan kumpulan air itu dinamai-Nya laut. Allah melihat bahwa semuanya itu baik.
\verse Berfirmanlah Allah: "Hendaklah tanah menumbuhkan tunas-tunas muda, tumbuh-tumbuhan yang berbiji, segala jenis pohon buah-buahan yang menghasilkan buah yang berbiji, supaya ada tumbuh-tumbuhan di bumi." Dan jadilah demikian.
\verse Tanah itu menumbuhkan tunas-tunas muda, segala jenis tumbuh-tumbuhan yang berbiji dan segala jenis pohon-pohonan yang menghasilkan buah yang berbiji. Allah melihat bahwa semuanya itu baik.
\verse Jadilah petang dan jadilah pagi, itulah hari ketiga.
\verse Berfirmanlah Allah: "Jadilah benda-benda penerang pada cakrawala untuk memisahkan siang dari malam. Biarlah benda-benda penerang itu menjadi tanda yang menunjukkan masa-masa yang tetap dan hari-hari dan tahun-tahun,
\verse dan sebagai penerang pada cakrawala biarlah benda-benda itu menerangi bumi." Dan jadilah demikian.
\verse Maka Allah menjadikan kedua benda penerang yang besar itu, yakni yang lebih besar untuk menguasai siang dan yang lebih kecil untuk menguasai malam, dan menjadikan juga bintang-bintang.
\verse Allah menaruh semuanya itu di cakrawala untuk menerangi bumi,
\verse dan untuk menguasai siang dan malam, dan untuk memisahkan terang dari gelap. Allah melihat bahwa semuanya itu baik.
\verse Jadilah petang dan jadilah pagi, itulah hari keempat.
\verse Berfirmanlah Allah: "Hendaklah dalam air berkeriapan makhluk yang hidup, dan hendaklah burung beterbangan di atas bumi melintasi cakrawala."
\verse Maka Allah menciptakan binatang-binatang laut yang besar dan segala jenis makhluk hidup yang bergerak, yang berkeriapan dalam air, dan segala jenis burung yang bersayap. Allah melihat bahwa semuanya itu baik.
\verse Lalu Allah memberkati semuanya itu, firman-Nya: "Berkembangbiaklah dan bertambah banyaklah serta penuhilah air dalam laut, dan hendaklah burung-burung di bumi bertambah banyak."
\verse Jadilah petang dan jadilah pagi, itulah hari kelima.
\verse Berfirmanlah Allah: "Hendaklah bumi mengeluarkan segala jenis makhluk yang hidup, ternak dan binatang melata dan segala jenis binatang liar." Dan jadilah demikian.
\verse Allah menjadikan segala jenis binatang liar dan segala jenis ternak dan segala jenis binatang melata di muka bumi. Allah melihat bahwa semuanya itu baik.
\verse Berfirmanlah Allah: "Baiklah Kita menjadikan manusia menurut gambar dan rupa Kita, supaya mereka berkuasa atas ikan-ikan di laut dan burung-burung di udara dan atas ternak dan atas seluruh bumi dan atas segala binatang melata yang merayap di bumi."
\verse Maka Allah menciptakan manusia itu menurut gambar-Nya, menurut gambar Allah diciptakan-Nya dia; laki-laki dan perempuan diciptakan-Nya mereka.
\verse Allah memberkati mereka, lalu Allah berfirman kepada mereka: "Beranakcuculah dan bertambah banyak; penuhilah bumi dan taklukkanlah itu, berkuasalah atas ikan-ikan di laut dan burung-burung di udara dan atas segala binatang yang merayap di bumi."
\verse Berfirmanlah Allah: "Lihatlah, Aku memberikan kepadamu segala tumbuh-tumbuhan yang berbiji di seluruh bumi dan segala pohon-pohonan yang buahnya berbiji; itulah akan menjadi makananmu.
\verse Tetapi kepada segala binatang di bumi dan segala burung di udara dan segala yang merayap di bumi, yang bernyawa, Kuberikan segala tumbuh-tumbuhan hijau menjadi makanannya." Dan jadilah demikian.
\verse Maka Allah melihat segala yang dijadikan-Nya itu, sungguh amat baik. Jadilah petang dan jadilah pagi, itulah hari keenam.
\end{biblechapter}

\begin{biblechapter} % Kejadian 2
\verse Demikianlah diselesaikan langit dan bumi dan segala isinya.
\verse Ketika Allah pada hari ketujuh telah menyelesaikan pekerjaan yang dibuat-Nya itu, berhentilah Ia pada hari ketujuh dari segala pekerjaan yang telah dibuat-Nya itu.
\verse Lalu Allah memberkati hari ketujuh itu dan menguduskannya, karena pada hari itulah Ia berhenti dari segala pekerjaan penciptaan yang telah dibuat-Nya itu.
\verse Demikianlah riwayat langit dan bumi pada waktu diciptakan. Ketika TUHAN Allah menjadikan bumi dan langit, ?
\verse belum ada semak apapun di bumi, belum timbul tumbuh-tumbuhan apapun di padang, sebab TUHAN Allah belum menurunkan hujan ke bumi, dan belum ada orang untuk mengusahakan tanah itu;
\verse tetapi ada kabut naik ke atas dari bumi dan membasahi seluruh permukaan bumi itu?
\verse ketika itulah TUHAN Allah membentuk manusia itu dari debu tanah dan menghembuskan nafas hidup ke dalam hidungnya; demikianlah manusia itu menjadi makhluk yang hidup.
\verseWithSubheading{Manusia dan taman Eden} Selanjutnya TUHAN Allah membuat taman di Eden, di sebelah timur; disitulah ditempatkan-Nya manusia yang dibentuk-Nya itu.
\verse Lalu TUHAN Allah menumbuhkan berbagai-bagai pohon dari bumi, yang menarik dan yang baik untuk dimakan buahnya; dan pohon kehidupan di tengah-tengah taman itu, serta pohon pengetahuan tentang yang baik dan yang jahat.
\verse Ada suatu sungai mengalir dari Eden untuk membasahi taman itu, dan dari situ sungai itu terbagi menjadi empat cabang.
\verse Yang pertama, namanya Pison, yakni yang mengalir mengelilingi seluruh tanah Hawila, tempat emas ada.
\verse Dan emas dari negeri itu baik; di sana ada damar bedolah dan batu krisopras.
\verse Nama sungai yang kedua ialah Gihon, yakni yang mengalir mengelilingi seluruh tanah Kush.
\verse Nama sungai yang ketiga ialah Tigris, yakni yang mengalir di sebelah timur Asyur. Dan sungai yang keempat ialah Efrat.
\verse TUHAN Allah mengambil manusia itu dan menempatkannya dalam taman Eden untuk mengusahakan dan memelihara taman itu.
\verse Lalu TUHAN Allah memberi perintah ini kepada manusia: "Semua pohon dalam taman ini boleh kaumakan buahnya dengan bebas,
\verse tetapi pohon pengetahuan tentang yang baik dan yang jahat itu, janganlah kaumakan buahnya, sebab pada hari engkau memakannya, pastilah engkau mati."
\verse TUHAN Allah berfirman: "Tidak baik, kalau manusia itu seorang diri saja. Aku akan menjadikan penolong baginya, yang sepadan dengan dia."
\verse Lalu TUHAN Allah membentuk dari tanah segala binatang hutan dan segala burung di udara. Dibawa-Nyalah semuanya kepada manusia itu untuk melihat, bagaimana ia menamainya; dan seperti nama yang diberikan manusia itu kepada tiap-tiap makhluk yang hidup, demikianlah nanti nama makhluk itu.
\verse Manusia itu memberi nama kepada segala ternak, kepada burung-burung di udara dan kepada segala binatang hutan, tetapi baginya sendiri ia tidak menjumpai penolong yang sepadan dengan dia.
\verse Lalu TUHAN Allah membuat manusia itu tidur nyenyak; ketika ia tidur, TUHAN Allah mengambil salah satu rusuk dari padanya, lalu menutup tempat itu dengan daging.
\verse Dan dari rusuk yang diambil TUHAN Allah dari manusia itu, dibangun-Nyalah seorang perempuan, lalu dibawa-Nya kepada manusia itu.
\verse Lalu berkatalah manusia itu: "Inilah dia, tulang dari tulangku dan daging dari dagingku. Ia akan dinamai perempuan, sebab ia diambil dari laki-laki."
\verse Sebab itu seorang laki-laki akan meninggalkan ayahnya dan ibunya dan bersatu dengan isterinya, sehingga keduanya menjadi satu daging.
\verse Mereka keduanya telanjang, manusia dan isterinya itu, tetapi mereka tidak merasa malu.
\end{biblechapter}

\begin{biblechapter} % Kejadian 3
\verseWithHeading{Manusia jatuh ke dalam dosa} Adapun ular ialah yang paling cerdik dari segala binatang di darat yang dijadikan oleh TUHAN Allah. Ular itu berkata kepada perempuan itu: "Tentulah Allah berfirman: Semua pohon dalam taman ini jangan kamu makan buahnya, bukan?"
\verse Lalu sahut perempuan itu kepada ular itu: "Buah pohon-pohonan dalam taman ini boleh kami makan,
\verse tetapi tentang buah pohon yang ada di tengah-tengah taman, Allah berfirman: Jangan kamu makan ataupun raba buah itu, nanti kamu mati."
\verse Tetapi ular itu berkata kepada perempuan itu: "Sekali-kali kamu tidak akan mati,
\verse tetapi Allah mengetahui, bahwa pada waktu kamu memakannya matamu akan terbuka, dan kamu akan menjadi seperti Allah, tahu tentang yang baik dan yang jahat."
\verse Perempuan itu melihat, bahwa buah pohon itu baik untuk dimakan dan sedap kelihatannya, lagipula pohon itu menarik hati karena memberi pengertian. Lalu ia mengambil dari buahnya dan dimakannya dan diberikannya juga kepada suaminya yang bersama-sama dengan dia, dan suaminyapun memakannya.
\verse Maka terbukalah mata mereka berdua dan mereka tahu, bahwa mereka telanjang; lalu mereka menyemat daun pohon ara dan membuat cawat.
\verse Ketika mereka mendengar bunyi langkah TUHAN Allah, yang berjalan-jalan dalam taman itu pada waktu hari sejuk, bersembunyilah manusia dan isterinya itu terhadap TUHAN Allah di antara pohon-pohonan dalam taman.
\verse Tetapi TUHAN Allah memanggil manusia itu dan berfirman kepadanya: "Di manakah engkau?"
\verse Ia menjawab: "Ketika aku mendengar, bahwa Engkau ada dalam taman ini, aku menjadi takut, karena aku telanjang; sebab itu aku bersembunyi."
\verse Firman-Nya: "Siapakah yang memberitahukan kepadamu, bahwa engkau telanjang? Apakah engkau makan dari buah pohon, yang Kularang engkau makan itu?"
\verse Manusia itu menjawab: "Perempuan yang Kautempatkan di sisiku, dialah yang memberi dari buah pohon itu kepadaku, maka kumakan."
\verse Kemudian berfirmanlah TUHAN Allah kepada perempuan itu: "Apakah yang telah kauperbuat ini?" Jawab perempuan itu: "Ular itu yang memperdayakan aku, maka kumakan."
\verse Lalu berfirmanlah TUHAN Allah kepada ular itu: "Karena engkau berbuat demikian, terkutuklah engkau di antara segala ternak dan di antara segala binatang hutan; dengan perutmulah engkau akan menjalar dan debu tanahlah akan kaumakan seumur hidupmu.
\verse Aku akan mengadakan permusuhan antara engkau dan perempuan ini, antara keturunanmu dan keturunannya; keturunannya akan meremukkan kepalamu, dan engkau akan meremukkan tumitnya."
\verse Firman-Nya kepada perempuan itu: "Susah payahmu waktu mengandung akan Kubuat sangat banyak; dengan kesakitan engkau akan melahirkan anakmu; namun engkau akan berahi kepada suamimu dan ia akan berkuasa atasmu."
\verse Lalu firman-Nya kepada manusia itu: "Karena engkau mendengarkan perkataan isterimu dan memakan dari buah pohon, yang telah Kuperintahkan kepadamu: Jangan makan dari padanya, maka terkutuklah tanah karena engkau; dengan bersusah payah engkau akan mencari rezekimu dari tanah seumur hidupmu:
\verse semak duri dan rumput duri yang akan dihasilkannya bagimu, dan tumbuh-tumbuhan di padang akan menjadi makananmu;
\verse dengan berpeluh engkau akan mencari makananmu, sampai engkau kembali lagi menjadi tanah, karena dari situlah engkau diambil; sebab engkau debu dan engkau akan kembali menjadi debu."
\verse Manusia itu memberi nama Hawa kepada isterinya, sebab dialah yang menjadi ibu semua yang hidup.
\verse Dan TUHAN Allah membuat pakaian dari kulit binatang untuk manusia dan untuk isterinya itu, lalu mengenakannya kepada mereka.
\verse Berfirmanlah TUHAN Allah: "Sesungguhnya manusia itu telah menjadi seperti salah satu dari Kita, tahu tentang yang baik dan yang jahat; maka sekarang jangan sampai ia mengulurkan tangannya dan mengambil pula dari buah pohon kehidupan itu dan memakannya, sehingga ia hidup untuk selama-lamanya."
\verse Lalu TUHAN Allah mengusir dia dari taman Eden supaya ia mengusahakan tanah dari mana ia diambil.
\verse Ia menghalau manusia itu dan di sebelah timur taman Eden ditempatkan-Nyala beberapa kerub dengan pedang yang bernyala-nyala dan menyambar-nyambar, untuk menjaga jalan ke pohon kehidupan.
\end{biblechapter}

\begin{biblechapter} % Kejadian 4
\verseWithHeading{Kain dan Habel} Kemudian manusia itu bersetubuh dengan Hawa, isterinya, dan mengandunglah perempuan itu, lalu melahirkan Kain; maka kata perempuan itu: "Aku telah mendapat seorang anak laki-laki dengan pertolongan TUHAN."
\verse Selanjutnya dilahirkannyalah Habel, adik Kain; dan Habel menjadi gembala kambing domba, Kain menjadi petani.
\verse Setelah beberapa waktu lamanya, maka Kain mempersembahkan sebagian dari hasil tanah itu kepada TUHAN sebagai korban persembahan;
\verse Habel juga mempersembahkan korban persembahan dari anak sulung kambing dombanya, yakni lemak-lemaknya; maka TUHAN mengindahkan Habel dan korban persembahannya itu,
\verse tetapi Kain dan korban persembahannya tidak diindahkan-Nya. Lalu hati Kain menjadi sangat panas, dan mukanya muram.
\verse Firman TUHAN kepada Kain: "Mengapa hatimu panas dan mukamu muram?
\verse Apakah mukamu tidak akan berseri, jika engkau berbuat baik? Tetapi jika engkau tidak berbuat baik, dosa sudah mengintip di depan pintu; ia sangat menggoda engkau, tetapi engkau harus berkuasa atasnya."
\verse Kata Kain kepada Habel, adiknya: "Marilah kita pergi ke padang." Ketika mereka ada di padang, tiba-tiba Kain memukul Habel, adiknya itu, lalu membunuh dia.
\verse Firman TUHAN kepada Kain: "Di mana Habel, adikmu itu?" Jawabnya: "Aku tidak tahu! Apakah aku penjaga adikku?"
\verse Firman-Nya: "Apakah yang telah kauperbuat ini? Darah adikmu itu berteriak kepada-Ku dari tanah.
\verse Maka sekarang, terkutuklah engkau, terbuang jauh dari tanah yang mengangakan mulutnya untuk menerima darah adikmu itu dari tanganmu.
\verse Apabila engkau mengusahakan tanah itu, maka tanah itu tidak akan memberikan hasil sepenuhnya lagi kepadamu; engkau menjadi seorang pelarian dan pengembara di bumi."
\verse Kata Kain kepada TUHAN: "Hukumanku itu lebih besar dari pada yang dapat kutanggung.
\verse Engkau menghalau aku sekarang dari tanah ini dan aku akan tersembunyi dari hadapan-Mu, seorang pelarian dan pengembara di bumi; maka barangsiapa yang akan bertemu dengan aku, tentulah akan membunuh aku."
\verse Firman TUHAN kepadanya: "Sekali-kali tidak! Barangsiapa yang membunuh Kain akan dibalaskan kepadanya tujuh kali lipat." Kemudian TUHAN menaruh tanda pada Kain, supaya ia jangan dibunuh oleh barangsiapapun yang bertemu dengan dia.
\verse Lalu Kain pergi dari hadapan TUHAN dan ia menetap di tanah Nod, di sebelah timur Eden.
\verseWithSubheading{Keturunan Kain, Set dan Enos} Kain bersetubuh dengan isterinya dan mengandunglah perempuan itu, lalu melahirkan Henokh; kemudian Kain mendirikan suatu kota dan dinamainya kota itu Henokh, menurut nama anaknya.
\verse Bagi Henokh lahirlah Irad, dan Irad itu memperanakkan Mehuyael dan Mehuyael memperanakkan Metusael, dan Metusael memperanakkan Lamekh.
\verse Lamekh mengambil isteri dua orang; yang satu namanya Ada, yang lain Zila.
\verse Ada itu melahirkan Yabal; dialah yang menjadi bapa orang yang diam dalam kemah dan memelihara ternak.
\verse Nama adiknya ialah Yubal; dialah yang menjadi bapa semua orang yang memainkan kecapi dan suling.
\verse Zila juga melahirkan anak, yakni Tubal-Kain, bapa semua tukang tembaga dan tukang besi. Adik perempuan Tubal-Kain ialah Naama.
\verse Berkatalah Lamekh kepada kedua isterinya itu: "Ada dan Zila, dengarkanlah suaraku: hai isteri-isteri Lamekh, pasanglah telingamu kepada perkataanku ini: Aku telah membunuh seorang laki-laki karena ia melukai aku, membunuh seorang muda karena ia memukul aku sampai bengkak;
\verse sebab jika Kain harus dibalaskan tujuh kali lipat, maka Lamekh tujuh puluh tujuh kali lipat."
\verse Adam bersetubuh pula dengan isterinya, lalu perempuan itu melahirkan seorang anak laki-laki dan menamainya Set, sebab katanya: "Allah telah mengaruniakan kepadaku anak yang lain sebagai ganti Habel; sebab Kain telah membunuhnya."
\verse Lahirlah seorang anak laki-laki bagi Set juga dan anak itu dinamainya Enos. Waktu itulah orang mulai memanggil nama TUHAN.
\end{biblechapter}

\begin{biblechapter} % Kejadian 5
\verseWithHeading{Keturunan Adam} Inilah daftar keturunan Adam. Pada waktu manusia itu diciptakan oleh Allah, dibuat-Nyalah dia menurut rupa Allah;
\verse laki-laki dan perempuan diciptakan-Nya mereka. Ia memberkati mereka dan memberikan nama "Manusia" kepada mereka, pada waktu mereka diciptakan.
\verse Setelah Adam hidup seratus tiga puluh tahun, ia memperanakkan seorang laki-laki menurut rupa dan gambarnya, lalu memberi nama Set kepadanya.
\verse Umur Adam, setelah memperanakkan Set, delapan ratus tahun, dan ia memperanakkan anak-anak lelaki dan perempuan.
\verse Jadi Adam mencapai umur sembilan ratus tiga puluh tahun, lalu ia mati.
\verse Setelah Set hidup seratus lima tahun, ia memperanakkan Enos.
\verse Dan Set masih hidup delapan ratus tujuh tahun, setelah ia memperanakkan Enos, dan ia memperanakkan anak-anak lelaki dan perempuan.
\verse Jadi Set mencapai umur sembilan ratus dua belas tahun, lalu ia mati.
\verse Setelah Enos hidup sembilan puluh tahun, ia memperanakkan Kenan.
\verse Dan Enos masih hidup delapan ratus lima belas tahun, setelah ia memperanakkan Kenan, dan ia memperanakkan anak-anak lelaki dan perempuan.
\verse Jadi Enos mencapai umur sembilan ratus lima tahun, lalu ia mati.
\verse Setelah Kenan hidup tujuh puluh tahun, ia memperanakkan Mahalaleel.
\verse Dan Kenan masih hidup delapan ratus empat puluh tahun, setelah ia memperanakkan Mahalaleel, dan ia memperanakkan anak-anak lelaki dan perempuan.
\verse Jadi Kenan mencapai umur sembilan ratus sepuluh tahun, lalu ia mati.
\verse Setelah Mahalaleel hidup enam puluh lima tahun, ia memperanakkan Yared.
\verse Dan Mahalaleel masih hidup delapan ratus tiga puluh tahun, setelah ia memperanakkan Yared, dan ia memperanakkan anak-anak lelaki dan perempuan.
\verse Jadi Mahalaleel mencapai umur delapan ratus sembilan puluh lima tahun, lalu ia mati.
\verse Setelah Yared hidup seratus enam puluh dua tahun, ia memperanakkan Henokh.
\verse Dan Yared masih hidup delapan ratus tahun, setelah ia memperanakkan Henokh, dan ia memperanakkan anak-anak lelaki dan perempuan.
\verse Jadi Yared mencapai umur sembilan ratus enam puluh dua tahun, lalu ia mati.
\verse Setelah Henokh hidup enam puluh lima tahun, ia memperanakkan Metusalah.
\verse Dan Henokh hidup bergaul dengan Allah selama tiga ratus tahun lagi, setelah ia memperanakkan Metusalah, dan ia memperanakkan anak-anak lelaki dan perempuan
\verse Jadi Henokh mencapai umur tiga ratus enam puluh lima tahun.
\verse Dan Henokh hidup bergaul dengan Allah, lalu ia tidak ada lagi, sebab ia telah diangkat oleh Allah.
\verse Setelah Metusalah hidup seratus delapan puluh tujuh tahun, ia memperanakkan Lamekh.
\verse Dan Metusalah masih hidup tujuh ratus delapan puluh dua tahun, setelah ia memperanakkan Lamekh, dan ia memperanakkan anak-anak lelaki dan perempuan.
\verse Jadi Metusalah mencapai umur sembilan ratus enam puluh sembilan tahun, lalu ia mati.
\verse Setelah Lamekh hidup seratus delapan puluh dua tahun, ia memperanakkan seorang anak laki-laki,
\verse dan memberi nama Nuh kepadanya, katanya: "Anak ini akan memberi kepada kita penghiburan dalam pekerjaan kita yang penuh susah payah di tanah yang telah terkutuk oleh TUHAN."
\verse Dan Lamekh masih hidup lima ratus sembilan puluh lima tahun, setelah ia memperanakkan Nuh, dan ia memperanakkan anak-anak lelaki dan perempuan.
\verse Jadi Lamekh mencapai umur tujuh ratus tujuh puluh tujuh tahun, lalu ia mati.
\verse Setelah Nuh berumur lima ratus tahun, ia memperanakkan Sem, Ham dan Yafet.
\end{biblechapter}

\begin{biblechapter} % Kejadian 6
\verseWithHeading{Kehatan manusia} Ketika manusia itu mulai bertambah banyak jumlahnya di muka bumi, dan bagi mereka lahir anak-anak perempuan,
\verse maka anak-anak Allah melihat, bahwa anak-anak perempuan manusia itu cantik-cantik, lalu mereka mengambil isteri dari antara perempuan-perempuan itu, siapa saja yang disukai mereka.
\verse Berfirmanlah TUHAN: "Roh-Ku tidak akan selama-lamanya tinggal di dalam manusia, karena manusia itu adalah daging, tetapi umurnya akan seratus dua puluh tahun saja."
\verse Pada waktu itu orang-orang raksasa ada di bumi, dan juga pada waktu sesudahnya, ketika anak-anak Allah menghampiri anak-anak perempuan manusia, dan perempuan-perempuan itu melahirkan anak bagi mereka; inilah orang-orang yang gagah perkasa di zaman purbakala, orang-orang yang kenamaan.
\verse Ketika dilihat TUHAN, bahwa kejahatan manusia besar di bumi dan bahwa segala kecenderungan hatinya selalu membuahkan kejahatan semata-mata,
\verse maka menyesallah TUHAN, bahwa Ia telah menjadikan manusia di bumi, dan hal itu memilukan hati-Nya.
\verse Berfirmanlah TUHAN: "Aku akan menghapuskan manusia yang telah Kuciptakan itu dari muka bumi, baik manusia maupun hewan dan binatang-binatang melata dan burung-burung di udara, sebab Aku menyesal, bahwa Aku telah menjadikan mereka."
\verse Tetapi Nuh mendapat kasih karunia di mata TUHAN.
\verseWithSubheading{Riwayat Nuh} Inilah riwayat Nuh: Nuh adalah seorang yang benar dan tidak bercela di antara orang-orang sezamannya; dan Nuh itu hidup bergaul dengan Allah.
\verse Nuh memperanakkan tiga orang laki-laki: Sem, Ham dan Yafet.
\verse Adapun bumi itu telah rusak di hadapan Allah dan penuh dengan kekerasan.
\verse Allah menilik bumi itu dan sungguhlah rusak benar, sebab semua manusia menjalankan hidup yang rusak di bumi.
\verse Berfirmanlah Allah kepada Nuh: "Aku telah memutuskan untuk mengakhiri hidup segala makhluk, sebab bumi telah penuh dengan kekerasan oleh mereka, jadi Aku akan memusnahkan mereka bersama-sama dengan bumi.
\verse Buatlah bagimu sebuah bahtera dari kayu gofir; bahtera itu harus kaubuat berpetak-petak dan harus kaututup dengan pakal dari luar dan dari dalam.
\verse Beginilah engkau harus membuat bahtera itu: tiga ratus hasta panjangnya, lima puluh hasta lebarnya dan tiga puluh hasta tingginya.
\verse Buatlah atap pada bahtera itu dan selesaikanlah bahtera itu sampai sehasta dari atas, dan pasanglah pintunya pada lambungnya; buatlah bahtera itu bertingkat bawah, tengah dan atas.
\verse Sebab sesungguhnya Aku akan mendatangkan air bah meliputi bumi untuk memusnahkan segala yang hidup dan bernyawa di kolong langit; segala yang ada di bumi akan mati binasa.
\verse Tetapi dengan engkau Aku akan mengadakan perjanjian-Ku, dan engkau akan masuk ke dalam bahtera itu: engkau bersama-sama dengan anak-anakmu dan isterimu dan isteri anak-anakmu.
\verse Dan dari segala yang hidup, dari segala makhluk, dari semuanya haruslah engkau bawa satu pasang ke dalam bahtera itu, supaya terpelihara hidupnya bersama-sama dengan engkau; jantan dan betina harus kaubawa.
\verse Dari segala jenis burung dan dari segala jenis hewan, dari segala jenis binatang melata di muka bumi, dari semuanya itu harus datang satu pasang kepadamu, supaya terpelihara hidupnya.
\verse Dan engkau, bawalah bagimu segala apa yang dapat dimakan; kumpulkanlah itu padamu untuk menjadi makanan bagimu dan bagi mereka."
\verse Lalu Nuh melakukan semuanya itu; tepat seperti yang diperintahkan Allah kepadanya, demikianlah dilakukannya.
\end{biblechapter}

\begin{biblechapter} % Kejadian 7
\verseWithHeading{Air bah} Lalu berfirmanlah TUHAN kepada Nuh: "Masuklah ke dalam bahtera itu, engkau dan seisi rumahmu, sebab engkaulah yang Kulihat benar di hadapan-Ku di antara orang zaman ini.
\verse Dari segala binatang yang tidak haram haruslah kauambil tujuh pasang, jantan dan betinanya, tetapi dari binatang yang haram satu pasang, jantan dan betinanya;
\verse juga dari burung-burung di udara tujuh pasang, jantan dan betina, supaya terpelihara hidup keturunannya di seluruh bumi.
\verse Sebab tujuh hari lagi Aku akan menurunkan hujan ke atas bumi empat puluh hari empat puluh malam lamanya, dan Aku akan menghapuskan dari muka bumi segala yang ada, yang Kujadikan itu."
\verse Lalu Nuh melakukan segala yang diperintahkan TUHAN kepadanya.
\verse Nuh berumur enam ratus tahun, ketika air bah datang meliputi bumi.
\verse Masuklah Nuh ke dalam bahtera itu bersama-sama dengan anak-anaknya dan isterinya dan isteri anak-anaknya karena air bah itu.
\verse Dari binatang yang tidak haram dan yang haram, dari burung-burung dan dari segala yang merayap di muka bumi,
\verse datanglah sepasang mendapatkan Nuh ke dalam bahtera itu, jantan dan betina, seperti yang diperintahkan Allah kepada Nuh.
\verse Setelah tujuh hari datanglah air bah meliputi bumi.
\verse Pada waktu umur Nuh enam ratus tahun, pada bulan yang kedua, pada hari yang ketujuh belas bulan itu, pada hari itulah terbelah segala mata air samudera raya yang dahsyat dan terbukalah tingkap-tingkap di langit.
\verse Dan turunlah hujan lebat meliputi bumi empat puluh hari empat puluh malam lamanya.
\verse Pada hari itu juga masuklah Nuh serta Sem, Ham dan Yafet, anak-anak Nuh, dan isteri Nuh, dan ketiga isteri anak-anaknya bersama-sama dengan dia, ke dalam bahtera itu,
\verse mereka itu dan segala jenis binatang liar dan segala jenis ternak dan segala jenis binatang melata yang merayap di bumi dan segala jenis burung, yakni segala yang berbulu bersayap;
\verse dari segala yang hidup dan bernyawa datanglah sepasang mendapatkan Nuh ke dalam bahtera itu.
\verse Dan yang masuk itu adalah jantan dan betina dari segala yang hidup, seperti yang diperintahkan Allah kepada Nuh; lalu TUHAN menutup pintu bahtera itu di belakang Nuh.
\verse Empat puluh hari lamanya air bah itu meliputi bumi; air itu naik dan mengangkat bahtera itu, sehingga melampung tinggi dari bumi.
\verse Ketika air itu makin bertambah-tambah dan naik dengan hebatnya di atas bumi, terapung-apunglah bahtera itu di muka air.
\verse Dan air itu sangat hebatnya bertambah-tambah meliputi bumi, dan ditutupinyalah segala gunung tinggi di seluruh kolong langit,
\verse sampai lima belas hasta di atasnya bertambah-tambah air itu, sehingga gunung-gunung ditutupinya.
\verse Lalu mati binasalah segala yang hidup, yang bergerak di bumi, burung-burung, ternak dan binatang liar dan segala binatang merayap, yang berkeriapan di bumi, serta semua manusia.
\verse Matilah segala yang ada nafas hidup dalam hidungnya, segala yang ada di darat.
\verse Demikianlah dihapuskan Allah segala yang ada, segala yang di muka bumi, baik manusia maupun hewan dan binatang melata dan burung-burung di udara, sehingga semuanya itu dihapuskan dari atas bumi; hanya Nuh yang tinggal hidup dan semua yang bersama-sama dengan dia dalam bahtera itu.
\verse Dan berkuasalah air itu di atas bumi seratus lima puluh hari lamanya.
\end{biblechapter}

\begin{biblechapter} % Kejadian 8
\verseWithHeading{Air bah surut} Maka Allah mengingat Nuh dan segala binatang liar dan segala ternak, yang bersama-sama dengan dia dalam bahtera itu, dan Allah membuat angin menghembus melalui bumi, sehingga air itu turun.
\verse Ditutuplah mata-mata air samudera raya serta tingkap-tingkap di langit dan berhentilah hujan lebat dari langit,
\verse dan makin surutlah air itu dari muka bumi. Demikianlah berkurang air itu sesudah seratus lima puluh hari.
\verse Dalam bulan yang ketujuh, pada hari yang ketujuh belas bulan itu, terkandaslah bahtera itu pada pegunungan Ararat.
\verse Sampai bulan yang kesepuluh makin berkuranglah air itu; dalam bulan yang kesepuluh, pada tanggal satu bulan itu, tampaklah puncak-puncak gunung.
\verse Sesudah lewat empat puluh hari, maka Nuh membuka tingkap yang dibuatnya pada bahtera itu.
\verse Lalu ia melepaskan seekor burung gagak; dan burung itu terbang pulang pergi, sampai air itu menjadi kering dari atas bumi.
\verse Kemudian dilepaskannya seekor burung merpati untuk melihat, apakah air itu telah berkurang dari muka bumi.
\verse Tetapi burung merpati itu tidak mendapat tempat tumpuan kakinya dan pulanglah ia kembali mendapatkan Nuh ke dalam bahtera itu, karena di seluruh bumi masih ada air; lalu Nuh mengulurkan tangannya, ditangkapnya burung itu dan dibawanya masuk ke dalam bahtera.
\verse Ia menunggu tujuh hari lagi, kemudian dilepaskannya pula burung merpati itu dari bahtera;
\verse menjelang waktu senja pulanglah burung merpati itu mendapatkan Nuh, dan pada paruhnya dibawanya sehelai daun zaitun yang segar. Dari situlah diketahui Nuh, bahwa air itu telah berkurang dari atas bumi.
\verse Selanjutnya ditunggunya pula tujuh hari lagi, kemudian dilepaskannya burung merpati itu, tetapi burung itu tidak kembali lagi kepadanya.
\verse Dalam tahun keenam ratus satu, dalam bulan pertama, pada tanggal satu bulan itu, sudahlah kering air itu dari atas bumi; kemudian Nuh membuka tutup bahtera itu dan melihat-lihat; ternyatalah muka bumi sudah mulai kering.
\verse Dalam bulan kedua, pada hari yang kedua puluh tujuh bulan itu, bumi telah kering.
\verse Lalu berfirmanlah Allah kepada Nuh:
\verse "Keluarlah dari bahtera itu, engkau bersama-sama dengan isterimu serta anak-anakmu dan isteri anak-anakmu;
\verse segala binatang yang bersama-sama dengan engkau, segala yang hidup: burung-burung, hewan dan segala binatang melata yang merayap di bumi, suruhlah keluar bersama-sama dengan engkau, supaya semuanya itu berkeriapan di bumi serta berkembang biak dan bertambah banyak di bumi."
\verse Lalu keluarlah Nuh bersama-sama dengan anak-anaknya dan isterinya dan isteri anak-anaknya.
\verse Segala binatang liar, segala binatang melata dan segala burung, semuanya yang bergerak di bumi, masing-masing menurut jenisnya, keluarlah juga dari bahtera itu.
\verse Lalu Nuh mendirikan mezbah bagi TUHAN; dari segala binatang yang tidak haram dan dari segala burung yang tidak haram diambilnyalah beberapa ekor, lalu ia mempersembahkan korban bakaran di atas mezbah itu.
\verse Ketika TUHAN mencium persembahan yang harum itu, berfirmanlah TUHAN dalam hati-Nya: "Aku takkan mengutuk bumi ini lagi karena manusia, sekalipun yang ditimbulkan hatinya adalah jahat dari sejak kecilnya, dan Aku takkan membinasakan lagi segala yang hidup seperti yang telah Kulakukan.
\verse Selama bumi masih ada, takkan berhenti-henti musim menabur dan menuai, dingin dan panas, kemarau dan hujan, siang dan malam."
\end{biblechapter}

\begin{biblechapter} % Kejadian 9
\verseWithHeading{Perjanjian Allah dengan Nuh} Lalu Allah memberkati Nuh dan anak-anaknya serta berfirman kepada mereka: "Beranakcuculah dan bertambah banyaklah serta penuhilah bumi.
\verse Akan takut dan akan gentar kepadamu segala binatang di bumi dan segala burung di udara, segala yang bergerak di muka bumi dan segala ikan di laut; ke dalam tanganmulah semuanya itu diserahkan.
\verse Segala yang bergerak, yang hidup, akan menjadi makananmu. Aku telah memberikan semuanya itu kepadamu seperti juga tumbuh-tumbuhan hijau.
\verse Hanya daging yang masih ada nyawanya, yakni darahnya, janganlah kamu makan.
\verse Tetapi mengenai darah kamu, yakni nyawa kamu, Aku akan menuntut balasnya; dari segala binatang Aku akan menuntutnya, dan dari setiap manusia Aku akan menuntut nyawa sesama manusia.
\verse Siapa yang menumpahkan darah manusia, darahnya akan tertumpah oleh manusia, sebab Allah membuat manusia itu menurut gambar-Nya sendiri.
\verse Dan kamu, beranakcuculah dan bertambah banyak, sehingga tak terbilang jumlahmu di atas bumi, ya, bertambah banyaklah di atasnya."
\verse Berfirmanlah Allah kepada Nuh dan kepada anak-anaknya yang bersama-sama dengan dia:
\verse "Sesungguhnya Aku mengadakan perjanjian-Ku dengan kamu dan dengan keturunanmu,
\verse dan dengan segala makhluk hidup yang bersama-sama dengan kamu: burung-burung, ternak dan binatang-binatang liar di bumi yang bersama-sama dengan kamu, segala yang keluar dari bahtera itu, segala binatang di bumi.
\verse Maka Kuadakan perjanjian-Ku dengan kamu, bahwa sejak ini tidak ada yang hidup yang akan dilenyapkan oleh air bah lagi, dan tidak akan ada lagi air bah untuk memusnahkan bumi."
\verse Dan Allah berfirman: "Inilah tanda perjanjian yang Kuadakan antara Aku dan kamu serta segala makhluk yang hidup, yang bersama-sama dengan kamu, turun-temurun, untuk selama-lamanya:
\verse Busur-Ku Kutaruh di awan, supaya itu menjadi tanda perjanjian antara Aku dan bumi.
\verse Apabila kemudian Kudatangkan awan di atas bumi dan busur itu tampak di awan,
\verse maka Aku akan mengingat perjanjian-Ku yang telah ada antara Aku dan kamu serta segala makhluk yang hidup, segala yang bernyawa, sehingga segenap air tidak lagi menjadi air bah untuk memusnahkan segala yang hidup.
\verse Jika busur itu ada di awan, maka Aku akan melihatnya, sehingga Aku mengingat perjanjian-Ku yang kekal antara Allah dan segala makhluk yang hidup, segala makhluk yang ada di bumi."
\verse Berfirmanlah Allah kepada Nuh: "Inilah tanda perjanjian yang Kuadakan antara Aku dan segala makhluk yang ada di bumi."
\verseWithSubheading{Nuh dan anak-anaknya} Anak-anak Nuh yang keluar dari bahtera ialah Sem, Ham dan Yafet; Ham adalah bapa Kanaan.
\verse Yang tiga inilah anak-anak Nuh, dan dari mereka inilah tersebar penduduk seluruh bumi.
\verse Nuh menjadi petani; dialah yang mula-mula membuat kebun anggur.
\verse Setelah ia minum anggur, mabuklah ia dan ia telanjang dalam kemahnya.
\verse Maka Ham, bapa Kanaan itu, melihat aurat ayahnya, lalu diceritakannya kepada kedua saudaranya di luar.
\verse Sesudah itu Sem dan Yafet mengambil sehelai kain dan membentangkannya pada bahu mereka berdua, lalu mereka berjalan mundur; mereka menutupi aurat ayahnya sambil berpaling muka, sehingga mereka tidak melihat aurat ayahnya.
\verse Setelah Nuh sadar dari mabuknya dan mendengar apa yang dilakukan anak bungsunya kepadanya,
\verse berkatalah ia: "Terkutuklah Kanaan, hendaklah ia menjadi hamba yang paling hina bagi saudara-saudaranya."
\verse Lagi katanya: "Terpujilah TUHAN, Allah Sem, tetapi hendaklah Kanaan menjadi hamba baginya.
\verse Allah meluaskan kiranya tempat kediaman Yafet, dan hendaklah ia tinggal dalam kemah-kemah Sem, tetapi hendaklah Kanaan menjadi hamba baginya."
\verse Nuh masih hidup tiga ratus lima puluh tahun sesudah air bah.
\verse Jadi Nuh mencapai umur sembilan ratus lima puluh tahun, lalu ia mati.
\end{biblechapter}

\begin{biblechapter} % Kejadian 10
\verseWithHeading{Daftar bangsa-bangsa keturunan Sem, Ham dan Yafet} Inilah keturunan Sem, Ham dan Yafet, anak-anak Nuh. Setelah air bah itu lahirlah anak-anak lelaki bagi mereka.
\verse Keturunan Yafet ialah Gomer, Magog, Madai, Yawan, Tubal, Mesekh dan Tiras.
\verse Keturunan Gomer ialah Askenas, Rifat dan Togarma.
\verse Keturunan Yawan ialah Elisa, Tarsis, orang Kitim dan orang Dodanim.
\verse Dari mereka inilah berpencar bangsa-bangsa daerah pesisir. Itulah keturunan Yafet, masing-masing di tanahnya, dengan bahasanya sendiri, menurut kaum dan bangsa mereka.
\verse Keturunan Ham ialah Kush, Misraim, Put dan Kanaan.
\verse Keturunan Kush ialah Seba, Hawila, Sabta, Raema dan Sabtekha; anak-anak Raema ialah Syeba dan Dedan.
\verse Kush memperanakkan Nimrod; dialah yang mula-mula sekali orang yang berkuasa di bumi;
\verse ia seorang pemburu yang gagah perkasa di hadapan TUHAN, sebab itu dikatakan orang: "Seperti Nimrod, seorang pemburu yang gagah perkasa di hadapan TUHAN."
\verse Mula-mula kerajaannya terdiri dari Babel, Erekh, dan Akad, semuanya di tanah Sinear.
\verse Dari negeri itu ia pergi ke Asyur, lalu mendirikan Niniwe, Rehobot-Ir, Kalah
\verse dan Resen di antara Niniwe dan Kalah; itulah kota besar itu.
\verse Misraim memperanakkan orang Ludim, orang Anamim, orang Lehabim, orang Naftuhim,
\verse orang Patrusim, orang Kasluhim dan orang Kaftorim; dari mereka inilah berasal orang Filistin.
\verse Kanaan memperanakkan Sidon, anak sulungnya, dan Het,
\verse serta orang Yebusi, orang Amori dan orang Girgasi;
\verse orang Hewi, orang Arki, orang Sini,
\verse orang Arwadi, orang Semari dan orang Hamati; kemudian berseraklah kaum-kaum orang Kanaan itu.
\verse Daerah orang Kanaan adalah dari Sidon ke arah Gerar sampai ke Gaza, ke arah Sodom, Gomora, Adma dan Zeboim sampai ke Lasa.
\verse Itulah keturunan Ham menurut kaum mereka, menurut bahasa mereka, menurut tanah mereka, menurut bangsa mereka.
\verse Lahirlah juga anak-anak bagi Sem, bapa semua anak Eber serta abang Yafet.
\verse Keturunan Sem ialah Elam, Asyur, Arpakhsad, Lud dan Aram.
\verse Keturunan Aram ialah Us, Hul, Geter dan Mas.
\verse Arpakhsad memperanakkan Selah, dan Selah memperanakkan Eber.
\verse Bagi Eber lahir dua anak laki-laki; nama yang seorang ialah Peleg, sebab dalam zamannya bumi terbagi, dan nama adiknya ialah Yoktan.
\verse Yoktan memperanakkan Almodad, Selef, Hazar-Mawet dan Yerah,
\verse Hadoram, Uzal dan Dikla,
\verse Obal, Abimael dan Syeba,
\verse Ofir, Hawila dan Yobab; itulah semuanya keturunan Yoktan.
\verse Daerah kediaman mereka terbentang dari Mesa ke arah Sefar, yaitu pegunungan di sebelah timur.
\verse Itulah keturunan Sem, menurut kaum mereka, menurut bahasa mereka, menurut tanah mereka, menurut bangsa mereka.
\verse Itulah segala kaum anak-anak Nuh menurut keturunan mereka, menurut bangsa mereka. Dan dari mereka itulah berpencar bangsa-bangsa di bumi setelah air bah itu.
\end{biblechapter}

\begin{biblechapter} % Kejadian 11
\verseWithHeading{Menara Babel} Adapun seluruh bumi, satu bahasanya dan satu logatnya.
\verse Maka berangkatlah mereka ke sebelah timur dan menjumpai tanah datar di tanah Sinear, lalu menetaplah mereka di sana.
\verse Mereka berkata seorang kepada yang lain: "Marilah kita membuat batu bata dan membakarnya baik-baik." Lalu bata itulah dipakai mereka sebagai batu dan tér gala-gala sebagai tanah liat.
\verse Juga kata mereka: "Marilah kita dirikan bagi kita sebuah kota dengan sebuah menara yang puncaknya sampai ke langit, dan marilah kita cari nama, supaya kita jangan terserak ke seluruh bumi."
\verse Lalu turunlah TUHAN untuk melihat kota dan menara yang didirikan oleh anak-anak manusia itu,
\verse dan Ia berfirman: "Mereka ini satu bangsa dengan satu bahasa untuk semuanya. Ini barulah permulaan usaha mereka; mulai dari sekarang apa pun juga yang mereka rencanakan, tidak ada yang tidak akan dapat terlaksana.
\verse Baiklah Kita turun dan mengacaubalaukan di sana bahasa mereka, sehingga mereka tidak mengerti lagi bahasa masing-masing."
\verse Demikianlah mereka diserakkan TUHAN dari situ ke seluruh bumi, dan mereka berhenti mendirikan kota itu.
\verse Itulah sebabnya sampai sekarang nama kota itu disebut Babel, karena di situlah dikacaubalaukan TUHAN bahasa seluruh bumi dan dari situlah mereka diserakkan TUHAN ke seluruh bumi.
\verseWithSubheading{Keturunan Sem} Inilah keturunan Sem. Setelah Sem berumur seratus tahun, ia memperanakkan Arpakhsad, dua tahun setelah air bah itu.
\verse Sem masih hidup lima ratus tahun, setelah ia memperanakkan Arpakhsad, dan ia memperanakkan anak-anak lelaki dan perempuan.
\verse Setelah Arpakhsad hidup tiga puluh lima tahun, ia memperanakkan Selah.
\verse Arpakhsad masih hidup empat ratus tiga tahun, setelah ia memperanakkan Selah, dan ia memperanakkan anak-anak lelaki dan perempuan.
\verse Setelah Selah hidup tiga puluh tahun, ia memperanakkan Eber.
\verse Selah masih hidup empat ratus tiga tahun, setelah ia memperanakkan Eber, dan ia memperanakkan anak-anak lelaki dan perempuan.
\verse Setelah Eber hidup tiga puluh empat tahun, ia memperanakkan Peleg.
\verse Eber masih hidup empat ratus tiga puluh tahun, setelah ia memperanakkan Peleg, dan ia memperanakkan anak-anak lelaki dan perempuan.
\verse Setelah Peleg hidup tiga puluh tahun, ia memperanakkan Rehu.
\verse Peleg masih hidup dua ratus sembilan tahun, setelah ia memperanakkan Rehu, dan ia memperanakkan anak-anak lelaki dan perempuan.
\verse Setelah Rehu hidup tiga puluh dua tahun, ia memperanakkan Serug.
\verse Rehu masih hidup dua ratus tujuh tahun, setelah ia memperanakkan Serug, dan ia memperanakkan anak-anak lelaki dan perempuan.
\verse Setelah Serug hidup tiga puluh tahun, ia memperanakkan Nahor.
\verse Serug masih hidup dua ratus tahun, setelah ia memperanakkan Nahor, dan ia memperanakkan anak-anak lelaki dan perempuan.
\verse Setelah Nahor hidup dua puluh sembilan tahun, ia memperanakkan Terah.
\verse Nahor masih hidup seratus sembilan belas tahun, setelah ia memperanakkan Terah, dan ia memperanakkan anak-anak lelaki dan perempuan.
\verse Setelah Terah hidup tujuh puluh tahun, ia memperanakkan Abram, Nahor dan Haran.
\verseWithSubheading{Keturunan Terah} Inilah keturunan Terah. Terah memperanakkan Abram, Nahor dan Haran, dan Haran memperanakkan Lot.
\verse Ketika Terah, ayahnya, masih hidup, matilah Haran di negeri kelahirannya, di Ur-Kasdim.
\verse Abram dan Nahor kedua-duanya kawin; nama isteri Abram ialah Sarai, dan nama isteri Nahor ialah Milka, anak Haran ayah Milka dan Yiska.
\verse Sarai itu mandul, tidak mempunyai anak.
\verse Lalu Terah membawa Abram, anaknya, serta cucunya, Lot, yaitu anak Haran, dan Sarai, menantunya, isteri Abram, anaknya; ia berangkat bersama-sama dengan mereka dari Ur-Kasdim untuk pergi ke tanah Kanaan, lalu sampailah mereka ke Haran, dan menetap di sana.
\verse Umur Terah ada dua ratus lima tahun; lalu ia mati di Haran.
\end{biblechapter}

\begin{biblechapter} % Kejadian 12
\verseWithHeading{Abram dipanggil Allah} Berfirmanlah TUHAN kepada Abram: "Pergilah dari negerimu dan dari sanak saudaramu dan dari rumah bapamu ini ke negeri yang akan Kutunjukkan kepadamu;
\verse Aku akan membuat engkau menjadi bangsa yang besar, dan memberkati engkau serta membuat namamu masyhur; dan engkau akan menjadi berkat.
\verse Aku akan memberkati orang-orang yang memberkati engkau, dan mengutuk orang-orang yang mengutuk engkau, dan olehmu semua kaum di muka bumi akan mendapat berkat."
\verse Lalu pergilah Abram seperti yang difirmankan TUHAN kepadanya, dan Lot pun ikut bersama-sama dengan dia; Abram berumur tujuh puluh lima tahun, ketika ia berangkat dari Haran.
\verse Abram membawa Sarai, isterinya, dan Lot, anak saudaranya, dan segala harta benda yang didapat mereka dan orang-orang yang diperoleh mereka di Haran; mereka berangkat ke tanah Kanaan, lalu sampai di situ.
\verse Abram berjalan melalui negeri itu sampai ke suatu tempat dekat Sikhem, yakni pohon tarbantin di More. Waktu itu orang Kanaan diam di negeri itu.
\verse Ketika itu TUHAN menampakkan diri kepada Abram dan berfirman: "Aku akan memberikan negeri ini kepada keturunanmu." Maka didirikannya di situ mezbah bagi TUHAN yang telah menampakkan diri kepadanya.
\verse Kemudian ia pindah dari situ ke pegunungan di sebelah timur Betel. Ia memasang kemahnya dengan Betel di sebelah barat dan Ai di sebelah timur, lalu ia mendirikan di situ mezbah bagi TUHAN dan memanggil nama TUHAN.
\verse Sesudah itu Abram berangkat dan makin jauh ia berjalan ke Tanah Negeb.
\verseWithSubheading{Abram di Mesir} Ketika kelaparan timbul di negeri itu, pergilah Abram ke Mesir untuk tinggal di situ sebagai orang asing, sebab hebat kelaparan di negeri itu.
\verse Pada waktu ia akan masuk ke Mesir, berkatalah ia kepada Sarai, isterinya: "Memang aku tahu, bahwa engkau adalah seorang perempuan yang cantik parasnya.
\verse Apabila orang Mesir melihat engkau, mereka akan berkata: Itu isterinya. Jadi mereka akan membunuh aku dan membiarkan engkau hidup.
\verse Katakanlah, bahwa engkau adikku, supaya aku diperlakukan mereka dengan baik karena engkau, dan aku dibiarkan hidup oleh sebab engkau."
\verse Sesudah Abram masuk ke Mesir, orang Mesir itu melihat, bahwa perempuan itu sangat cantik,
\verse dan ketika punggawa-punggawa Firaun melihat Sarai, mereka memuji-mujinya di hadapan Firaun, sehingga perempuan itu dibawa ke istananya.
\verse Firaun menyambut Abram dengan baik-baik, karena ia mengingini perempuan itu, dan Abram mendapat kambing domba, lembu sapi, keledai jantan, budak laki-laki dan perempuan, keledai betina dan unta.
\verse Tetapi TUHAN menimpakan tulah yang hebat kepada Firaun, demikian juga kepada seisi istananya, karena Sarai, isteri Abram itu.
\verse Lalu Firaun memanggil Abram serta berkata: "Apakah yang kauperbuat ini terhadap aku? Mengapa tidak kauberitahukan, bahwa ia isterimu?
\verse Mengapa engkau katakan: dia adikku, sehingga aku mengambilnya menjadi isteriku? Sekarang, inilah isterimu, ambillah dan pergilah!"
\verse Lalu Firaun memerintahkan beberapa orang untuk mengantarkan Abram pergi, bersama-sama dengan isterinya dan segala kepunyaannya.
\end{biblechapter}

\begin{biblechapter} % Kejadian 13
\verseWithHeading{Abram dan Lot berpisah} Maka pergilah Abram dari Mesir ke Tanah Negeb dengan isterinya dan segala kepunyaannya, dan Lot pun bersama-sama dengan dia.
\verse Adapun Abram sangat kaya, banyak ternak, perak dan emasnya.
\verse Ia berjalan dari tempat persinggahan ke tempat persinggahan, dari Tanah Negeb sampai dekat Betel, di mana kemahnya mula-mula berdiri, antara Betel dan Ai,
\verse ke tempat mezbah yang dibuatnya dahulu di sana; di situlah Abram memanggil nama TUHAN.
\verse Juga Lot, yang ikut bersama-sama dengan Abram, mempunyai domba dan lembu dan kemah.
\verse Tetapi negeri itu tidak cukup luas bagi mereka untuk diam bersama-sama, sebab harta milik mereka amat banyak, sehingga mereka tidak dapat diam bersama-sama.
\verse Karena itu terjadilah perkelahian antara para gembala Abram dan para gembala Lot. Waktu itu orang Kanaan dan orang Feris diam di negeri itu.
\verse Maka berkatalah Abram kepada Lot: "Janganlah kiranya ada perkelahian antara aku dan engkau, dan antara para gembalaku dan para gembalamu, sebab kita ini kerabat.
\verse Bukankah seluruh negeri ini terbuka untuk engkau? Baiklah pisahkan dirimu dari padaku; jika engkau ke kiri, maka aku ke kanan, jika engkau ke kanan, maka aku ke kiri."
\verse Lalu Lot melayangkan pandangnya dan dilihatnyalah, bahwa seluruh Lembah Yordan banyak airnya, seperti taman TUHAN, seperti tanah Mesir, sampai ke Zoar. -- Hal itu terjadi sebelum TUHAN memusnahkan Sodom dan Gomora. --
\verse Sebab itu Lot memilih baginya seluruh Lembah Yordan itu, lalu ia berangkat ke sebelah timur dan mereka berpisah.
\verse Abram menetap di tanah Kanaan, tetapi Lot menetap di kota-kota Lembah Yordan dan berkemah di dekat Sodom.
\verse Adapun orang Sodom sangat jahat dan berdosa terhadap TUHAN.
\verse Setelah Lot berpisah dari pada Abram, berfirmanlah TUHAN kepada Abram: "Pandanglah sekelilingmu dan lihatlah dari tempat engkau berdiri itu ke timur dan barat, utara dan selatan,
\verse sebab seluruh negeri yang kaulihat itu akan Kuberikan kepadamu dan kepada keturunanmu untuk selama-lamanya.
\verse Dan Aku akan menjadikan keturunanmu seperti debu tanah banyaknya, sehingga, jika seandainya ada yang dapat menghitung debu tanah, keturunanmu pun akan dapat dihitung juga.
\verse Bersiaplah, jalanilah negeri itu menurut panjang dan lebarnya, sebab kepadamulah akan Kuberikan negeri itu."
\verse Sesudah itu Abram memindahkan kemahnya dan menetap di dekat pohon-pohon tarbantin di Mamre, dekat Hebron, lalu didirikannyalah mezbah di situ bagi TUHAN.
\end{biblechapter}

\begin{biblechapter} % Kejadian 14
\verseWithHeading{Abram mengalahkan raja-raja di Timur dan menolong Lot} Pada zaman Amrafel, raja Sinear, Ariokh, raja Elasar, Kedorlaomer, raja Elam, dan Tideal, raja Goyim, terjadilah,
\verse bahwa raja-raja ini berperang melawan Bera, raja Sodom, Birsya, raja Gomora, Syinab, raja Adma, Syemeber, raja Zeboim dan raja negeri Bela, yakni negeri Zoar.
\verse Raja-raja yang disebut terakhir ini semuanya bersekutu dan datang ke lembah Sidim, yakni Laut Asin.
\verse Dua belas tahun lamanya mereka takluk kepada Kedorlaomer, tetapi dalam tahun yang ketiga belas mereka memberontak.
\verse Dalam tahun yang keempat belas datanglah Kedorlaomer serta raja-raja yang bersama-sama dengan dia, lalu mereka mengalahkan orang Refaim di Asyterot-Karnaim, orang Zuzim di Ham, orang Emim di Syawe-Kiryataim
\verse dan orang Hori di pegunungan mereka yang bernama Seir, sampai ke El-Paran di tepi padang gurun.
\verse Sesudah itu baliklah mereka dan sampai ke En-Mispat, yakni Kadesh, dan mengalahkan seluruh daerah orang Amalek, dan juga orang Amori, yang diam di Hazezon-Tamar.
\verse Lalu keluarlah raja negeri Sodom, raja negeri Gomora, raja negeri Adma, raja negeri Zeboim dan raja negeri Bela, yakni negeri Zoar, dan mengatur barisan perangnya melawan mereka di lembah Sidim,
\verse melawan Kedorlaomer, raja Elam, Tideal, raja Goyim, Amrafel, raja Sinear, dan Ariokh, raja Elasar, empat raja lawan lima.
\verse Di lembah Sidim itu di mana-mana ada sumur aspal. Ketika raja Sodom dan raja Gomora melarikan diri, jatuhlah mereka ke dalamnya, dan orang-orang yang masih tinggal hidup melarikan diri ke pegunungan.
\verse Segala harta benda Sodom dan Gomora beserta segala bahan makanan dirampas musuh, lalu mereka pergi.
\verse Juga Lot, anak saudara Abram, beserta harta bendanya, dibawa musuh, lalu mereka pergi -- sebab Lot itu diam di Sodom.
\verse Kemudian datanglah seorang pelarian dan menceritakan hal ini kepada Abram, orang Ibrani itu, yang tinggal dekat pohon-pohon tarbantin kepunyaan Mamre, orang Amori itu, saudara Eskol dan Aner, yakni teman-teman sekutu Abram.
\verse Ketika Abram mendengar, bahwa anak saudaranya tertawan, maka dikerahkannyalah orang-orangnya yang terlatih, yakni mereka yang lahir di rumahnya, tiga ratus delapan belas orang banyaknya, lalu mengejar musuh sampai ke Dan.
\verse Dan pada waktu malam berbagilah mereka, ia dan hamba-hambanya itu, untuk melawan musuh; mereka mengalahkan dan mengejar musuh sampai ke Hoba di sebelah utara Damsyik.
\verse Dibawanyalah kembali segala harta benda itu; juga Lot, anak saudaranya itu, serta harta bendanya dibawanya kembali, demikian juga perempuan-perempuan dan orang-orangnya.
\verseWithSubheading{Pertemuan Abram dengan Melkisedek} Setelah Abram kembali dari mengalahkan Kedorlaomer dan para raja yang bersama-sama dengan dia, maka keluarlah raja Sodom menyongsong dia ke lembah Syawe, yakni Lembah Raja.
\verse Melkisedek, raja Salem, membawa roti dan anggur; ia seorang imam Allah Yang Mahatinggi.
\verse Lalu ia memberkati Abram, katanya: "Diberkatilah kiranya Abram oleh Allah Yang Mahatinggi, Pencipta langit dan bumi,
\verse dan terpujilah Allah Yang Mahatinggi, yang telah menyerahkan musuhmu ke tanganmu." Lalu Abram memberikan kepadanya sepersepuluh dari semuanya.
\verse Berkatalah raja Sodom itu kepada Abram: "Berikanlah kepadaku orang-orang itu, dan ambillah untukmu harta benda itu."
\verse Tetapi kata Abram kepada raja negeri Sodom itu: "Aku bersumpah demi TUHAN, Allah Yang Mahatinggi, Pencipta langit dan bumi:
\verse Aku tidak akan mengambil apa-apa dari kepunyaanmu itu, sepotong benang atau tali kasut pun tidak, supaya engkau jangan dapat berkata: Aku telah membuat Abram menjadi kaya.
\verse Kalau aku, jangan sekali-kali! Hanya apa yang telah dimakan oleh bujang-bujang ini dan juga bagian orang-orang yang pergi bersama-sama dengan aku, yakni Aner, Eskol dan Mamre, biarlah mereka itu mengambil bagiannya masing-masing."
\end{biblechapter}

\begin{biblechapter} % Kejadian 15
\verseWithHeading{Perjanjian Allah dengan Abram; janji tentang keturunannya} Kemudian datanglah firman TUHAN kepada Abram dalam suatu penglihatan: "Janganlah takut, Abram, Akulah perisaimu; upahmu akan sangat besar."
\verse Abram menjawab: "Ya Tuhan ALLAH, apakah yang akan Engkau berikan kepadaku, karena aku akan meninggal dengan tidak mempunyai anak, dan yang akan mewarisi rumahku ialah Eliezer, orang Damsyik itu."
\verse Lagi kata Abram: "Engkau tidak memberikan kepadaku keturunan, sehingga seorang hambaku nanti menjadi ahli warisku."
\verse Tetapi datanglah firman TUHAN kepadanya, demikian: "Orang ini tidak akan menjadi ahli warismu, melainkan anak kandungmu, dialah yang akan menjadi ahli warismu."
\verse Lalu TUHAN membawa Abram ke luar serta berfirman: "Coba lihat ke langit, hitunglah bintang-bintang, jika engkau dapat menghitungnya." Maka firman-Nya kepadanya: "Demikianlah banyaknya nanti keturunanmu."
\verse Lalu percayalah Abram kepada TUHAN, maka TUHAN memperhitungkan hal itu kepadanya sebagai kebenaran.
\verse Lagi firman TUHAN kepadanya: "Akulah TUHAN, yang membawa engkau keluar dari Ur-Kasdim untuk memberikan negeri ini kepadamu menjadi milikmu."
\verse Kata Abram: "Ya Tuhan ALLAH, dari manakah aku tahu, bahwa aku akan memilikinya?"
\verse Firman TUHAN kepadanya: "Ambillah bagi-Ku seekor lembu betina berumur tiga tahun, seekor kambing betina berumur tiga tahun, seekor domba jantan berumur tiga tahun, seekor burung tekukur dan seekor anak burung merpati."
\verse Diambilnyalah semuanya itu bagi TUHAN, dipotong dua, lalu diletakkannya bagian-bagian itu yang satu di samping yang lain, tetapi burung-burung itu tidak dipotong dua.
\verse Ketika burung-burung buas hinggap pada daging binatang-binatang itu, maka Abram mengusirnya.
\verse Menjelang matahari terbenam, tertidurlah Abram dengan nyenyak. Lalu turunlah meliputinya gelap gulita yang mengerikan.
\verse Firman TUHAN kepada Abram: "Ketahuilah dengan sesungguhnya bahwa keturunanmu akan menjadi orang asing dalam suatu negeri, yang bukan kepunyaan mereka, dan bahwa mereka akan diperbudak dan dianiaya, empat ratus tahun lamanya.
\verse Tetapi bangsa yang akan memperbudak mereka, akan Kuhukum, dan sesudah itu mereka akan keluar dengan membawa harta benda yang banyak.
\verse Tetapi engkau akan pergi kepada nenek moyangmu dengan sejahtera; engkau akan dikuburkan pada waktu telah putih rambutmu.
\verse Tetapi keturunan yang keempat akan kembali ke sini, sebab sebelum itu kedurjanaan orang Amori itu belum genap."
\verse Ketika matahari telah terbenam, dan hari menjadi gelap, maka kelihatanlah perapian yang berasap beserta suluh yang berapi lewat di antara potongan-potongan daging itu.
\verse Pada hari itulah TUHAN mengadakan perjanjian dengan Abram serta berfirman: "Kepada keturunanmulah Kuberikan negeri ini, mulai dari sungai Mesir sampai ke sungai yang besar itu, sungai Efrat:
\verse yakni tanah orang Keni, orang Kenas, orang Kadmon,
\verse orang Het, orang Feris, orang Refaim,
\verse orang Amori, orang Kanaan, orang Girgasi dan orang Yebus itu."
\end{biblechapter}

\begin{biblechapter} % Kejadian 16
\verseWithHeading{Hagar dan Ismael} Adapun Sarai, isteri Abram itu, tidak beranak. Ia mempunyai seorang hamba perempuan, orang Mesir, Hagar namanya.
\verse Berkatalah Sarai kepada Abram: "Engkau tahu, TUHAN tidak memberi aku melahirkan anak. Karena itu baiklah hampiri hambaku itu; mungkin oleh dialah aku dapat memperoleh seorang anak." Dan Abram mendengarkan perkataan Sarai.
\verse Jadi Sarai, isteri Abram itu, mengambil Hagar, hambanya, orang Mesir itu, -- yakni ketika Abram telah sepuluh tahun tinggal di tanah Kanaan --, lalu memberikannya kepada Abram, suaminya, untuk menjadi isterinya.
\verse Abram menghampiri Hagar, lalu mengandunglah perempuan itu. Ketika Hagar tahu, bahwa ia mengandung, maka ia memandang rendah akan nyonyanya itu.
\verse Lalu berkatalah Sarai kepada Abram: "Penghinaan yang kuderita ini adalah tanggung jawabmu; akulah yang memberikan hambaku ke pangkuanmu, tetapi baru saja ia tahu, bahwa ia mengandung, ia memandang rendah akan aku; TUHAN kiranya yang menjadi Hakim antara aku dan engkau."
\verse Kata Abram kepada Sarai: "Hambamu itu di bawah kekuasaanmu; perbuatlah kepadanya apa yang kaupandang baik." Lalu Sarai menindas Hagar, sehingga ia lari meninggalkannya.
\verse Lalu Malaikat TUHAN menjumpainya dekat suatu mata air di padang gurun, yakni dekat mata air di jalan ke Syur.
\verse Katanya: "Hagar, hamba Sarai, dari manakah datangmu dan ke manakah pergimu?" Jawabnya: "Aku lari meninggalkan Sarai, nyonyaku."
\verse Lalu kata Malaikat TUHAN itu kepadanya: "Kembalilah kepada nyonyamu, biarkanlah engkau ditindas di bawah kekuasaannya."
\verse Lagi kata Malaikat TUHAN itu kepadanya: "Aku akan membuat sangat banyak keturunanmu, sehingga tidak dapat dihitung karena banyaknya."
\verse Selanjutnya kata Malaikat TUHAN itu kepadanya: "Engkau mengandung dan akan melahirkan seorang anak laki-laki dan akan menamainya Ismael, sebab TUHAN telah mendengar tentang penindasan atasmu itu.
\verse Seorang laki-laki yang lakunya seperti keledai liar, demikianlah nanti anak itu; tangannya akan melawan tiap-tiap orang dan tangan tiap-tiap orang akan melawan dia, dan di tempat kediamannya ia akan menentang semua saudaranya."
\verse Kemudian Hagar menamakan TUHAN yang telah berfirman kepadanya itu dengan sebutan: "Engkaulah El-Roi." Sebab katanya: "Bukankah di sini kulihat Dia yang telah melihat aku?"
\verse Sebab itu sumur tadi disebutkan orang: sumur Lahai-Roi; letaknya antara Kadesh dan Bered.
\verse Lalu Hagar melahirkan seorang anak laki-laki bagi Abram dan Abram menamai anak yang dilahirkan Hagar itu Ismael.
\verse Abram berumur delapan puluh enam tahun, ketika Hagar melahirkan Ismael baginya.
\end{biblechapter}

\begin{biblechapter} % Kejadian 17
\verseWithHeading{Sunat sebagai tanda perjanjian Allah dengan Abraham} Ketika Abram berumur sembilan puluh sembilan tahun, maka TUHAN menampakkan diri kepada Abram dan berfirman kepadanya: "Akulah Allah Yang Mahakuasa, hiduplah di hadapan-Ku dengan tidak bercela.
\verse Aku akan mengadakan perjanjian antara Aku dan engkau, dan Aku akan membuat engkau sangat banyak."
\verse Lalu sujudlah Abram, dan Allah berfirman kepadanya:
\verse "Dari pihak-Ku, inilah perjanjian-Ku dengan engkau: Engkau akan menjadi bapa sejumlah besar bangsa.
\verse Karena itu namamu bukan lagi Abram, melainkan Abraham, karena engkau telah Kutetapkan menjadi bapa sejumlah besar bangsa.
\verse Aku akan membuat engkau beranak cucu sangat banyak; engkau akan Kubuat menjadi bangsa-bangsa, dan dari padamu akan berasal raja-raja.
\verse Aku akan mengadakan perjanjian antara Aku dan engkau serta keturunanmu turun-temurun menjadi perjanjian yang kekal, supaya Aku menjadi Allahmu dan Allah keturunanmu.
\verse Kepadamu dan kepada keturunanmu akan Kuberikan negeri ini yang kaudiami sebagai orang asing, yakni seluruh tanah Kanaan akan Kuberikan menjadi milikmu untuk selama-lamanya; dan Aku akan menjadi Allah mereka."
\verse Lagi firman Allah kepada Abraham: "Dari pihakmu, engkau harus memegang perjanjian-Ku, engkau dan keturunanmu turun-temurun.
\verse Inilah perjanjian-Ku, yang harus kamu pegang, perjanjian antara Aku dan kamu serta keturunanmu, yaitu setiap laki-laki di antara kamu harus disunat;
\verse haruslah dikerat kulit khatanmu dan itulah akan menjadi tanda perjanjian antara Aku dan kamu.
\verse Anak yang berumur delapan hari haruslah disunat, yakni setiap laki-laki di antara kamu, turun-temurun: baik yang lahir di rumahmu, maupun yang dibeli dengan uang dari salah seorang asing, tetapi tidak termasuk keturunanmu.
\verse Orang yang lahir di rumahmu dan orang yang engkau beli dengan uang harus disunat; maka dalam dagingmulah perjanjian-Ku itu menjadi perjanjian yang kekal.
\verse Dan orang yang tidak disunat, yakni laki-laki yang tidak dikerat kulit khatannya, maka orang itu harus dilenyapkan dari antara orang-orang sebangsanya: ia telah mengingkari perjanjian-Ku."
\verse Selanjutnya Allah berfirman kepada Abraham: "Tentang isterimu Sarai, janganlah engkau menyebut dia lagi Sarai, tetapi Sara, itulah namanya.
\verse Aku akan memberkatinya, dan dari padanya juga Aku akan memberikan kepadamu seorang anak laki-laki, bahkan Aku akan memberkatinya, sehingga ia menjadi ibu bangsa-bangsa; raja-raja bangsa-bangsa akan lahir dari padanya."
\verse Lalu tertunduklah Abraham dan tertawa serta berkata dalam hatinya: "Mungkinkah bagi seorang yang berumur seratus tahun dilahirkan seorang anak dan mungkinkah Sara, yang telah berumur sembilan puluh tahun itu melahirkan seorang anak?"
\verse Dan Abraham berkata kepada Allah: "Ah, sekiranya Ismael diperkenankan hidup di hadapan-Mu!"
\verse Tetapi Allah berfirman: "Tidak, melainkan isterimu Saralah yang akan melahirkan anak laki-laki bagimu, dan engkau akan menamai dia Ishak, dan Aku akan mengadakan perjanjian-Ku dengan dia menjadi perjanjian yang kekal untuk keturunannya.
\verse Tentang Ismael, Aku telah mendengarkan permintaanmu; ia akan Kuberkati, Kubuat beranak cucu dan sangat banyak; ia akan memperanakkan dua belas raja, dan Aku akan membuatnya menjadi bangsa yang besar.
\verse Tetapi perjanjian-Ku akan Kuadakan dengan Ishak, yang akan dilahirkan Sara bagimu tahun yang akan datang pada waktu seperti ini juga."
\verse Setelah selesai berfirman kepada Abraham, naiklah Allah meninggalkan Abraham.
\verse Setelah itu Abraham memanggil Ismael, anaknya, dan semua orang yang lahir di rumahnya, juga semua orang yang dibelinya dengan uang, yakni setiap laki-laki dari isi rumahnya, lalu ia mengerat kulit khatan mereka pada hari itu juga, seperti yang telah difirmankan Allah kepadanya.
\verse Abraham berumur sembilan puluh sembilan tahun ketika dikerat kulit khatannya.
\verse Dan Ismael, anaknya, berumur tiga belas tahun ketika dikerat kulit khatannya.
\verse Pada hari itu juga Abraham dan Ismael, anaknya, disunat.
\verse Dan semua orang dari isi rumah Abraham, baik yang lahir di rumahnya, maupun yang dibeli dengan uang dari orang asing, disunat bersama-sama dengan dia.
\end{biblechapter}

\begin{biblechapter} % Kejadian 18
\verseWithHeading{Allah mengulangi menjanjikan seorang anak laki-laki kepada Abraham} Kemudian TUHAN menampakkan diri kepada Abraham dekat pohon tarbantin di Mamre, sedang ia duduk di pintu kemahnya waktu hari panas terik.
\verse Ketika ia mengangkat mukanya, ia melihat tiga orang berdiri di depannya. Sesudah dilihatnya mereka, ia berlari dari pintu kemahnya menyongsong mereka, lalu sujudlah ia sampai ke tanah,
\verse serta berkata: "Tuanku, jika aku telah mendapat kasih tuanku, janganlah kiranya lampaui hambamu ini.
\verse Biarlah diambil air sedikit, basuhlah kakimu dan duduklah beristirahat di bawah pohon ini;
\verse biarlah kuambil sepotong roti, supaya tuan-tuan segar kembali; kemudian bolehlah tuan-tuan meneruskan perjalanannya; sebab tuan-tuan telah datang ke tempat hambamu ini." Jawab mereka: "Perbuatlah seperti yang kaukatakan itu."
\verse Lalu Abraham segera pergi ke kemah mendapatkan Sara serta berkata: "Segeralah! Ambil tiga sukat tepung yang terbaik! Remaslah itu dan buatlah roti bundar!"
\verse Lalu berlarilah Abraham kepada lembu sapinya, ia mengambil seekor anak lembu yang empuk dan baik dagingnya dan memberikannya kepada seorang bujangnya, lalu orang ini segera mengolahnya.
\verse Kemudian diambilnya dadih dan susu serta anak lembu yang telah diolah itu, lalu dihidangkannya di depan orang-orang itu; dan ia berdiri di dekat mereka di bawah pohon itu, sedang mereka makan.
\verse Lalu kata mereka kepadanya: "Di manakah Sara, isterimu?" Jawabnya: "Di sana, di dalam kemah."
\verse Dan firman-Nya: "Sesungguhnya Aku akan kembali tahun depan mendapatkan engkau, pada waktu itulah Sara, isterimu, akan mempunyai seorang anak laki-laki." Dan Sara mendengarkan pada pintu kemah yang di belakang-Nya.
\verse Adapun Abraham dan Sara telah tua dan lanjut umurnya dan Sara telah mati haid.
\verse Jadi tertawalah Sara dalam hatinya, katanya: "Akan berahikah aku, setelah aku sudah layu, sedangkan tuanku sudah tua?"
\verse Lalu berfirmanlah TUHAN kepada Abraham: "Mengapakah Sara tertawa dan berkata: Sungguhkah aku akan melahirkan anak, sedangkan aku telah tua?
\verse Adakah sesuatu apa pun yang mustahil untuk TUHAN? Pada waktu yang telah ditetapkan itu, tahun depan, Aku akan kembali mendapatkan engkau, pada waktu itulah Sara mempunyai seorang anak laki-laki."
\verse Lalu Sara menyangkal, katanya: "Aku tidak tertawa," sebab ia takut; tetapi TUHAN berfirman: "Tidak, memang engkau tertawa!"
\verseWithSubheading{Doa syafaat Abraham untuk Sodom} Lalu berangkatlah orang-orang itu dari situ dan memandang ke arah Sodom; dan Abraham berjalan bersama-sama dengan mereka untuk mengantarkan mereka.
\verse Berpikirlah TUHAN: "Apakah Aku akan menyembunyikan kepada Abraham apa yang hendak Kulakukan ini?
\verse Bukankah sesungguhnya Abraham akan menjadi bangsa yang besar serta berkuasa, dan oleh dia segala bangsa di atas bumi akan mendapat berkat?
\verse Sebab Aku telah memilih dia, supaya diperintahkannya kepada anak-anaknya dan kepada keturunannya supaya tetap hidup menurut jalan yang ditunjukkan TUHAN, dengan melakukan kebenaran dan keadilan, dan supaya TUHAN memenuhi kepada Abraham apa yang dijanjikan-Nya kepadanya."
\verse Sesudah itu berfirmanlah TUHAN: "Sesungguhnya banyak keluh kesah orang tentang Sodom dan Gomora dan sesungguhnya sangat berat dosanya.
\verse Baiklah Aku turun untuk melihat, apakah benar-benar mereka telah berkelakuan seperti keluh kesah orang yang telah sampai kepada-Ku atau tidak; Aku hendak mengetahuinya."
\verse Lalu berpalinglah orang-orang itu dari situ dan berjalan ke Sodom, tetapi Abraham masih tetap berdiri di hadapan TUHAN.
\verse Abraham datang mendekat dan berkata: "Apakah Engkau akan melenyapkan orang benar bersama-sama dengan orang fasik?
\verse Bagaimana sekiranya ada lima puluh orang benar dalam kota itu? Apakah Engkau akan melenyapkan tempat itu dan tidakkah Engkau mengampuninya karena kelima puluh orang benar yang ada di dalamnya itu?
\verse Jauhlah kiranya dari pada-Mu untuk berbuat demikian, membunuh orang benar bersama-sama dengan orang fasik, sehingga orang benar itu seolah-olah sama dengan orang fasik! Jauhlah kiranya yang demikian dari pada-Mu! Masakan Hakim segenap bumi tidak menghukum dengan adil?"
\verse TUHAN berfirman: "Jika Kudapati lima puluh orang benar dalam kota Sodom, Aku akan mengampuni seluruh tempat itu karena mereka."
\verse Abraham menyahut: "Sesungguhnya aku telah memberanikan diri berkata kepada Tuhan, walaupun aku debu dan abu.
\verse Sekiranya kurang lima orang dari kelima puluh orang benar itu, apakah Engkau akan memusnahkan seluruh kota itu karena yang lima itu?" Firman-Nya: "Aku tidak memusnahkannya, jika Kudapati empat puluh lima di sana."
\verse Lagi Abraham melanjutkan perkataannya kepada-Nya: "Sekiranya empat puluh didapati di sana?" Firman-Nya: "Aku tidak akan berbuat demikian karena yang empat puluh itu."
\verse Katanya: "Janganlah kiranya Tuhan murka, kalau aku berkata sekali lagi. Sekiranya tiga puluh didapati di sana?" Firman-Nya: "Aku tidak akan berbuat demikian, jika Kudapati tiga puluh di sana."
\verse Katanya: "Sesungguhnya aku telah memberanikan diri berkata kepada Tuhan. Sekiranya dua puluh didapati di sana?" Firman-Nya: "Aku tidak akan memusnahkannya karena yang dua puluh itu."
\verse Katanya: "Janganlah kiranya Tuhan murka, kalau aku berkata lagi sekali ini saja. Sekiranya sepuluh didapati di sana?" Firman-Nya: "Aku tidak akan memusnahkannya karena yang sepuluh itu."
\verse Lalu pergilah TUHAN, setelah Ia selesai berfirman kepada Abraham; dan kembalilah Abraham ke tempat tinggalnya.
\end{biblechapter}

\begin{biblechapter} % Kejadian 19
\verseWithHeading{Sodom dan Gomora dimusnahkan Lot diselamatkan} Kedua malaikat itu tiba di Sodom pada waktu petang. Lot sedang duduk di pintu gerbang Sodom dan ketika melihat mereka, bangunlah ia menyongsong mereka, lalu sujud dengan mukanya sampai ke tanah,
\verse serta berkata: "Tuan-tuan, silakanlah singgah ke rumah hambamu ini, bermalamlah di sini dan basuhlah kakimu, maka besok pagi tuan-tuan boleh melanjutkan perjalanannya." Jawab mereka: "Tidak, kami akan bermalam di tanah lapang."
\verse Tetapi karena ia sangat mendesak mereka, singgahlah mereka dan masuk ke dalam rumahnya, kemudian ia menyediakan hidangan bagi mereka, ia membakar roti yang tidak beragi, lalu mereka makan.
\verse Tetapi sebelum mereka tidur, orang-orang lelaki dari kota Sodom itu, dari yang muda sampai yang tua, bahkan seluruh kota, tidak ada yang terkecuali, datang mengepung rumah itu.
\verse Mereka berseru kepada Lot: "Di manakah orang-orang yang datang kepadamu malam ini? Bawalah mereka keluar kepada kami, supaya kami pakai mereka."
\verse Lalu keluarlah Lot menemui mereka, ke depan pintu, tetapi pintu ditutupnya di belakangnya,
\verse dan ia berkata: "Saudara-saudaraku, janganlah kiranya berbuat jahat.
\verse Kamu tahu, aku mempunyai dua orang anak perempuan yang belum pernah dijamah laki-laki, baiklah mereka kubawa ke luar kepadamu; perbuatlah kepada mereka seperti yang kamu pandang baik; hanya jangan kamu apa-apakan orang-orang ini, sebab mereka memang datang untuk berlindung di dalam rumahku."
\verse Tetapi mereka berkata: "Enyahlah!" Lagi kata mereka: "Orang ini datang ke sini sebagai orang asing dan dia mau menjadi hakim atas kita! Sekarang kami akan menganiaya engkau lebih dari pada kedua orang itu!" Lalu mereka mendesak orang itu, yaitu Lot, dengan keras, dan mereka mendekat untuk mendobrak pintu.
\verse Tetapi kedua orang itu mengulurkan tangannya, menarik Lot masuk ke dalam rumah, lalu menutup pintu.
\verse Dan mereka membutakan mata orang-orang yang di depan pintu rumah itu, dari yang kecil sampai yang besar, sehingga percumalah orang-orang itu mencari-cari pintu.
\verse Lalu kedua orang itu berkata kepada Lot: "Siapakah kaummu yang ada di sini lagi? Menantu atau anakmu laki-laki, anakmu perempuan, atau siapa saja kaummu di kota ini, bawalah mereka keluar dari tempat ini,
\verse sebab kami akan memusnahkan tempat ini, karena banyak keluh kesah orang tentang kota ini di hadapan TUHAN; sebab itulah TUHAN mengutus kami untuk memusnahkannya."
\verse Keluarlah Lot, lalu berbicara dengan kedua bakal menantunya, yang akan kawin dengan kedua anaknya perempuan, katanya: "Bangunlah, keluarlah dari tempat ini, sebab TUHAN akan memusnahkan kota ini." Tetapi ia dipandang oleh kedua bakal menantunya itu sebagai orang yang berolok-olok saja.
\verse Ketika fajar telah menyingsing, kedua malaikat itu mendesak Lot, supaya bersegera, katanya: "Bangunlah, bawalah isterimu dan kedua anakmu yang ada di sini, supaya engkau jangan mati lenyap karena kedurjanaan kota ini."
\verse Ketika ia berlambat-lambat, maka tangannya, tangan isteri dan tangan kedua anaknya dipegang oleh kedua orang itu, sebab TUHAN hendak mengasihani dia; lalu kedua orang itu menuntunnya ke luar kota dan melepaskannya di sana.
\verse Sesudah kedua orang itu menuntun mereka sampai ke luar, berkatalah seorang: "Larilah, selamatkanlah nyawamu; janganlah menoleh ke belakang, dan janganlah berhenti di mana pun juga di Lembah Yordan, larilah ke pegunungan, supaya engkau jangan mati lenyap."
\verse Kata Lot kepada mereka: "Janganlah kiranya demikian, tuanku.
\verse Sungguhlah hambamu ini telah dikaruniai belas kasihan di hadapanmu, dan tuanku telah berbuat kemurahan besar kepadaku dengan memelihara hidupku, tetapi jika aku harus lari ke pegunungan, pastilah aku akan tersusul oleh bencana itu, sehingga matilah aku.
\verse Sungguhlah kota yang di sana itu cukup dekat kiranya untuk lari ke sana; kota itu kecil; izinkanlah kiranya aku lari ke sana. Bukankah kota itu kecil? Jika demikian, nyawaku akan terpelihara."
\verse Sahut malaikat itu kepadanya: "Baiklah, dalam hal ini pun permintaanmu akan kuterima dengan baik; yakni kota yang telah kau sebut itu tidak akan kutunggangbalikkan.
\verse Cepatlah, larilah ke sana, sebab aku tidak dapat berbuat apa-apa, sebelum engkau sampai ke sana." Itulah sebabnya nama kota itu disebut Zoar.
\verse Matahari telah terbit menyinari bumi, ketika Lot tiba di Zoar.
\verse Kemudian TUHAN menurunkan hujan belerang dan api atas Sodom dan Gomora, berasal dari TUHAN, dari langit;
\verse dan ditunggangbalikkan-Nyalah kota-kota itu dan Lembah Yordan dan semua penduduk kota-kota serta tumbuh-tumbuhan di tanah.
\verse Tetapi isteri Lot, yang berjalan mengikutnya, menoleh ke belakang, lalu menjadi tiang garam.
\verse Ketika Abraham pagi-pagi pergi ke tempat ia berdiri di hadapan TUHAN itu,
\verse dan memandang ke arah Sodom dan Gomora serta ke seluruh tanah Lembah Yordan, maka dilihatnyalah asap dari bumi membubung ke atas sebagai asap dari dapur peleburan.
\verse Demikianlah pada waktu Allah memusnahkan kota-kota di Lembah Yordan dan menunggangbalikkan kota-kota kediaman Lot, maka Allah ingat kepada Abraham, lalu dikeluarkan-Nyalah Lot dari tengah-tengah tempat yang ditunggangbalikkan itu.
\verseWithSubheading{Lot dan kedua anaknya perempuan} Pergilah Lot dari Zoar dan ia menetap bersama-sama dengan kedua anaknya perempuan di pegunungan, sebab ia tidak berani tinggal di Zoar, maka diamlah ia dalam suatu gua beserta kedua anaknya.
\verse Kata kakaknya kepada adiknya: "Ayah kita telah tua, dan tidak ada laki-laki di negeri ini yang dapat menghampiri kita, seperti kebiasaan seluruh bumi.
\verse Marilah kita beri ayah kita minum anggur, lalu kita tidur dengan dia, supaya kita menyambung keturunan dari ayah kita."
\verse Pada malam itu mereka memberi ayah mereka minum anggur, lalu masuklah yang lebih tua untuk tidur dengan ayahnya; dan ayahnya itu tidak mengetahui ketika anaknya itu tidur dan ketika ia bangun.
\verse Keesokan harinya berkatalah kakaknya kepada adiknya: "Tadi malam aku telah tidur dengan ayah; baiklah malam ini juga kita beri dia minum anggur; masuklah engkau untuk tidur dengan dia, supaya kita menyambung keturunan dari ayah kita."
\verse Demikianlah juga pada malam itu mereka memberi ayah mereka minum anggur, lalu bangunlah yang lebih muda untuk tidur dengan ayahnya; dan ayahnya itu tidak mengetahui ketika anaknya itu tidur dan ketika ia bangun.
\verse Lalu mengandunglah kedua anak Lot itu dari ayah mereka.
\verse Yang lebih tua melahirkan seorang anak laki-laki, dan menamainya Moab; dialah bapa orang Moab yang sekarang.
\verse Yang lebih muda pun melahirkan seorang anak laki-laki, dan menamainya Ben-Ami; dialah bapa bani Amon yang sekarang.
\end{biblechapter}

\begin{biblechapter} % Kejadian 20
\verseWithHeading{Abraham dan Abimelekh} Lalu Abraham berangkat dari situ ke Tanah Negeb dan ia menetap antara Kadesh dan Syur. Ia tinggal di Gerar sebagai orang asing.
\verse Oleh karena Abraham telah mengatakan tentang Sara, isterinya: "Dia saudaraku," maka Abimelekh, raja Gerar, menyuruh mengambil Sara.
\verse Tetapi pada waktu malam Allah datang kepada Abimelekh dalam suatu mimpi serta berfirman kepadanya: "Engkau harus mati oleh karena perempuan yang telah kauambil itu; sebab ia sudah bersuami."
\verse Adapun Abimelekh belum menghampiri Sara. Berkatalah ia: "Tuhan! Apakah Engkau membunuh bangsa yang tak bersalah?
\verse Bukankah orang itu sendiri mengatakan kepadaku: Dia saudaraku? Dan perempuan itu sendiri telah mengatakan: Ia saudaraku. Jadi hal ini kulakukan dengan hati yang tulus dan dengan tangan yang suci."
\verse Lalu berfirmanlah Allah kepadanya dalam mimpi: "Aku tahu juga, bahwa engkau telah melakukan hal itu dengan hati yang tulus, maka Aku pun telah mencegah engkau untuk berbuat dosa terhadap Aku; sebab itu Aku tidak membiarkan engkau menjamah dia.
\verse Jadi sekarang, kembalikanlah isteri orang itu, sebab dia seorang nabi; ia akan berdoa untuk engkau, maka engkau tetap hidup; tetapi jika engkau tidak mengembalikan dia, ketahuilah, engkau pasti mati, engkau dan semua orang yang bersama-sama dengan engkau."
\verse Keesokan harinya pagi-pagi Abimelekh memanggil semua hambanya dan memberitahukan seluruh peristiwa itu kepada mereka, lalu sangat takutlah orang-orang itu.
\verse Kemudian Abimelekh memanggil Abraham dan berkata kepadanya: "Perbuatan apakah yang kaulakukan ini terhadap kami, dan kesalahan apakah yang kulakukan terhadap engkau, sehingga engkau mendatangkan dosa besar atas diriku dan kerajaanku? Engkau telah berbuat hal-hal yang tidak patut kepadaku."
\verse Lagi kata Abimelekh kepada Abraham: "Apakah maksudmu, maka engkau melakukan hal ini?"
\verse Lalu Abraham berkata: "Aku berpikir: Takut akan Allah tidak ada di tempat ini; tentulah aku akan dibunuh karena isteriku.
\verse Lagipula ia benar-benar saudaraku, anak ayahku, hanya bukan anak ibuku, tetapi kemudian ia menjadi isteriku.
\verse Ketika Allah menyuruh aku mengembara keluar dari rumah ayahku, berkatalah aku kepada isteriku: Tunjukkanlah kasihmu kepadaku, yakni: katakanlah tentang aku di tiap-tiap tempat di mana kita tiba: Ia saudaraku."
\verse Kemudian Abimelekh mengambil kambing domba dan lembu sapi, hamba laki-laki dan perempuan, lalu memberikan semuanya itu kepada Abraham; Sara, isteri Abraham, juga dikembalikannya kepadanya.
\verse Dan Abimelekh berkata: "Negeriku ini terbuka untuk engkau; menetaplah, di mana engkau suka."
\verse Lalu katanya kepada Sara: "Telah kuberikan kepada saudaramu seribu syikal perak, itulah bukti kesucianmu bagi semua orang yang bersama-sama dengan engkau. Maka dalam segala hal engkau dibenarkan."
\verse Lalu Abraham berdoa kepada Allah, dan Allah menyembuhkan Abimelekh dan isterinya dan budak-budaknya perempuan, sehingga mereka melahirkan anak.
\verse Sebab tadinya TUHAN telah menutup kandungan setiap perempuan di istana Abimelekh karena Sara, isteri Abraham itu.
\end{biblechapter}

\begin{biblechapter} % Kejadian 21
\verseWithHeading{Ishak lahir} TUHAN memperhatikan Sara, seperti yang difirmankan-Nya, dan TUHAN melakukan kepada Sara seperti yang dijanjikan-Nya.
\verse Maka mengandunglah Sara, lalu ia melahirkan seorang anak laki-laki bagi Abraham dalam masa tuanya, pada waktu yang telah ditetapkan, sesuai dengan firman Allah kepadanya.
\verse Abraham menamai anaknya yang baru lahir itu Ishak, yang dilahirkan Sara baginya.
\verse Kemudian Abraham menyunat Ishak, anaknya itu, ketika berumur delapan hari, seperti yang diperintahkan Allah kepadanya.
\verse Adapun Abraham berumur seratus tahun, ketika Ishak, anaknya, lahir baginya.
\verse Berkatalah Sara: "Allah telah membuat aku tertawa; setiap orang yang mendengarnya akan tertawa karena aku."
\verse Lagi katanya: "Siapakah tadinya yang dapat mengatakan kepada Abraham: Sara menyusui anak? Namun aku telah melahirkan seorang anak laki-laki baginya pada masa tuanya."
\verseWithSubheading{Abraham mengusir Hagar dan Ismael} Bertambah besarlah anak itu dan ia disapih, lalu Abraham mengadakan perjamuan besar pada hari Ishak disapih itu.
\verse Pada waktu itu Sara melihat, bahwa anak yang dilahirkan Hagar, perempuan Mesir itu bagi Abraham, sedang main dengan Ishak, anaknya sendiri.
\verse Berkatalah Sara kepada Abraham: "Usirlah hamba perempuan itu beserta anaknya, sebab anak hamba ini tidak akan menjadi ahli waris bersama-sama dengan anakku Ishak."
\verse Hal ini sangat menyebalkan Abraham oleh karena anaknya itu.
\verse Tetapi Allah berfirman kepada Abraham: "Janganlah sebal hatimu karena hal anak dan budakmu itu; dalam segala yang dikatakan Sara kepadamu, haruslah engkau mendengarkannya, sebab yang akan disebut keturunanmu ialah yang berasal dari Ishak.
\verse Tetapi keturunan dari hambamu itu juga akan Kubuat menjadi suatu bangsa, karena ia pun anakmu."
\verse Keesokan harinya pagi-pagi Abraham mengambil roti serta sekirbat air dan memberikannya kepada Hagar. Ia meletakkan itu beserta anaknya di atas bahu Hagar, kemudian disuruhnyalah perempuan itu pergi. Maka pergilah Hagar dan mengembara di padang gurun Bersyeba.
\verse Ketika air yang dikirbat itu habis, dibuangnyalah anak itu ke bawah semak-semak,
\verse dan ia duduk agak jauh, kira-kira sepemanah jauhnya, sebab katanya: "Tidak tahan aku melihat anak itu mati." Sedang ia duduk di situ, menangislah ia dengan suara nyaring.
\verse Allah mendengar suara anak itu, lalu Malaikat Allah berseru dari langit kepada Hagar, kata-Nya kepadanya: "Apakah yang engkau susahkan, Hagar? Janganlah takut, sebab Allah telah mendengar suara anak itu dari tempat ia terbaring.
\verse Bangunlah, angkatlah anak itu, dan bimbinglah dia, sebab Aku akan membuat dia menjadi bangsa yang besar."
\verse Lalu Allah membuka mata Hagar, sehingga ia melihat sebuah sumur; ia pergi mengisi kirbatnya dengan air, kemudian diberinya anak itu minum.
\verse Allah menyertai anak itu, sehingga ia bertambah besar; ia menetap di padang gurun dan menjadi seorang pemanah.
\verse Maka tinggallah ia di padang gurun Paran, dan ibunya mengambil seorang isteri baginya dari tanah Mesir.
\verseWithSubheading{Perjanjian Abraham dengan Abimelekh} Pada waktu itu Abimelekh, beserta Pikhol, panglima tentaranya, berkata kepada Abraham: "Allah menyertai engkau dalam segala sesuatu yang engkau lakukan.
\verse Oleh sebab itu, bersumpahlah kepadaku di sini demi Allah, bahwa engkau tidak akan berlaku curang kepadaku, atau kepada anak-anakku, atau kepada cucu cicitku; sesuai dengan persahabatan yang kulakukan kepadamu, demikianlah harus engkau berlaku kepadaku dan kepada negeri yang kautinggali sebagai orang asing."
\verse Lalu kata Abraham: "Aku bersumpah!"
\verse Tetapi Abraham menyesali Abimelekh tentang sebuah sumur yang telah dirampas oleh hamba-hamba Abimelekh.
\verse Jawab Abimelekh: "Aku tidak tahu, siapa yang melakukan hal itu; lagi tidak kauberitahukan kepadaku, dan sampai hari ini belum pula kudengar."
\verse Lalu Abraham mengambil domba dan lembu dan memberikan semuanya itu kepada Abimelekh, kemudian kedua orang itu mengadakan perjanjian.
\verse Tetapi Abraham memisahkan tujuh anak domba betina dari domba-domba itu.
\verse Lalu kata Abimelekh kepada Abraham: "Untuk apakah ketujuh anak domba yang kaupisahkan ini?"
\verse Jawabnya: "Ketujuh anak domba ini harus kauterima dari tanganku untuk menjadi tanda bukti bagiku, bahwa akulah yang menggali sumur ini."
\verse Sebab itu orang menyebutkan tempat itu Bersyeba, karena kedua orang itu telah bersumpah di sana.
\verse Setelah mereka mengadakan perjanjian di Bersyeba, pulanglah Abimelekh beserta Pikhol, panglima tentaranya, ke negeri orang Filistin.
\verse Lalu Abraham menanam sebatang pohon tamariska di Bersyeba, dan memanggil di sana nama TUHAN, Allah yang kekal.
\verse Dan masih lama Abraham tinggal sebagai orang asing di negeri orang Filistin.
\end{biblechapter}

\begin{biblechapter} % Kejadian 22
\verseWithHeading{Kepercayaan Abraham diuji} Setelah semuanya itu Allah mencoba Abraham. Ia berfirman kepadanya: "Abraham," lalu sahutnya: "Ya, Tuhan."
\verse Firman-Nya: "Ambillah anakmu yang tunggal itu, yang engkau kasihi, yakni Ishak, pergilah ke tanah Moria dan persembahkanlah dia di sana sebagai korban bakaran pada salah satu gunung yang akan Kukatakan kepadamu."
\verse Keesokan harinya pagi-pagi bangunlah Abraham, ia memasang pelana keledainya dan memanggil dua orang bujangnya beserta Ishak, anaknya; ia membelah juga kayu untuk korban bakaran itu, lalu berangkatlah ia dan pergi ke tempat yang dikatakan Allah kepadanya.
\verse Ketika pada hari ketiga Abraham melayangkan pandangnya, kelihatanlah kepadanya tempat itu dari jauh.
\verse Kata Abraham kepada kedua bujangnya itu: "Tinggallah kamu di sini dengan keledai ini; aku beserta anak ini akan pergi ke sana; kami akan sembahyang, sesudah itu kami kembali kepadamu."
\verse Lalu Abraham mengambil kayu untuk korban bakaran itu dan memikulkannya ke atas bahu Ishak, anaknya, sedang di tangannya dibawanya api dan pisau. Demikianlah keduanya berjalan bersama-sama.
\verse Lalu berkatalah Ishak kepada Abraham, ayahnya: "Bapa." Sahut Abraham: "Ya, anakku." Bertanyalah ia: "Di sini sudah ada api dan kayu, tetapi di manakah anak domba untuk korban bakaran itu?"
\verse Sahut Abraham: "Allah yang akan menyediakan anak domba untuk korban bakaran bagi-Nya, anakku." Demikianlah keduanya berjalan bersama-sama.
\verse Sampailah mereka ke tempat yang dikatakan Allah kepadanya. Lalu Abraham mendirikan mezbah di situ, disusunnyalah kayu, diikatnya Ishak, anaknya itu, dan diletakkannya di mezbah itu, di atas kayu api.
\verse Sesudah itu Abraham mengulurkan tangannya, lalu mengambil pisau untuk menyembelih anaknya.
\verse Tetapi berserulah Malaikat TUHAN dari langit kepadanya: "Abraham, Abraham." Sahutnya: "Ya, Tuhan."
\verse Lalu Ia berfirman: "Jangan bunuh anak itu dan jangan kauapa-apakan dia, sebab telah Kuketahui sekarang, bahwa engkau takut akan Allah, dan engkau tidak segan-segan untuk menyerahkan anakmu yang tunggal kepada-Ku."
\verse Lalu Abraham menoleh dan melihat seekor domba jantan di belakangnya, yang tanduknya tersangkut dalam belukar. Abraham mengambil domba itu, lalu mengorbankannya sebagai korban bakaran pengganti anaknya.
\verse Dan Abraham menamai tempat itu: "TUHAN menyediakan"; sebab itu sampai sekarang dikatakan orang: "Di atas gunung TUHAN, akan disediakan."
\verse Untuk kedua kalinya berserulah Malaikat TUHAN dari langit kepada Abraham,
\verse kata-Nya: "Aku bersumpah demi diri-Ku sendiri -- demikianlah firman TUHAN --: Karena engkau telah berbuat demikian, dan engkau tidak segan-segan untuk menyerahkan anakmu yang tunggal kepada-Ku,
\verse maka Aku akan memberkati engkau berlimpah-limpah dan membuat keturunanmu sangat banyak seperti bintang di langit dan seperti pasir di tepi laut, dan keturunanmu itu akan menduduki kota-kota musuhnya.
\verse Oleh keturunanmulah semua bangsa di bumi akan mendapat berkat, karena engkau mendengarkan firman-Ku."
\verse Kemudian kembalilah Abraham kepada kedua bujangnya, dan mereka bersama-sama berangkat ke Bersyeba; dan Abraham tinggal di Bersyeba.
\verseWithSubheading{Keturunan Nahor} Sesudah itu Abraham mendapat kabar: "Juga Milka telah melahirkan anak-anak lelaki bagi Nahor, saudaramu:
\verse Us, anak sulung, dan Bus, adiknya, dan Kemuel, ayah Aram,
\verse juga Kesed, Hazo, Pildash, Yidlaf dan Betuel."
\verse Dan Betuel memperanakkan Ribka. Kedelapan orang inilah dilahirkan Milka bagi Nahor, saudara Abraham itu.
\verse Dan gundik Nahor, yang namanya Reuma, melahirkan anak juga, yakni Tebah, Gaham, Tahash dan Maakha.
\end{biblechapter}

\begin{biblechapter} % Kejadian 23
\verseWithHeading{Sara mati dan dikuburkan} Sara hidup seratus dua puluh tujuh tahun lamanya; itulah umur Sara.
\verse Kemudian matilah Sara di Kiryat-Arba, yaitu Hebron, di tanah Kanaan, lalu Abraham datang meratapi dan menangisinya.
\verse Sesudah itu Abraham bangkit dan meninggalkan isterinya yang mati itu, lalu berkata kepada bani Het:
\verse "Aku ini orang asing dan pendatang di antara kamu; berikanlah kiranya kuburan milik kepadaku di tanah kamu ini, supaya kiranya aku dapat mengantarkan dan menguburkan isteriku yang mati itu."
\verse Bani Het menjawab Abraham:
\verse "Dengarlah kepada kami, tuanku. Tuanku ini seorang raja agung di tengah-tengah kami; jadi kuburkanlah isterimu yang mati itu dalam kuburan kami yang terpilih, tidak akan ada seorang pun dari kami yang menolak menyediakan kuburannya bagimu untuk menguburkan isterimu yang mati itu."
\verse Kemudian bangunlah Abraham lalu sujud kepada bani Het, penduduk negeri itu,
\verse serta berkata kepada mereka: "Jika kamu setuju, bahwa aku mengantarkan dan menguburkan isteriku yang mati itu, maka dengarkanlah aku dan tolonglah mintakan dengan sangat kepada Efron bin Zohar,
\verse supaya ia memberikan kepadaku gua Makhpela miliknya itu, yang terletak di ujung ladangnya; baiklah itu diberikannya kepadaku dengan harga penuh untuk menjadi kuburan milikku di tengah-tengah kamu."
\verse Pada waktu itu Efron hadir di tengah-tengah bani Het. Maka jawab Efron, orang Het itu, kepada Abraham dengan didengar oleh bani Het, oleh semua orang yang datang di pintu gerbang kota:
\verse "Tidak, tuanku, dengarkanlah aku; ladang itu kuberikan kepadamu dan gua yang di sana pun kuberikan kepadamu; di depan mata orang-orang sebangsaku kuberikan itu kepadamu; kuburkanlah isterimu yang mati itu."
\verse Lalu sujudlah Abraham di depan penduduk negeri itu
\verse serta berkata kepada Efron dengan didengar oleh mereka: "Sesungguhnya, jika engkau suka, dengarkanlah aku: aku membayar harga ladang itu; terimalah itu dari padaku, supaya aku dapat menguburkan isteriku yang mati itu di sana."
\verse Jawab Efron kepada Abraham:
\verse "Tuanku, dengarkanlah aku: sebidang tanah dengan harga empat ratus syikal perak, apa artinya itu bagi kita? Kuburkan sajalah isterimu yang mati itu."
\verse Lalu Abraham menerima usul Efron, maka ditimbangnyalah perak untuk Efron, sebanyak yang dimintanya dengan didengar oleh bani Het itu, empat ratus syikal perak, seperti yang berlaku di antara para saudagar.
\verse Demikianlah ladang Efron, yang letaknya di Makhpela di sebelah timur Mamre, ladang dan gua yang di sana, serta segala pohon di ladang itu, bahkan di seluruh tanah itu sampai ke tepi-tepinya,
\verse diserahkan kepada Abraham menjadi tanah belian, di depan mata bani Het itu, di depan semua orang yang datang di pintu gerbang kota.
\verse Sesudah itu Abraham menguburkan Sara, isterinya, di dalam gua ladang Makhpela itu, di sebelah timur Mamre, yaitu Hebron di tanah Kanaan.
\verse Demikianlah dari pihak bani Het ladang dengan gua yang ada di sana diserahkan kepada Abraham menjadi kuburan miliknya.
\end{biblechapter}

\begin{biblechapter} % Kejadian 24
\verseWithHeading{Ribka dipinang bagi Ishak} Adapun Abraham telah tua dan lanjut umurnya, serta diberkati TUHAN dalam segala hal.
\verse Berkatalah Abraham kepada hambanya yang paling tua dalam rumahnya, yang menjadi kuasa atas segala kepunyaannya, katanya “Baiklah letakkan tanganmu di bawah pangkal pahaku,
\verse supaya aku mengambil sumpahmu demi TUHAN, Allah yang empunya langit dan yang empunya bumi, bahwa engkau tidak akan mengambil untuk anakku seorang isteri dari antara perempuan Kanaan yang di antaranya aku diam.
\verse Tetapi engkau harus pergi ke negeriku dan kepada sanak saudaraku untuk mengambil seorang isteri bagi Ishak, anakku. “
\verse Lalu berkatalah hambanya itu kepadanya “Mungkin perempuan itu tidak suka mengikuti aku ke negeri ini; haruskah aku membawa anakmu itu kembali ke negeri dari mana tuanku keluar? “
\verse Tetapi Abraham berkata kepadanya “Awas, jangan kaubawa anakku itu kembali ke sana.
\verse TUHAN, Allah yang empunya langit, yang telah memanggil aku dari rumah ayahku serta dari negeri sanak saudaraku, dan yang telah berfirman kepadaku, serta yang bersumpah kepadaku, demikian kepada keturunanmulah akan Kuberikan negeri ini -- Dialah juga akan mengutus malaikat-Nya berjalan di depanmu, sehingga engkau dapat mengambil seorang isteri dari sana untuk anakku.
\verse Tetapi jika perempuan itu tidak mau mengikuti engkau, maka lepaslah engkau dari sumpahmu kepadaku ini; hanya saja, janganlah anakku itu kaubawa kembali ke sana. “
\verse Lalu hamba itu meletakkan tangannya di bawah pangkal paha Abraham, tuannya, dan bersumpah kepadanya tentang hal itu.
\verse Kemudian hamba itu mengambil sepuluh ekor dari unta tuannya dan pergi dengan membawa berbagai-bagai barang berharga kepunyaan tuannya; demikianlah ia berangkat menuju Aram-Mesopotamia ke kota Nahor.
\verse Di sana disuruhnyalah unta itu berhenti di luar kota dekat suatu sumur, pada waktu petang hari, waktu perempuan-perempuan keluar untuk menimba air.
\verse Lalu berkatalah ia “TUHAN, Allah tuanku Abraham, buatlah kiranya tercapai tujuanku pada hari ini, tunjukkanlah kasih setia-Mu kepada tuanku Abraham.
\verse Di sini aku berdiri di dekat mata air, dan anak-anak perempuan penduduk kota ini datang keluar untuk menimba air.
\verse Kiranya terjadilah begini anak gadis, kepada siapa aku berkata Tolong miringkan buyungmu itu, supaya aku minum, dan yang menjawab Minumlah, dan unta-untamu juga akan kuberi minum -- dialah kiranya yang Kau tentukan bagi hamba-Mu, Ishak; maka dengan begitu akan kuketahui, bahwa Engkau telah menunjukkan kasih setia-Mu kepada tuanku itu. “
\verse Sebelum ia selesai berkata, maka datanglah Ribka, yang lahir bagi Betuel, anak laki-laki Milka, isteri Nahor, saudara Abraham; buyungnya dibawanya di atas bahunya.
\verse Anak gadis itu sangat cantik parasnya, seorang perawan, belum pernah bersetubuh dengan laki-laki; ia turun ke mata air itu dan mengisi buyungnya, lalu kembali naik.
\verse Kemudian berlarilah hamba itu mendapatkannya serta berkata “Tolong beri aku minum air sedikit dari buyungmu itu. “
\verse Jawabnya “Minumlah, tuan, “ maka segeralah diturunkannya buyungnya itu ke tangannya, serta diberinya dia minum.
\verse Setelah ia selesai memberi hamba itu minum, berkatalah ia “Baiklah untuk unta-untamu juga kutimba air, sampai semuanya puas minum. “
\verse Kemudian segeralah dituangnya air yang di buyungnya itu ke dalam palungan, lalu berlarilah ia sekali lagi ke sumur untuk menimba air dan ditimbanyalah untuk semua unta orang itu.
\verse Dan orang itu mengamat-amatinya dengan berdiam diri untuk mengetahui apakah TUHAN membuat perjalanannya berhasil atau tidak.
\verse Setelah unta-unta itu puas minum, maka orang itu mengambil anting-anting emas yang setengah syikal beratnya, dan sepasang gelang tangan yang sepuluh syikal emas beratnya,
\verse serta berkata “Anak siapakah engkau? Baiklah katakan kepadaku! Adakah di rumah ayahmu tempat bermalam bagi kami? “
\verse Lalu jawabnya kepadanya “Ayahku Betuel, anak Milka, yang melahirkannya bagi Nahor. “
\verse Lagi kata gadis itu “Baik jerami, baik makanan unta banyak pada kami, tempat bermalam pun ada. “
\verse Lalu berlututlah orang itu dan sujud menyembah TUHAN,
\verse serta berkata “Terpujilah TUHAN, Allah tuanku Abraham, yang tidak menarik kembali kasih-Nya dan setia-Nya dari tuanku itu; dan TUHAN telah menuntun aku di jalan ke rumah saudara-saudara tuanku ini! “
\verse Berlarilah gadis itu pergi menceritakan kejadian itu ke rumah ibunya.
\verse Ribka mempunyai saudara laki-laki, namanya Laban. Laban berlari ke luar mendapatkan orang itu, ke mata air tadi,
\verse sesudah dilihatnya anting-anting itu dan gelang pada tangan saudaranya, dan sesudah didengarnya perkataan Ribka, saudaranya, yang bunyinya “Begitulah dikatakan orang itu kepadaku. “ Ia mendapatkan orang itu, yang masih berdiri di samping unta-untanya di dekat mata air itu,
\verse dan berkata “Marilah engkau yang diberkati TUHAN, mengapa engkau berdiri di luar, padahal telah kusediakan rumah bagimu, dan juga tempat untuk unta-untamu. “
\verse Masuklah orang itu ke dalam rumah. Ditanggalkanlah pelana unta-unta, diberikan jerami dan makanan kepada unta-unta itu, lalu dibawa air pembasuh kaki untuk orang itu dan orang-orang yang bersama-sama dengan dia.
\verse Tetapi ketika dihidangkan makanan di depannya, berkatalah orang itu “Aku tidak akan makan sebelum kusampaikan pesan yang kubawa ini. “ Jawab Laban “Silakan! “
\verse Lalu berkatalah ia “Aku ini hamba Abraham.
\verse TUHAN sangat memberkati tuanku itu, sehingga ia telah menjadi kaya; TUHAN telah memberikan kepadanya kambing domba dan lembu sapi, emas dan perak, budak laki-laki dan perempuan, unta dan keledai.
\verse Dan Sara, isteri tuanku itu, sesudah tua, telah melahirkan anak laki-laki bagi tuanku itu; kepada anaknya itu telah diberikan tuanku segala harta miliknya.
\verse Tuanku itu telah mengambil sumpahku Engkau tidak akan mengambil untuk anakku seorang isteri dari antara perempuan Kanaan, yang negerinya kudiami ini,
\verse tetapi engkau harus pergi ke rumah ayahku dan kepada kaumku untuk mengambil seorang isteri bagi anakku.
\verse Jawabku kepada tuanku itu Mungkin perempuan itu tidak mau mengikut aku.
\verse Tetapi katanya kepadaku TUHAN, yang di hadapan-Nya aku hidup, akan mengutus malaikat-Nya menyertai engkau, dan akan membuat perjalananmu berhasil, sehingga engkau akan mengambil bagi anakku seorang isteri dari kaumku dan dari rumah ayahku.
\verse Barulah engkau lepas dari sumpahmu kepadaku, jika engkau sampai kepada kaumku dan mereka tidak memberikan perempuan itu kepadamu; hanya dalam hal itulah engkau lepas dari sumpahmu kepadaku.
\verse Dan hari ini aku sampai ke mata air tadi, lalu kataku TUHAN, Allah tuanku Abraham, sudilah kiranya Engkau membuat berhasil perjalanan yang kutempuh ini.
\verse Di sini aku berdiri di dekat mata air ini; kiranya terjadi begini Apabila seorang gadis datang ke luar untuk menimba air dan aku berkata kepadanya Tolong berikan aku minum air sedikit dari buyungmu itu,
\verse dan ia menjawab Minumlah, dan untuk unta-untamu juga akan kutimba air, -- dialah kiranya isteri, yang telah TUHAN tentukan bagi anak tuanku itu.
\verse Belum lagi aku habis berkata dalam hatiku, Ribka telah datang membawa buyung di atas bahunya, dan turun ke mata air itu, lalu menimba air. Kataku kepadanya Tolong berikan aku minum.
\verse Segeralah ia menurunkan buyung itu dari atas bahunya serta berkata Minumlah, dan unta-untamu juga akan kuberi minum. Lalu aku minum, dan unta-unta itu juga diberinya minum.
\verse Sesudah itu aku bertanya kepadanya Anak siapakah engkau? Jawabnya Ayahku Betuel anak Nahor yang dilahirkan Milka. Lalu aku mengenakan anting-anting pada hidungnya dan gelang pada tangannya.
\verse Kemudian berlututlah aku dan sujud menyembah TUHAN, serta memuji TUHAN, Allah tuanku Abraham, yang telah menuntun aku di jalan yang benar untuk mengambil anak perempuan saudara tuanku ini bagi anaknya.
\verse Jadi sekarang, apabila kamu mau menunjukkan kasih dan setia kepada tuanku itu, beritahukanlah kepadaku; dan jika tidak, beritahukanlah juga kepadaku, supaya aku tahu entah berpaling ke kanan atau ke kiri. “
\verse Lalu Laban dan Betuel menjawab “Semuanya ini datangnya dari TUHAN; kami tidak dapat mengatakan kepadamu baiknya atau buruknya.
\verse Lihat, Ribka ada di depanmu, bawalah dia dan pergilah, supaya ia menjadi isteri anak tuanmu, seperti yang difirmankan TUHAN. “
\verse Ketika hamba Abraham itu mendengar perkataan mereka, sujudlah ia sampai ke tanah menyembah TUHAN.
\verse Kemudian hamba itu mengeluarkan perhiasan emas dan perak serta pakaian kebesaran, dan memberikan semua itu kepada Ribka; juga kepada saudaranya dan kepada ibunya diberikannya pemberian yang indah-indah.
\verse Sesudah itu makan dan minumlah mereka, ia dan orang-orang yang bersama-sama dengan dia, dan mereka bermalam di situ. Paginya sesudah mereka bangun, berkatalah hamba itu “Lepaslah aku pulang kepada tuanku. “
\verse Tetapi saudara Ribka berkata, serta ibunya juga “Biarkanlah anak gadis itu tinggal pada kami barang sepuluh hari lagi, kemudian bolehlah engkau pergi. “
\verse Tetapi jawabnya kepada mereka “Janganlah tahan aku, sedang TUHAN telah membuat perjalananku berhasil; lepaslah aku, supaya aku pulang kepada tuanku. “
\verse Kata mereka “Baiklah kita panggil anak gadis itu dan menanyakan kepadanya sendiri. “
\verse Lalu mereka memanggil Ribka dan berkata kepadanya “Maukah engkau pergi beserta orang ini? “ Jawabnya “Mau. “
\verse Maka Ribka, saudara mereka itu, dan inang pengasuhnya beserta hamba Abraham dan orang-orangnya dibiarkan mereka pergi.
\verse Dan mereka memberkati Ribka, kata mereka kepadanya “Saudara kami, moga-moga engkau menjadi beribu-ribu laksa, dan moga-moga keturunanmu menduduki kota-kota musuhnya. “
\verse Lalu berkemaslah Ribka beserta hamba-hambanya perempuan, dan mereka naik unta mengikuti orang itu. Demikianlah hamba itu membawa Ribka lalu berjalan pulang.
\verse Adapun Ishak telah datang dari arah sumur Lahai-Roi; ia tinggal di Tanah Negeb.
\verse Menjelang senja Ishak sedang keluar untuk berjalan-jalan di padang. Ia melayangkan pandangnya, maka dilihatnyalah ada unta-unta datang.
\verse Ribka juga melayangkan pandangnya dan ketika dilihatnya Ishak, turunlah ia dari untanya.
\verse Katanya kepada hamba itu “Siapakah laki-laki itu yang berjalan di padang ke arah kita? “ Jawab hamba itu “Dialah tuanku itu. “ Lalu Ribka mengambil telekungnya dan bertelekunglah ia.
\verse Kemudian hamba itu menceritakan kepada Ishak segala yang dilakukannya.
\verse Lalu Ishak membawa Ribka ke dalam kemah Sara, ibunya, dan mengambil dia menjadi isterinya. Ishak mencintainya dan demikian ia dihiburkan setelah ibunya meninggal.
\end{biblechapter}

\begin{biblechapter} % Kejadian 25
\verseWithHeading{Keturunan Abraham dari Ketura} Abraham mengambil pula seorang isteri, namanya Ketura.
\verse Perempuan itu melahirkan baginya Zimran, Yoksan, Medan, Midian, Isybak dan Suah.
\verse Yoksan memperanakkan Syeba dan Dedan. Keturunan Dedan ialah orang Asyur, orang Letush dan orang Leum.
\verse Anak-anak Midian ialah Efa, Efer, Henokh, Abida dan Eldaa. Itulah semuanya keturunan Ketura.
\verse Abraham memberikan segala harta miliknya kepada Ishak,
\verse tetapi kepada anak-anaknya yang diperolehnya dari gundik-gundiknya ia memberikan pemberian; kemudian ia menyuruh mereka -- masih pada waktu ia hidup -- meninggalkan Ishak, anaknya, dan pergi ke sebelah timur, ke Tanah Timur.
\verseWithSubheading{Abraham meninggal dan dikuburkan} Abraham mencapai umur seratus tujuh puluh lima tahun,
\verse lalu ia meninggal. Ia mati pada waktu telah putih rambutnya, tua dan suntuk umur, maka ia dikumpulkan kepada kaum leluhurnya.
\verse Dan anak-anaknya, Ishak dan Ismael, menguburkan dia dalam gua Makhpela, di padang Efron bin Zohar, orang Het itu, padang yang letaknya di sebelah timur Mamre,
\verse yang telah dibeli Abraham dari bani Het; di sanalah terkubur Abraham dan Sara isterinya.
\verse Setelah Abraham mati, Allah memberkati Ishak, anaknya itu; dan Ishak diam dekat sumur Lahai-Roi.
\verseWithSubheading{Keturunan Ismael} Inilah keturunan Ismael, anak Abraham, yang telah dilahirkan baginya oleh Hagar, perempuan Mesir, hamba Sara itu.
\verse Inilah nama anak-anak Ismael, disebutkan menurut urutan lahirnya: Nebayot, anak sulung Ismael, selanjutnya Kedar, Adbeel, Mibsam,
\verse Misyma, Duma, Masa,
\verse Hadad, Tema, Yetur, Nafish dan Kedma.
\verse Itulah anak-anak Ismael, dan itulah nama-nama mereka, menurut kampung mereka dan menurut perkemahan mereka, dua belas orang raja, masing-masing dengan sukunya.
\verse Umur Ismael ialah seratus tiga puluh tujuh tahun. Sesudah itu ia meninggal. Ia mati dan dikumpulkan kepada kaum leluhurnya.
\verse Mereka itu mendiami daerah dari Hawila sampai Syur, yang letaknya di sebelah timur Mesir ke arah Asyur. Mereka menetap berhadapan dengan semua saudara mereka.
\verseWithSubheading{Esau dan Yakub} Inilah riwayat keturunan Ishak, anak Abraham. Abraham memperanakkan Ishak.
\verse Dan Ishak berumur empat puluh tahun, ketika Ribka, anak Betuel, orang Aram dari Padan-Aram, saudara perempuan Laban orang Aram itu, diambilnya menjadi isterinya.
\verse Berdoalah Ishak kepada TUHAN untuk isterinya, sebab isterinya itu mandul; TUHAN mengabulkan doanya, sehingga Ribka, isterinya itu, mengandung.
\verse Tetapi anak-anaknya bertolak-tolakan di dalam rahimnya dan ia berkata: "Jika demikian halnya, mengapa aku hidup?" Dan ia pergi meminta petunjuk kepada TUHAN.
\verse Firman TUHAN kepadanya: "Dua bangsa ada dalam kandunganmu, dan dua suku bangsa akan berpencar dari dalam rahimmu; suku bangsa yang satu akan lebih kuat dari yang lain, dan anak yang tua akan menjadi hamba kepada anak yang muda."
\verse Setelah genap harinya untuk bersalin, memang anak kembar yang di dalam kandungannya.
\verse Keluarlah yang pertama, warnanya merah, seluruh tubuhnya seperti jubah berbulu; sebab itu ia dinamai Esau.
\verse Sesudah itu keluarlah adiknya; tangannya memegang tumit Esau, sebab itu ia dinamai Yakub. Ishak berumur enam puluh tahun pada waktu mereka lahir.
\verse Lalu bertambah besarlah kedua anak itu: Esau menjadi seorang yang pandai berburu, seorang yang suka tinggal di padang, tetapi Yakub adalah seorang yang tenang, yang suka tinggal di kemah.
\verse Ishak sayang kepada Esau, sebab ia suka makan daging buruan, tetapi Ribka kasih kepada Yakub.
\verse Pada suatu kali Yakub sedang memasak sesuatu, lalu datanglah Esau dengan lelah dari padang.
\verse Kata Esau kepada Yakub: "Berikanlah kiranya aku menghirup sedikit dari yang merah-merah itu, karena aku lelah." Itulah sebabnya namanya disebutkan Edom.
\verse Tetapi kata Yakub: "Juallah dahulu kepadaku hak kesulunganmu."
\verse Sahut Esau: "Sebentar lagi aku akan mati; apakah gunanya bagiku hak kesulungan itu?"
\verse Kata Yakub: "Bersumpahlah dahulu kepadaku." Maka bersumpahlah ia kepada Yakub dan dijualnyalah hak kesulungannya kepadanya.
\verse Lalu Yakub memberikan roti dan masakan kacang merah itu kepada Esau; ia makan dan minum, lalu berdiri dan pergi. Demikianlah Esau memandang ringan hak kesulungan itu.
\end{biblechapter}

\begin{biblechapter} % Kejadian 26
\verseWithHeading{Ishak di negeri orang Filistin} Maka timbullah kelaparan di negeri itu. -- Ini bukan kelaparan yang pertama, yang telah terjadi dalam zaman Abraham. Sebab itu Ishak pergi ke Gerar, kepada Abimelekh, raja orang Filistin.
\verse Lalu TUHAN menampakkan diri kepadanya serta berfirman “Janganlah pergi ke Mesir, diamlah di negeri yang akan Kukatakan kepadamu.
\verse Tinggallah di negeri ini sebagai orang asing, maka Aku akan menyertai engkau dan memberkati engkau, sebab kepadamulah dan kepada keturunanmu akan Kuberikan seluruh negeri ini, dan Aku akan menepati sumpah yang telah Kuikrarkan kepada Abraham, ayahmu.
\verse Aku akan membuat banyak keturunanmu seperti bintang di langit; Aku akan memberikan kepada keturunanmu seluruh negeri ini, dan oleh keturunanmu semua bangsa di bumi akan mendapat berkat,
\verse karena Abraham telah mendengarkan firman-Ku dan memelihara kewajibannya kepada-Ku, yaitu segala perintah, ketetapan dan hukum-Ku. “
\verse Jadi tinggallah Ishak di Gerar.
\verse Ketika orang-orang di tempat itu bertanya tentang isterinya, berkatalah ia “Dia saudaraku, “ sebab ia takut mengatakan “Ia isteriku, “ karena pikirnya “Jangan-jangan aku dibunuh oleh penduduk tempat ini karena Ribka, sebab elok parasnya. “
\verse Setelah beberapa lama ia ada di sana, pada suatu kali menjenguklah Abimelekh, raja orang Filistin itu dari jendela, maka dilihatnya Ishak sedang bercumbu-cumbuan dengan Ribka, isterinya.
\verse Lalu Abimelekh memanggil Ishak dan berkata “Sesungguhnya dia isterimu, masakan engkau berkata Dia saudaraku? “ Jawab Ishak kepadanya “Karena pikirku Jangan-jangan aku mati karena dia. “
\verse Tetapi Abimelekh berkata “Apakah juga yang telah kauperbuat ini terhadap kami? Mudah sekali terjadi, salah seorang dari bangsa ini tidur dengan isterimu, sehingga dengan demikian engkau mendatangkan kesalahan atas kami. “
\verse Lalu Abimelekh memberi perintah kepada seluruh bangsa itu “Siapa yang mengganggu orang ini atau isterinya, pastilah ia akan dihukum mati. “
\verse Maka menaburlah Ishak di tanah itu dan dalam tahun itu juga ia mendapat hasil seratus kali lipat; sebab ia diberkati TUHAN.
\verse Dan orang itu menjadi kaya, bahkan kian lama kian kaya, sehingga ia menjadi sangat kaya.
\verse Ia mempunyai kumpulan kambing domba dan lembu sapi serta banyak anak buah, sehingga orang Filistin itu cemburu kepadanya.
\verse Segala sumur, yang digali dalam zaman Abraham, ayahnya, oleh hamba-hamba ayahnya itu, telah ditutup oleh orang Filistin dan ditimbun dengan tanah.
\verse Lalu kata Abimelekh kepada Ishak “Pergilah dari tengah-tengah kami sebab engkau telah menjadi jauh lebih berkuasa dari pada kami. “
\verse Jadi pergilah Ishak dari situ dan berkemahlah ia di lembah Gerar, dan ia menetap di situ.
\verse Kemudian Ishak menggali kembali sumur-sumur yang digali dalam zaman Abraham, ayahnya, dan yang telah ditutup oleh orang Filistin sesudah Abraham mati; disebutkannyalah nama sumur-sumur itu menurut nama-nama yang telah diberikan oleh ayahnya.
\verse Ketika hamba-hamba Ishak menggali di lembah itu, mereka mendapati di situ mata air yang berbual-bual airnya.
\verse Lalu bertengkarlah para gembala Gerar dengan para gembala Ishak. Kata mereka “Air ini kepunyaan kami. “ Dan Ishak menamai sumur itu Esek, karena mereka bertengkar dengan dia di sana.
\verse Kemudian mereka menggali sumur lain, dan mereka bertengkar juga tentang itu. Maka Ishak menamai sumur itu Sitna.
\verse Ia pindah dari situ dan menggali sumur yang lain lagi, tetapi tentang sumur ini mereka tidak bertengkar. Sumur ini dinamainya Rehobot, dan ia berkata “Sekarang TUHAN telah memberikan kelonggaran kepada kita, sehingga kita dapat beranak cucu di negeri ini. “
\verse Dari situ ia pergi ke Bersyeba.
\verse Lalu pada malam itu TUHAN menampakkan diri kepadanya serta berfirman “Akulah Allah ayahmu Abraham; janganlah takut, sebab Aku menyertai engkau; Aku akan memberkati engkau dan membuat banyak keturunanmu karena Abraham, hamba-Ku itu. “
\verse Sesudah itu Ishak mendirikan mezbah di situ dan memanggil nama TUHAN. Ia memasang kemahnya di situ, lalu hamba-hambanya menggali sumur di situ.
\verse Datanglah Abimelekh dari Gerar mendapatkannya, bersama-sama dengan Ahuzat, sahabatnya, dan Pikhol, kepala pasukannya.
\verse Tetapi kata Ishak kepada mereka “Mengapa kamu datang mendapatkan aku? Bukankah kamu benci kepadaku, dan telah menyuruh aku keluar dari tanahmu? “
\verse Jawab mereka “Kami telah melihat sendiri, bahwa TUHAN menyertai engkau; sebab itu kami berkata baiklah kita mengadakan sumpah setia, antara kami dan engkau; dan baiklah kami mengikat perjanjian dengan engkau,
\verse bahwa engkau tidak akan berbuat jahat kepada kami, seperti kami tidak mengganggu engkau, dan seperti kami semata-mata berbuat baik kepadamu dan membiarkan engkau pergi dengan damai; bukankah engkau sekarang yang diberkati TUHAN. “
\verse Kemudian Ishak mengadakan perjamuan bagi mereka, lalu mereka makan dan minum.
\verse Keesokan harinya pagi-pagi bersumpah-sumpahanlah mereka. Kemudian Ishak melepas mereka, dan mereka meninggalkan dia dengan damai.
\verse Pada hari itu datanglah hamba-hamba Ishak memberitahukan kepadanya tentang sumur yang telah digali mereka, serta berkata kepadanya “Kami telah mendapat air. “
\verse Lalu dinamainyalah sumur itu Syeba. Sebab itu nama kota itu adalah Bersyeba, sampai sekarang.
\verse Ketika Esau telah berumur empat puluh tahun, ia mengambil Yudit, anak Beeri orang Het, dan Basmat, anak Elon orang Het, menjadi isterinya.
\verse Kedua perempuan itu menimbulkan kepedihan hati bagi Ishak dan bagi Ribka.
\end{biblechapter}

\begin{biblechapter} % Kejadian 27
\verseWithHeading{Yakub diberkati Ishak sebagai anak sulung} Ketika Ishak sudah tua, dan matanya telah kabur, sehingga ia tidak dapat melihat lagi, dipanggilnyalah Esau, anak sulungnya, serta berkata kepadanya : "Anakku." Sahut Esau: "Ya, bapa."
\verse Berkatalah Ishak: "Lihat, aku sudah tua, aku tidak tahu bila hari kematianku.
\verse Maka sekarang, ambillah senjatamu, tabung panah dan busurmu, pergilah ke padang dan burulah bagiku seekor binatang;
\verse olahlah bagiku makanan yang enak, seperti yang kugemari, sesudah itu bawalah kepadaku, supaya kumakan, agar aku memberkati engkau, sebelum aku mati."
\verse Tetapi Ribka mendengarkannya, ketika Ishak berkata kepada Esau, anaknya. Setelah Esau pergi ke padang memburu seekor binatang untuk dibawanya kepada ayahnya,
\verse berkatalah Ribka kepada Yakub, anaknya: "Telah kudengar ayahmu berkata kepada Esau, kakakmu:
\verse Bawalah bagiku seekor binatang buruan dan olahlah bagiku makanan yang enak, supaya kumakan, dan supaya aku memberkati engkau di hadapan TUHAN, sebelum aku mati.
\verse Maka sekarang, anakku, dengarkanlah perkataanku seperti yang kuperintahkan kepadamu.
\verse Pergilah ke tempat kambing domba kita, ambillah dari sana dua anak kambing yang baik, maka aku akan mengolahnya menjadi makanan yang enak bagi ayahmu, seperti yang digemarinya.
\verse Bawalah itu kepada ayahmu, supaya dimakannya, agar dia memberkati engkau, sebelum ia mati."
\verse Lalu kata Yakub kepada Ribka, ibunya: "Tetapi Esau, kakakku, adalah seorang yang berbulu badannya, sedang aku ini kulitku licin.
\verse Mungkin ayahku akan meraba aku; maka nanti ia akan menyangka bahwa aku mau memperolok-olokkan dia; dengan demikian aku akan mendatangkan kutuk atas diriku dan bukan berkat."
\verse Tetapi ibunya berkata kepadanya: "Akulah yang menanggung kutuk itu, anakku; dengarkan saja perkataanku, pergilah ambil kambing-kambing itu."
\verse Lalu ia pergi mengambil kambing-kambing itu dan membawanya kepada ibunya; sesudah itu ibunya mengolah makanan yang enak, seperti yang digemari ayahnya.
\verse Kemudian Ribka mengambil pakaian yang indah kepunyaan Esau, anak sulungnya, pakaian yang disimpannya di rumah, lalu disuruhnyalah dikenakan oleh Yakub, anak bungsunya.
\verse Dan kulit anak kambing itu dipalutkannya pada kedua tangan Yakub dan pada lehernya yang licin itu.
\verse Lalu ia memberikan makanan yang enak dan roti yang telah diolahnya itu kepada Yakub, anaknya.
\verse Demikianlah Yakub masuk ke tempat ayahnya serta berkata: "Bapa!" Sahut ayahnya: "Ya, anakku; siapakah engkau?"
\verse Kata Yakub kepada ayahnya: "Akulah Esau, anak sulungmu. Telah kulakukan, seperti yang bapa katakan kepadaku. Bangunlah, duduklah dan makanlah daging buruan masakanku ini, agar bapa memberkati aku."
\verse Lalu Ishak berkata kepada anaknya itu: "Lekas juga engkau mendapatnya, anakku!" Jawabnya : "Karena TUHAN, Allahmu, membuat aku mencapai tujuanku."
\verse Lalu kata Ishak kepada Yakub: "Datanglah mendekat, anakku, supaya aku meraba engkau, apakah engkau ini anakku Esau atau bukan."
\verse Maka Yakub mendekati Ishak, ayahnya, dan ayahnya itu merabanya serta berkata: "Kalau suara, suara Yakub; kalau tangan, tangan Esau."
\verse Jadi Ishak tidak mengenal dia, karena tangannya berbulu seperti tangan Esau, kakaknya. Ishak hendak memberkati dia,
\verse tetapi ia masih bertanya: "Benarkah engkau ini anakku Esau?" Jawabnya: "Ya!"
\verse Lalu berkatalah Ishak: "Dekatkanlah makanan itu kepadaku, supaya kumakan daging buruan masakan anakku, agar aku memberkati engkau." Jadi didekatkannyalah makanan itu kepada ayahnya, lalu ia makan, dibawanya juga anggur kepadanya, lalu ia minum.
\verse Berkatalah Ishak, ayahnya, kepadanya: "Datanglah dekat-dekat dan ciumlah aku, anakku."
\verse Lalu datanglah Yakub dekat-dekat dan diciumnyalah ayahnya. Ketika Ishak mencium bau pakaian Yakub, diberkatinyalah dia, katanya: "Sesungguhnya bau anakku adalah sebagai bau padang yang diberkati TUHAN.
\verse Allah akan memberikan kepadamu embun yang dari langit dan tanah-tanah gemuk di bumi dan gandum serta anggur berlimpah-limpah.
\verse Bangsa-bangsa akan takluk kepadamu, dan suku-suku bangsa akan sujud kepadamu; jadilah tuan atas saudara-saudaramu, dan anak-anak ibumu akan sujud kepadamu. Siapa yang mengutuk engkau, terkutuklah ia, dan siapa yang memberkati engkau, diberkatilah ia."
\verse Setelah Ishak selesai memberkati Yakub, dan baru saja Yakub keluar meninggalkan Ishak, ayahnya, pulanglah Esau, kakaknya, dari berburu.
\verse Ia juga menyediakan makanan yang enak, lalu membawanya kepada ayahnya. Katanya kepada ayahnya: "Bapa, bangunlah dan makan daging buruan masakan anakmu, agar engkau memberkati aku."
\verse Tetapi kata Ishak, ayahnya, kepadanya: "Siapakah engkau ini?" Sahutnya: "Akulah anakmu, anak sulungmu, Esau."
\verse Lalu terkejutlah Ishak dengan sangat serta berkata: "Siapakah gerangan dia, yang memburu binatang itu dan yang telah membawanya kepadaku? Aku telah memakan semuanya, sebelum engkau datang, dan telah memberkati dia; dan dia akan tetap orang yang diberkati."
\verse Sesudah Esau mendengar perkataan ayahnya itu, meraung-raunglah ia dengan sangat keras dalam kepedihan hatinya serta berkata kepada ayahnya: "Berkatilah aku ini juga, ya bapa!"
\verse Jawab ayahnya: "Adikmu telah datang dengan tipu daya dan telah merampas berkat yang untukmu itu."
\verse Kata Esau: "Bukankah tepat namanya Yakub, karena ia telah dua kali menipu aku. Hak kesulunganku telah dirampasnya, dan sekarang dirampasnya pula berkat yang untukku." Lalu katanya: "Apakah bapa tidak mempunyai berkat lain bagiku?"
\verse Lalu Ishak menjawab Esau, katanya: "Sesungguhnya telah kuangkat dia menjadi tuan atas engkau, dan segala saudaranya telah kuberikan kepadanya menjadi hambanya, dan telah kubekali dia dengan gandum dan anggur; maka kepadamu, apa lagi yang dapat kuperbuat, ya anakku?"
\verse Kata Esau kepada ayahnya: "Hanya berkat yang satu itukah ada padamu, ya bapa? Berkatilah aku ini juga, ya bapa!" Dan dengan suara keras menangislah Esau.
\verse Lalu Ishak, ayahnya, menjawabnya: "Sesungguhnya tempat kediamanmu akan jauh dari tanah-tanah gemuk di bumi dan jauh dari embun dari langit di atas.
\verse Engkau akan hidup dari pedangmu dan engkau akan menjadi hamba adikmu. Tetapi akan terjadi kelak, apabila engkau berusaha sungguh-sungguh, maka engkau akan melemparkan kuk itu dari tengkukmu."
\verseWithSubheading{Yakub lari ke Mesopotamia} Esau menaruh dendam kepada Yakub karena berkat yang telah diberikan oleh ayahnya kepadanya, lalu ia berkata kepada dirinya sendiri: "Hari-hari berkabung karena kematian ayahku itu tidak akan lama lagi; pada waktu itulah Yakub, adikku, akan kubunuh."
\verse Ketika diberitahukan perkataan Esau, anak sulungnya itu kepada Ribka, maka disuruhnyalah memanggil Yakub, anak bungsunya, lalu berkata kepadanya: "Esau, kakakmu, bermaksud membalas dendam membunuh engkau.
\verse Jadi sekarang, anakku, dengarkanlah perkataanku, bersiaplah engkau dan larilah kepada Laban, saudaraku, ke Haran,
\verse dan tinggallah padanya beberapa waktu lamanya, sampai kegeraman
\verse dan kemarahan kakakmu itu surut dari padamu, dan ia lupa apa yang telah engkau perbuat kepadanya; kemudian aku akan menyuruh orang menjemput engkau dari situ. Mengapa aku akan kehilangan kamu berdua pada satu hari juga?"
\verse Kemudian Ribka berkata kepada Ishak: "Aku telah jemu hidup karena perempuan-perempuan Het itu; jikalau Yakub juga mengambil seorang isteri dari antara perempuan negeri ini, semacam perempuan Het itu, apa gunanya aku hidup lagi?"
\end{biblechapter}

\begin{biblechapter} % Kejadian 28
\verse Kemudian Ishak memanggil Yakub, lalu memberkati dia serta memesankan kepadanya, katanya: "Janganlah mengambil isteri dari perempuan Kanaan.
\verse Bersiaplah, pergilah ke Padan-Aram, ke rumah Betuel, ayah ibumu, dan ambillah dari situ seorang isteri dari anak-anak Laban, saudara ibumu.
\verse Moga-moga Allah Yang Mahakuasa memberkati engkau, membuat engkau beranak cucu dan membuat engkau menjadi banyak, sehingga engkau menjadi sekumpulan bangsa-bangsa.
\verse Moga-moga Ia memberikan kepadamu berkat yang untuk Abraham, kepadamu serta kepada keturunanmu, sehingga engkau memiliki negeri ini yang kaudiami sebagai orang asing, yang telah diberikan Allah kepada Abraham."
\verse Demikianlah Ishak melepas Yakub, lalu berangkatlah Yakub ke Padan-Aram, kepada Laban anak Betuel, orang Aram itu, saudara Ribka ibu Yakub dan Esau.
\verse Ketika Esau melihat, bahwa Ishak telah memberkati Yakub dan melepasnya ke Padan-Aram untuk mengambil isteri dari situ -- pada waktu ia memberkatinya ia telah memesankan kepada Yakub: "Janganlah ambil isteri dari antara perempuan Kanaan" --
\verse dan bahwa Yakub mendengarkan perkataan ayah dan ibunya, dan pergi ke Padan-Aram,
\verse maka Esau pun menyadari, bahwa perempuan Kanaan itu tidak disukai oleh Ishak, ayahnya.
\verse Sebab itu ia pergi kepada Ismael dan mengambil Mahalat menjadi isterinya, di samping kedua isterinya yang telah ada. Mahalat adalah anak Ismael anak Abraham, adik Nebayot.
\verseWithSubheading{Mimpi Yakub di Betel} Maka Yakub berangkat dari Bersyeba dan pergi ke Haran.
\verse Ia sampai di suatu tempat, dan bermalam di situ, karena matahari telah terbenam. Ia mengambil sebuah batu yang terletak di tempat itu dan dipakainya sebagai alas kepala, lalu membaringkan dirinya di tempat itu.
\verse Maka bermimpilah ia, di bumi ada didirikan sebuah tangga yang ujungnya sampai di langit, dan tampaklah malaikat-malaikat Allah turun naik di tangga itu.
\verse Berdirilah TUHAN di sampingnya dan berfirman: "Akulah TUHAN, Allah Abraham, nenekmu, dan Allah Ishak; tanah tempat engkau berbaring ini akan Kuberikan kepadamu dan kepada keturunanmu.
\verse Keturunanmu akan menjadi seperti debu tanah banyaknya, dan engkau akan mengembang ke sebelah timur, barat, utara dan selatan, dan olehmu serta keturunanmu semua kaum di muka bumi akan mendapat berkat.
\verse Sesungguhnya Aku menyertai engkau dan Aku akan melindungi engkau, ke mana pun engkau pergi, dan Aku akan membawa engkau kembali ke negeri ini, sebab Aku tidak akan meninggalkan engkau, melainkan tetap melakukan apa yang Kujanjikan kepadamu."
\verse Ketika Yakub bangun dari tidurnya, berkatalah ia: "Sesungguhnya TUHAN ada di tempat ini, dan aku tidak mengetahuinya."
\verse Ia takut dan berkata: "Alangkah dahsyatnya tempat ini. Ini tidak lain dari rumah Allah, ini pintu gerbang sorga."
\verse Keesokan harinya pagi-pagi Yakub mengambil batu yang dipakainya sebagai alas kepala dan mendirikan itu menjadi tugu dan menuang minyak ke atasnya.
\verse Ia menamai tempat itu Betel; dahulu nama kota itu Lus.
\verse Lalu bernazarlah Yakub: "Jika Allah akan menyertai dan akan melindungi aku di jalan yang kutempuh ini, memberikan kepadaku roti untuk dimakan dan pakaian untuk dipakai,
\verse sehingga aku selamat kembali ke rumah ayahku, maka TUHAN akan menjadi Allahku.
\verse Dan batu yang kudirikan sebagai tugu ini akan menjadi rumah Allah. Dari segala sesuatu yang Engkau berikan kepadaku akan selalu kupersembahkan sepersepuluh kepada-Mu."
\end{biblechapter}

\begin{biblechapter} % Kejadian 29
\verse Kemudian berangkatlah Yakub dari situ dan pergi ke negeri Bani Timur.
\verse Ketika ia memandang sekelilingnya, dilihatnya ada sebuah sumur di padang, dan ada tiga kumpulan kambing domba berbaring di dekatnya, sebab dari sumur itulah orang memberi minum kumpulan-kumpulan kambing domba itu. Adapun batu penutup sumur itu besar;
\verse dan apabila segala kumpulan kambing domba itu digiring berkumpul ke sana, maka gembala-gembala menggulingkan batu itu dari mulut sumur, lalu kambing domba itu diberi minum; kemudian dikembalikanlah batu itu lagi ke mulut sumur itu.
\verse Bertanyalah Yakub kepada mereka: "Saudara-saudara, dari manakah kamu ini?" Jawab mereka: "Kami ini dari Haran."
\verse Lagi katanya kepada mereka: "Kenalkah kamu Laban, cucu Nahor?" Jawab mereka: "Kami kenal."
\verse Selanjutnya katanya kepada mereka: "Selamatkah ia?" Jawab mereka: "Selamat! Tetapi lihat, itu datang anaknya perempuan, Rahel, dengan kambing dombanya."
\verse Lalu kata Yakub: "Hari masih siang, belum waktunya untuk mengumpulkan ternak; berilah minum kambing dombamu itu, kemudian pergilah menggembalakannya lagi."
\verse Tetapi jawab mereka: "Kami tidak dapat melakukan itu selama segala kumpulan binatang itu belum berkumpul; barulah batu itu digulingkan dari mulut sumur dan kami memberi minum kambing domba kami."
\verse Selagi ia berkata-kata dengan mereka, datanglah Rahel dengan kambing domba ayahnya, sebab dialah yang menggembalakannya.
\verse Ketika Yakub melihat Rahel, anak Laban saudara ibunya, serta kambing domba Laban, ia datang mendekat, lalu menggulingkan batu itu dari mulut sumur, dan memberi minum kambing domba itu.
\verse Kemudian Yakub mencium Rahel serta menangis dengan suara keras.
\verse Lalu Yakub menceritakan kepada Rahel, bahwa ia sanak saudara ayah Rahel, dan anak Ribka. Maka berlarilah Rahel menceritakannya kepada ayahnya.
\verse Segera sesudah Laban mendengar kabar tentang Yakub, anak saudaranya itu, berlarilah ia menyongsong dia, lalu mendekap dan mencium dia, kemudian membawanya ke rumahnya. Maka Yakub menceritakan segala hal ihwalnya kepada Laban.
\verse Kata Laban kepadanya: "Sesungguhnya engkau sedarah sedaging dengan aku." Maka tinggallah Yakub padanya genap sebulan lamanya.
\verse Kemudian berkatalah Laban kepada Yakub: "Masakan karena engkau adalah sanak saudaraku, engkau bekerja padaku dengan cuma-cuma? Katakanlah kepadaku apa yang patut menjadi upahmu."
\verse Laban mempunyai dua anak perempuan; yang lebih tua namanya Lea dan yang lebih muda namanya Rahel.
\verse Lea tidak berseri matanya, tetapi Rahel itu elok sikapnya dan cantik parasnya.
\verse Yakub cinta kepada Rahel, sebab itu ia berkata: "Aku mau bekerja padamu tujuh tahun lamanya untuk mendapat Rahel, anakmu yang lebih muda itu."
\verse Sahut Laban: "Lebih baiklah ia kuberikan kepadamu dari pada kepada orang lain; maka tinggallah padaku."
\verse Jadi bekerjalah Yakub tujuh tahun lamanya untuk mendapat Rahel itu, tetapi yang tujuh tahun itu dianggapnya seperti beberapa hari saja, karena cintanya kepada Rahel.
\verse Sesudah itu berkatalah Yakub kepada Laban: "Berikanlah kepadaku bakal isteriku itu, sebab jangka waktuku telah genap, supaya aku akan kawin dengan dia."
\verse Lalu Laban mengundang semua orang di tempat itu, dan mengadakan perjamuan.
\verse Tetapi pada waktu malam diambilnyalah Lea, anaknya, lalu dibawanya kepada Yakub. Maka Yakub pun menghampiri dia.
\verse Lagipula Laban memberikan Zilpa, budaknya perempuan, kepada Lea, anaknya itu, menjadi budaknya.
\verse Tetapi pada waktu pagi tampaklah bahwa itu Lea! Lalu berkatalah Yakub kepada Laban: "Apakah yang kauperbuat terhadap aku ini? Bukankah untuk mendapat Rahel aku bekerja padamu? Mengapa engkau menipu aku?"
\verse Jawab Laban: "Tidak biasa orang berbuat demikian di tempat kami ini, mengawinkan adiknya lebih dahulu dari pada kakaknya.
\verse Genapilah dahulu tujuh hari perkawinanmu dengan anakku ini; kemudian anakku yang lain pun akan diberikan kepadamu sebagai upah, asal engkau bekerja pula padaku tujuh tahun lagi."
\verse Maka Yakub berbuat demikian; ia menggenapi ketujuh hari perkawinannya dengan Lea, kemudian Laban memberikan kepadanya Rahel, anaknya itu, menjadi isterinya.
\verse Lagipula Laban memberikan Bilha, budaknya perempuan, kepada Rahel, anaknya itu, menjadi budaknya.
\verse Yakub menghampiri Rahel juga, malah ia lebih cinta kepada Rahel dari pada kepada Lea. Demikianlah ia bekerja pula pada Laban tujuh tahun lagi.
\verseWithSubheading{Anak-anak Yakub} Ketika TUHAN melihat, bahwa Lea tidak dicintai, dibuka-Nyalah kandungannya, tetapi Rahel mandul.
\verse Lea mengandung, lalu melahirkan seorang anak laki-laki, dan menamainya Ruben, sebab katanya: "Sesungguhnya TUHAN telah memperhatikan kesengsaraanku; sekarang tentulah aku akan dicintai oleh suamiku."
\verse Mengandung pulalah ia, lalu melahirkan seorang anak laki-laki, maka ia berkata: "Sesungguhnya, TUHAN telah mendengar, bahwa aku tidak dicintai, lalu diberikan-Nya pula anak ini kepadaku." Maka ia menamai anak itu Simeon.
\verse Mengandung pulalah ia, lalu melahirkan seorang anak laki-laki, maka ia berkata: "Sekali ini suamiku akan lebih erat kepadaku, karena aku telah melahirkan tiga anak laki-laki baginya." Itulah sebabnya ia menamai anak itu Lewi.
\verse Mengandung pulalah ia, lalu melahirkan seorang anak laki-laki, maka ia berkata: "Sekali ini aku akan bersyukur kepada TUHAN." Itulah sebabnya ia menamai anak itu Yehuda. Sesudah itu ia tidak melahirkan lagi.
\end{biblechapter}

\begin{biblechapter} % Kejadian 30
\verse Ketika dilihat Rahel, bahwa ia tidak melahirkan anak bagi Yakub, cemburulah ia kepada kakaknya itu, lalu berkata kepada Yakub: "Berikanlah kepadaku anak; kalau tidak, aku akan mati."
\verse Maka bangkitlah amarah Yakub terhadap Rahel dan ia berkata: "Akukah pengganti Allah, yang telah menghalangi engkau mengandung?"
\verse Kata Rahel: "Ini Bilha, budakku perempuan, hampirilah dia, supaya ia melahirkan anak di pangkuanku, dan supaya oleh dia aku pun mempunyai keturunan."
\verse Maka diberikannyalah Bilha, budaknya itu, kepada Yakub menjadi isterinya dan Yakub menghampiri budak itu.
\verse Bilha mengandung, lalu melahirkan seorang anak laki-laki bagi Yakub.
\verse Berkatalah Rahel: "Allah telah memberikan keadilan kepadaku, juga telah didengarkan-Nya permohonanku dan diberikan-Nya kepadaku seorang anak laki-laki." Itulah sebabnya ia menamai anak itu Dan.
\verse Mengandung pulalah Bilha, budak perempuan Rahel, lalu melahirkan anak laki-laki yang kedua bagi Yakub.
\verse Berkatalah Rahel: "Aku telah sangat hebat bergulat dengan kakakku, dan aku pun menang." Maka ia menamai anak itu Naftali.
\verse Ketika dilihat Lea, bahwa ia tidak melahirkan lagi, diambilnyalah Zilpa, budaknya perempuan, dan diberikannya kepada Yakub menjadi isterinya.
\verse Dan Zilpa, budak perempuan Lea, melahirkan seorang anak laki-laki bagi Yakub.
\verse Berkatalah Lea: "Mujur telah datang." Maka ia menamai anak itu Gad.
\verse Dan Zilpa, budak perempuan Lea, melahirkan anak laki-laki yang kedua bagi Yakub.
\verse Berkatalah Lea: "Aku ini berbahagia! Tentulah perempuan-perempuan akan menyebutkan aku berbahagia." Maka ia menamai anak itu Asyer.
\verse Ketika Ruben pada musim menuai gandum pergi berjalan-jalan, didapatinyalah di padang buah dudaim, lalu dibawanya kepada Lea, ibunya. Kata Rahel kepada Lea: "Berilah aku beberapa buah dudaim yang didapat oleh anakmu itu."
\verse Jawab Lea kepadanya: "Apakah belum cukup bagimu mengambil suamiku? Sekarang pula mau mengambil lagi buah dudaim anakku?" Kata Rahel: "Kalau begitu biarlah ia tidur dengan engkau pada malam ini sebagai ganti buah dudaim anakmu itu."
\verse Ketika Yakub pada waktu petang datang dari padang, pergilah Lea mendapatkannya, sambil berkata: "Engkau harus singgah kepadaku malam ini, sebab memang engkau telah kusewa dengan buah dudaim anakku." Sebab itu tidurlah Yakub dengan Lea pada malam itu.
\verse Lalu Allah mendengarkan permohonan Lea. Lea mengandung dan melahirkan anak laki-laki yang kelima bagi Yakub.
\verse Lalu kata Lea: "Allah telah memberi upahku, karena aku telah memberi budakku perempuan kepada suamiku." Maka ia menamai anak itu Isakhar.
\verse Kemudian Lea mengandung pula dan melahirkan anak laki-laki yang keenam bagi Yakub.
\verse Berkatalah Lea: "Allah telah memberikan hadiah yang indah kepadaku; sekali ini suamiku akan tinggal bersama-sama dengan aku, karena aku telah melahirkan enam orang anak laki-laki baginya." Maka ia menamai anak itu Zebulon.
\verse Sesudah itu ia melahirkan seorang anak perempuan dan menamai anak itu Dina.
\verse Lalu ingatlah Allah akan Rahel; Allah mendengarkan permohonannya serta membuka kandungannya.
\verse Maka mengandunglah Rahel dan melahirkan seorang anak laki-laki. Berkatalah ia: "Allah telah menghapuskan aibku."
\verse Maka ia menamai anak itu Yusuf, sambil berkata: "Mudah-mudahan TUHAN menambah seorang anak laki-laki lagi bagiku."
\verseWithSubheading{Yakub memperoleh ternak} Setelah Rahel melahirkan Yusuf, berkatalah Yakub kepada Laban: "Izinkanlah aku pergi, supaya aku pulang ke tempat kelahiranku dan ke negeriku.
\verse Berikanlah isteri-isteriku dan anak-anakku, yang menjadi upahku selama aku bekerja padamu, supaya aku pulang, sebab engkau tahu, betapa keras aku bekerja padamu."
\verse Tetapi Laban berkata kepadanya: "Sekiranya aku mendapat kasihmu! Telah nyata kepadaku, bahwa TUHAN memberkati aku karena engkau."
\verse Lagi katanya: "Tentukanlah upahmu yang harus kubayar, maka aku akan memberikannya."
\verse Sahut Yakub kepadanya: "Engkau sendiri tahu, bagaimana aku bekerja padamu, dan bagaimana keadaan ternakmu dalam penjagaanku,
\verse sebab harta milikmu tidak begitu banyak sebelum aku datang, tetapi sekarang telah berkembang dengan sangat, dan TUHAN telah memberkati engkau sejak aku berada di sini; jadi, bilakah dapat aku bekerja untuk rumah tanggaku sendiri?"
\verse Kata Laban: "Apakah yang harus kuberikan kepadamu?" Jawab Yakub: "Tidak usah kauberikan apa-apa kepadaku; aku mau lagi menggembalakan kambing dombamu dan menjaganya, asal engkau mengizinkan hal ini kepadaku:
\verse Hari ini aku akan lewat dari tengah-tengah segala kambing dombamu dan akan mengasingkan dari situ setiap binatang yang berbintik-bintik dan berbelang-belang; segala domba yang hitam dan segala kambing yang berbelang-belang dan berbintik-bintik, itulah upahku.
\verse Dan kejujuranku akan terbukti di kemudian hari, apabila engkau datang memeriksa upahku: Segala yang tidak berbintik-bintik atau berbelang-belang di antara kambing-kambing dan yang tidak hitam di antara domba-domba, anggaplah itu tercuri olehku."
\verse Kemudian kata Laban: "Baik, jadilah seperti perkataanmu itu."
\verse Lalu diasingkannyalah pada hari itu kambing-kambing jantan yang bercoreng-coreng dan berbelang-belang dan segala kambing yang berbintik-bintik dan berbelang-belang, segala yang ada warna putih pada badannya, serta segala yang hitam di antara domba-domba, dan diserahkannyalah semuanya itu kepada anak-anaknya untuk dijaga.
\verse Kemudian Laban menentukan jarak tiga hari perjalanan jauhnya antara dia dan Yakub, maka tetaplah Yakub menggembalakan kambing domba yang tinggal itu.
\verse Lalu Yakub mengambil dahan hijau dari pohon hawar, pohon badam dan pohon berangan, dikupasnyalah dahan-dahan itu sehingga berbelang-belang, sampai yang putihnya kelihatan.
\verse Ia meletakkan dahan-dahan yang dikupasnya itu dalam palungan, dalam tempat minum, ke mana kambing domba itu datang minum, sehingga tepat di depan kambing domba itu. Adapun kambing domba itu suka berkelamin pada waktu datang minum.
\verse Jika kambing domba itu berkelamin dekat dahan-dahan itu, maka anaknya bercoreng-coreng, berbintik-bintik dan berbelang-belang.
\verse Kemudian Yakub memisahkan domba-domba itu, dihadapkannya kepala-kepala kambing domba itu kepada yang bercoreng-coreng dan kepada segala yang hitam di antara kambing domba Laban. Demikianlah ia beroleh kumpulan-kumpulan hewan baginya sendiri, dan tidak ditempatkannya pada kambing domba Laban.
\verse Dan setiap kali, apabila berkelamin kambing domba yang kuat, maka Yakub meletakkan dahan-dahan itu ke dalam palungan di depan mata kambing domba itu, supaya berkelamin dekat dahan-dahan itu.
\verse Tetapi apabila datang kambing domba yang lemah, ia tidak meletakkan dahan-dahan itu ke dalamnya. Jadi hewan yang lemah untuk Laban dan yang kuat untuk Yakub.
\verse Maka sangatlah bertambah-tambah harta Yakub, dan ia mempunyai banyak kambing domba, budak perempuan dan laki-laki, unta dan keledai.
\end{biblechapter}

\begin{biblechapter} % Kejadian 31
\verseWithHeading{Yakub lari meninggalkan Laban} Kedengaranlah kepada Yakub anak-anak Laban berkata demikian: "Yakub telah mengambil segala harta milik ayah kita dan dari harta itulah ia membangun segala kekayaannya."
\verse Lagi kelihatan kepada Yakub dari muka Laban, bahwa Laban tidak lagi seperti yang sudah-sudah kepadanya.
\verse Lalu berfirmanlah TUHAN kepada Yakub: "Pulanglah ke negeri nenek moyangmu dan kepada kaummu, dan Aku akan menyertai engkau."
\verse Sesudah itu Yakub menyuruh memanggil Rahel dan Lea untuk datang ke padang, ke tempat kambing dombanya,
\verse lalu ia berkata kepada mereka: "Telah kulihat dari muka ayahmu, bahwa ia tidak lagi seperti yang sudah-sudah kepadaku, tetapi Allah ayahku menyertai aku.
\verse Juga kamu sendiri tahu, bahwa aku telah bekerja sekuat-kuatku pada ayahmu.
\verse Tetapi ayahmu telah berlaku curang kepadaku dan telah sepuluh kali mengubah upahku, tetapi Allah tidak membiarkan dia berbuat jahat kepadaku.
\verse Apabila ia berkata: yang berbintik-bintiklah akan menjadi upahmu, maka segala kambing domba itu beroleh anak yang berbintik-bintik; dan apabila ia berkata: yang bercoreng-corenglah akan menjadi upahmu, maka segala kambing domba itu beroleh anak yang bercoreng-coreng.
\verse Demikianlah Allah mengambil ternak ayahmu dan memberikannya kepadaku.
\verse Pada suatu kali pada masa kambing domba itu suka berkelamin, maka aku bermimpi dan melihat, bahwa jantan-jantan yang menjantani kambing domba itu bercoreng-coreng, berbintik-bintik dan berbelang-belang.
\verse Dan Malaikat Allah berfirman kepadaku dalam mimpi itu: Yakub! Jawabku: Ya Tuhan!
\verse Lalu Ia berfirman: Angkatlah mukamu dan lihatlah, bahwa segala jantan yang menjantani kambing domba itu bercoreng-coreng, berbintik-bintik dan berbelang-belang, sebab telah Kulihat semua yang dilakukan oleh Laban itu kepadamu.
\verse Akulah Allah yang di Betel itu, di mana engkau mengurapi tugu, dan di mana engkau bernazar kepada-Ku; maka sekarang, bersiaplah engkau, pergilah dari negeri ini dan pulanglah ke negeri sanak saudaramu."
\verse Lalu Rahel dan Lea menjawab Yakub, katanya: "Bukankah tidak ada lagi bagian atau warisan kami dalam rumah ayah kami?
\verse Bukankah kami ini dianggapnya sebagai orang asing, karena ia telah menjual kami? Juga bagian kami telah dihabiskannya sama sekali.
\verse Tetapi segala kekayaan, yang telah diambil Allah dari ayah kami, adalah milik kami dan anak-anak kami; maka sekarang, perbuatlah segala yang difirmankan Allah kepadamu."
\verse Lalu bersiaplah Yakub, dinaikkannya anak-anaknya dan isteri-isterinya ke atas unta,
\verse digiringnya seluruh ternaknya dan segala apa yang telah diperolehnya, yakni ternak kepunyaannya, yang telah diperolehnya di Padan-Aram, dengan maksud pergi kepada Ishak, ayahnya, ke tanah Kanaan.
\verse Adapun Laban telah pergi menggunting bulu domba-dombanya. Ketika itulah Rahel mencuri terafim ayahnya.
\verse Dan Yakub mengakali Laban, orang Aram itu, dengan tidak memberitahukan kepadanya, bahwa ia mau lari.
\verse Demikianlah ia lari dengan segala harta miliknya. Ia berangkat, menyeberangi sungai Efrat dan berjalan menuju pegunungan Gilead.
\verseWithSubheading{Laban mengejar Yakub} Ketika pada hari ketiga dikabarkan kepada Laban, bahwa Yakub telah lari,
\verse dibawanyalah sanak saudaranya bersama-sama, dikejarnya Yakub tujuh hari perjalanan jauhnya, lalu ia dapat menyusulnya di pegunungan Gilead.
\verse Pada waktu malam datanglah Allah dalam suatu mimpi kepada Laban, orang Aram itu, serta berfirman kepadanya: "Jagalah baik-baik, supaya engkau jangan mengatai Yakub dengan sepatah kata pun."
\verse Ketika Laban sampai kepada Yakub, -- Yakub telah memasang kemahnya di pegunungan, juga Laban dengan sanak saudaranya telah memasang kemahnya di pegunungan Gilead --
\verse berkatalah Laban kepada Yakub: "Apakah yang kauperbuat ini, maka engkau mengakali aku dan mengangkut anak-anakku perempuan sebagai orang tawanan?
\verse Mengapa engkau lari diam-diam dan mengakali aku? Mengapa engkau tidak memberitahu kepadaku, supaya aku menghantarkan engkau dengan sukacita dan nyanyian dengan rebana dan kecapi?
\verse Lagipula engkau tidak memberikan aku kesempatan untuk mencium cucu-cucuku laki-laki dan anak-anakku perempuan. Memang bodoh perbuatanmu itu.
\verse Aku ini berkuasa untuk berbuat jahat kepadamu, tetapi Allah ayahmu telah berfirman kepadaku tadi malam: Jagalah baik-baik, jangan engkau mengatai Yakub dengan sepatah kata pun.
\verse Maka sekarang, kalau memang engkau harus pergi, semata-mata karena sangat rindu ke rumah ayahmu, mengapa engkau mencuri dewa-dewaku?"
\verse Lalu Yakub menjawab Laban: "Aku takut, karena pikirku, jangan-jangan engkau merampas anak-anakmu itu dari padaku.
\verse Tetapi pada siapa engkau menemui dewa-dewamu itu, janganlah ia hidup lagi. Periksalah di depan saudara-saudara kita segala barang yang ada padaku dan ambillah barangmu." Sebab Yakub tidak tahu, bahwa Rahel yang mencuri terafim itu.
\verse Lalu masuklah Laban ke dalam kemah Yakub dan ke dalam kemah Lea dan ke dalam kemah kedua budak perempuan itu, tetapi terafim itu tidak ditemuinya. Setelah keluar dari kemah Lea, ia masuk ke dalam kemah Rahel.
\verse Tetapi Rahel telah mengambil terafim itu dan memasukkannya ke dalam pelana untanya, dan duduk di atasnya. Laban menggeledah seluruh kemah itu, tetapi terafim itu tidak ditemuinya.
\verse Lalu kata Rahel kepada ayahnya: "Janganlah bapa marah, karena aku tidak dapat bangun berdiri di depanmu, sebab aku sedang haid." Dan Laban mencari dengan teliti, tetapi ia tidak menemui terafim itu.
\verse Lalu hati Yakub panas dan ia bertengkar dengan Laban. Ia berkata kepada Laban: "Apakah kesalahanku, apakah dosaku, maka engkau memburu aku sehebat itu?
\verse Engkau telah menggeledah segala barangku, sekarang apakah yang kautemui dari segala barang rumahmu? Letakkanlah di sini di depan saudara-saudaraku dan saudara-saudaramu, supaya mereka mengadili antara kita berdua.
\verse Selama dua puluh tahun ini aku bersama-sama dengan engkau; domba dan kambing betinamu tidak pernah keguguran dan jantan dari kambing dombamu tidak pernah kumakan.
\verse Yang diterkam oleh binatang buas tidak pernah kubawa kepadamu, aku sendiri yang menggantinya; yang dicuri orang, baik waktu siang, baik waktu malam, selalu engkau tuntut dari padaku.
\verse Aku dimakan panas hari waktu siang dan kedinginan waktu malam, dan mataku jauh dari pada tertidur.
\verse Selama dua puluh tahun ini aku di rumahmu; aku telah bekerja padamu empat belas tahun lamanya untuk mendapat kedua anakmu dan enam tahun untuk mendapat ternakmu, dan engkau telah sepuluh kali mengubah upahku.
\verse Seandainya Allah ayahku, Allah Abraham dan Yang Disegani oleh Ishak tidak menyertai aku, tentulah engkau sekarang membiarkan aku pergi dengan tangan hampa; tetapi kesengsaraanku dan jerih payahku telah diperhatikan Allah dan Ia telah menjatuhkan putusan tadi malam."
\verseWithSubheading{Perjanjian antara Yakub dan Laban} Lalu Laban menjawab Yakub: "Perempuan-perempuan ini anakku dan anak-anak lelaki ini cucuku dan ternak ini ternakku, bahkan segala yang kaulihat di sini adalah milikku; jadi apakah yang dapat kuperbuat sekarang kepada anak-anakku ini atau kepada anak-anak yang dilahirkan mereka?
\verse Maka sekarang, marilah kita mengikat perjanjian, aku dan engkau, supaya itu menjadi kesaksian antara aku dan engkau."
\verse Kemudian Yakub mengambil sebuah batu dan didirikannya menjadi tugu.
\verse Selanjutnya berkatalah Yakub kepada sanak saudaranya: "Kumpulkanlah batu." Maka mereka mengambil batu dan membuat timbunan, lalu makanlah mereka di sana di dekat timbunan itu.
\verse Laban menamai timbunan batu itu Yegar-Sahaduta, tetapi Yakub menamainya Galed.
\verse Lalu kata Laban: "Timbunan batu inilah pada hari ini menjadi kesaksian antara aku dan engkau." Itulah sebabnya timbunan itu dinamainya Galed,
\verse dan juga Mizpa, sebab katanya: "TUHAN kiranya berjaga-jaga antara aku dan engkau, apabila kita berjauhan.
\verse Jika engkau mengaibkan anak-anakku, dan jika engkau mengambil isteri lain di samping anak-anakku itu, ingatlah, walaupun tidak ada orang dekat kita, Allah juga yang menjadi saksi antara aku dan engkau."
\verse Selanjutnya kata Laban kepada Yakub: "Inilah timbunan batu, dan inilah tugu yang kudirikan antara aku dan engkau --
\verse timbunan batu dan tugu inilah menjadi kesaksian, bahwa aku tidak akan melewati timbunan batu ini mendapatkan engkau, dan bahwa engkau pun tidak akan melewati timbunan batu dan tugu ini mendapatkan aku, dengan berniat jahat.
\verse Allah Abraham dan Allah Nahor, Allah ayah mereka, kiranya menjadi hakim antara kita." Lalu Yakub bersumpah demi Yang Disegani oleh Ishak, ayahnya.
\verse Dan Yakub mempersembahkan korban sembelihan di gunung itu. Ia mengundang makan sanak saudaranya, lalu mereka makan serta bermalam di gunung itu.
\verse Keesokan harinya pagi-pagi Laban mencium cucu-cucunya dan anak-anaknya serta memberkati mereka, kemudian pulanglah Laban kembali ke tempat tinggalnya.
\end{biblechapter}

\begin{biblechapter} % Kejadian 32
\verseWithHeading{Yakub takut bertemu dengan Esau} Yakub melanjutkan perjalanannya, lalu bertemulah malaikat-malaikat Allah dengan dia.
\verse Ketika Yakub melihat mereka, berkatalah ia: "Ini bala tentara Allah." Sebab itu dinamainyalah tempat itu Mahanaim.
\verse Sesudah itu Yakub menyuruh utusannya berjalan lebih dahulu mendapatkan Esau, kakaknya, ke tanah Seir, daerah Edom.
\verse Ia memerintahkan kepada mereka: "Beginilah kamu katakan kepada tuanku, kepada Esau: Beginilah kata hambamu Yakub: Aku telah tinggal pada Laban sebagai orang asing dan diam di situ selama ini.
\verse Aku telah mempunyai lembu sapi, keledai dan kambing domba, budak laki-laki dan perempuan, dan aku menyuruh memberitahukan hal ini kepada tuanku, supaya aku mendapat kasihmu."
\verse Kemudian pulanglah para utusan itu kepada Yakub dan berkata: "Kami telah sampai kepada kakakmu, kepada Esau, dan ia pun sedang di jalan menemui engkau, diiringi oleh empat ratus orang."
\verse Lalu sangat takutlah Yakub dan merasa sesak hati; maka dibaginyalah orang-orangnya yang bersama-sama dengan dia, kambing dombanya, lembu sapi dan untanya menjadi dua pasukan.
\verse Sebab pikirnya: "Jika Esau datang menyerang pasukan yang satu, sehingga terpukul kalah, maka pasukan yang tinggal akan terluput."
\verse Kemudian berkatalah Yakub: "Ya Allah nenekku Abraham dan Allah ayahku Ishak, ya TUHAN, yang telah berfirman kepadaku: Pulanglah ke negerimu serta kepada sanak saudaramu dan Aku akan berbuat baik kepadamu --
\verse sekali-kali aku tidak layak untuk menerima segala kasih dan kesetiaan yang Engkau tunjukkan kepada hamba-Mu ini, sebab aku membawa hanya tongkatku ini waktu aku menyeberangi sungai Yordan ini, tetapi sekarang telah menjadi dua pasukan.
\verse Lepaskanlah kiranya aku dari tangan kakakku, dari tangan Esau, sebab aku takut kepadanya, jangan-jangan ia datang membunuh aku, juga ibu-ibu dengan anak-anaknya.
\verse Bukankah Engkau telah berfirman: Tentu Aku akan berbuat baik kepadamu dan menjadikan keturunanmu sebagai pasir di laut, yang karena banyaknya tidak dapat dihitung."
\verse Lalu bermalamlah ia di sana pada malam itu. Kemudian diambilnyalah dari apa yang ada padanya suatu persembahan untuk Esau, kakaknya,
\verse yaitu dua ratus kambing betina dan dua puluh kambing jantan, dua ratus domba betina dan dua puluh domba jantan,
\verse tiga puluh unta yang sedang menyusui beserta anak-anaknya, empat puluh lembu betina dan sepuluh lembu jantan, dua puluh keledai betina dan sepuluh keledai jantan.
\verse Diserahkannyalah semuanya itu kepada budak-budaknya untuk dijaga, tiap-tiap kumpulan tersendiri, dan ia berkata kepada mereka: "Berjalanlah kamu lebih dahulu dan jagalah supaya ada jarak antara kumpulan yang satu dengan kumpulan yang lain."
\verse Diperintahkannyalah kepada yang paling di muka: "Apabila Esau, kakakku, bertemu dengan engkau dan bertanya kepadamu: Siapakah tuanmu? dan ke manakah engkau pergi? dan milik siapakah ternak yang di depanmu itu? --
\verse jawablah: milik hambamu Yakub; inilah persembahan yang dikirim kepada tuanku Esau, dan Yakub sendiri pun ada di belakang kami."
\verse Begitulah diperintahkannya baik kepada yang kedua maupun kepada yang ketiga dan kepada sekalian orang yang berjalan menggiring kumpulan hewan itu, katanya: "Seperti perkataanku tadilah kamu katakan kepada Esau, apabila kamu berjumpa dengan dia;
\verse dan kamu harus mengatakan juga: Hambamu Yakub sendiri ada di belakang kami." Sebab pikir Yakub: "Baiklah aku mendamaikan hatinya dengan persembahan yang diantarkan lebih dahulu, kemudian barulah aku akan melihat mukanya; mungkin ia akan menerima aku dengan baik."
\verse Jadi persembahan itu diantarkan lebih dahulu, tetapi ia sendiri bermalam pada malam itu di tempat perkemahannya.
\verseWithSubheading{Pergumulan Yakub dengan Allah} Pada malam itu Yakub bangun dan ia membawa kedua isterinya, kedua budaknya perempuan dan kesebelas anaknya, dan menyeberang di tempat penyeberangan sungai Yabok.
\verse Sesudah ia menyeberangkan mereka, ia menyeberangkan juga segala miliknya.
\verse Lalu tinggallah Yakub seorang diri. Dan seorang laki-laki bergulat dengan dia sampai fajar menyingsing.
\verse Ketika orang itu melihat, bahwa ia tidak dapat mengalahkannya, ia memukul sendi pangkal paha Yakub, sehingga sendi pangkal paha itu terpelecok, ketika ia bergulat dengan orang itu.
\verse Lalu kata orang itu: "Biarkanlah aku pergi, karena fajar telah menyingsing." Sahut Yakub: "Aku tidak akan membiarkan engkau pergi, jika engkau tidak memberkati aku."
\verse Bertanyalah orang itu kepadanya: "Siapakah namamu?" Sahutnya: "Yakub."
\verse Lalu kata orang itu: "Namamu tidak akan disebutkan lagi Yakub, tetapi Israel, sebab engkau telah bergumul melawan Allah dan manusia, dan engkau menang."
\verse Bertanyalah Yakub: "Katakanlah juga namamu." Tetapi sahutnya: "Mengapa engkau menanyakan namaku?" Lalu diberkatinyalah Yakub di situ.
\verse Yakub menamai tempat itu Pniel, sebab katanya: "Aku telah melihat Allah berhadapan muka, tetapi nyawaku tertolong!"
\verse Lalu tampaklah kepadanya matahari terbit, ketika ia telah melewati Pniel; dan Yakub pincang karena pangkal pahanya.
\verse Itulah sebabnya sampai sekarang orang Israel tidak memakan daging yang menutupi sendi pangkal paha, karena Dia telah memukul sendi pangkal paha Yakub, pada otot pangkal pahanya.
\end{biblechapter}

\begin{biblechapter} % Kejadian 33
\verseWithHeading{Yakub berbaik kembali dengan Esau} Yakub pun melayangkan pandangnya, lalu dilihatnyalah Esau datang dengan diiringi oleh empat ratus orang. Maka diserahkannyalah sebagian dari anak-anak itu kepada Lea dan sebagian kepada Rahel serta kepada kedua budak perempuan itu.
\verse Ia menempatkan budak-budak perempuan itu beserta anak-anak mereka di muka, Lea beserta anak-anaknya di belakang mereka, dan Rahel beserta Yusuf di belakang sekali.
\verse Dan ia sendiri berjalan di depan mereka dan ia sujud sampai ke tanah tujuh kali, hingga ia sampai ke dekat kakaknya itu.
\verse Tetapi Esau berlari mendapatkan dia, didekapnya dia, dipeluk lehernya dan diciumnya dia, lalu bertangis-tangisanlah mereka.
\verse Kemudian Esau melayangkan pandangnya, dilihatnyalah perempuan-perempuan dan anak-anak itu, lalu ia bertanya: "Siapakah orang-orang yang beserta engkau itu?" Jawab Yakub: "Anak-anak yang telah dikaruniakan Allah kepada hambamu ini."
\verse Sesudah itu mendekatlah budak-budak perempuan itu beserta anak-anaknya, lalu mereka sujud.
\verse Mendekat jugalah Lea beserta anak-anaknya, dan mereka pun sujud. Kemudian mendekatlah Yusuf beserta Rahel, dan mereka juga sujud.
\verse Berkatalah Esau: "Apakah maksudmu dengan seluruh pasukan, yang telah bertemu dengan aku tadi?" Jawabnya: "Untuk mendapat kasih tuanku."
\verse Tetapi kata Esau: "Aku mempunyai banyak, adikku; peganglah apa yang ada padamu."
\verse Tetapi kata Yakub: "Janganlah kiranya demikian; jikalau aku telah mendapat kasihmu, terimalah persembahanku ini dari tanganku, karena memang melihat mukamu adalah bagiku serasa melihat wajah Allah, dan engkau pun berkenan menyambut aku.
\verse Terimalah kiranya pemberian tanda salamku ini, yang telah kubawa kepadamu, sebab Allah telah memberi karunia kepadaku dan aku pun mempunyai segala-galanya." Lalu dibujuk-bujuknyalah Esau, sehingga diterimanya.
\verse Kata Esau: "Baiklah kita berangkat berjalan terus; aku akan menyertai engkau."
\verse Tetapi Yakub berkata kepadanya: "Tuanku maklum, bahwa anak-anak ini masih kurang kuat, dan bahwa beserta aku ada kambing domba dan lembu sapi yang masih menyusui, jika diburu-buru, satu hari saja, maka seluruh kumpulan binatang itu akan mati.
\verse Biarlah kiranya tuanku berjalan lebih dahulu dari hambamu ini dan aku mau dengan hati-hati beringsut maju menurut langkah hewan, yang berjalan di depanku dan menurut langkah anak-anak, sampai aku tiba pada tuanku di Seir."
\verse Lalu kata Esau: "Kalau begitu, baiklah kutinggalkan padamu beberapa orang dari pengiringku." Tetapi Yakub berkata: "Tidak usah demikian! Biarlah aku mendapat kasih tuanku saja."
\verse Jadi pulanglah Esau pada hari itu berjalan ke Seir.
\verse Tetapi Yakub berangkat ke Sukot, lalu mendirikan rumah, dan untuk ternaknya dibuatnya gubuk-gubuk. Itulah sebabnya tempat itu dinamai Sukot.
\verse Dalam perjalanannya dari Padan-Aram sampailah Yakub dengan selamat ke Sikhem, di tanah Kanaan, lalu ia berkemah di sebelah timur kota itu.
\verse Kemudian dibelinyalah dari anak-anak Hemor, bapa Sikhem, sebidang tanah, tempat ia memasang kemahnya, dengan harga seratus kesita.
\verse Ia mendirikan mezbah di situ dan dinamainya itu: "Allah Israel ialah Allah."
\end{biblechapter}

\begin{biblechapter} % Kejadian 34
\verseWithHeading{Dina dan Sikhem} Pada suatu kali pergilah Dina, anak perempuan Lea yang dilahirkannya bagi Yakub, mengunjungi perempuan-perempuan di negeri itu.
\verse Ketika itu terlihatlah ia oleh Sikhem, anak Hemor, orang Hewi, raja negeri itu, lalu Dina itu dilarikannya dan diperkosanya.
\verse Tetapi terikatlah hatinya kepada Dina, anak Yakub; ia cinta kepada gadis itu, lalu menenangkan hati gadis itu.
\verse Sebab itu berkatalah Sikhem kepada Hemor, ayahnya: "Ambillah bagiku gadis ini untuk menjadi isteriku."
\verse Kedengaranlah kepada Yakub, bahwa Sikhem mencemari Dina. Tetapi anak-anaknya ada di padang menjaga ternaknya, jadi Yakub mendiamkan soal itu sampai mereka pulang.
\verse Lalu Hemor ayah Sikhem, pergi mendapatkan Yakub untuk berbicara dengan dia.
\verse Sementara itu anak-anak Yakub pulang dari padang, dan sesudah mendengar peristiwa itu orang-orang ini sakit hati dan sangat marah karena Sikhem telah berbuat noda di antara orang Israel dengan memperkosa anak perempuan Yakub, sebab yang demikian itu tidak patut dilakukan.
\verse Berbicaralah Hemor kepada mereka itu: "Hati Sikhem anakku mengingini anakmu; kiranya kamu memberikan dia kepadanya menjadi isterinya
\verse dan biarlah kita ambil-mengambil: berikanlah gadis-gadis kamu kepada kami dan ambillah gadis-gadis kami.
\verse Tinggallah pada kami: negeri ini terbuka untuk kamu; tinggallah di sini, jalanilah negeri ini dengan bebas, dan menetaplah di sini."
\verse Lalu Sikhem berkata kepada ayah anak itu dan kepada kakak-kakaknya: "Biarlah kiranya aku mendapat kasihmu, aku akan memberikan kepadamu apa yang kamu minta;
\verse walaupun kamu bebankan kepadaku uang jujuran dan uang mahar seberapa banyak pun, aku akan memberikan apa yang kamu minta; tetapi berilah gadis itu kepadaku menjadi isteriku."
\verse Lalu anak-anak Yakub menjawab Sikhem dan Hemor, ayahnya, dengan tipu muslihat. Karena Sikhem telah mencemari Dina, adik mereka itu,
\verse berkatalah mereka kepada kedua orang itu: "Kami tidak dapat berbuat demikian, memberikan adik kami kepada seorang laki-laki yang tidak bersunat, sebab hal itu aib bagi kami.
\verse Hanyalah dengan syarat ini kami dapat menyetujui permintaanmu: kamu harus sama seperti kami, yaitu setiap laki-laki di antara kamu harus disunat,
\verse barulah kami akan memberikan gadis-gadis kami kepada kamu dan mengambil gadis-gadis kamu; maka kami akan tinggal padamu, dan kita akan menjadi satu bangsa.
\verse Tetapi jika kamu tidak mendengarkan perkataan kami dan kamu tidak disunat, maka kami akan mengambil kembali anak itu, lalu pergi."
\verse Lalu Hemor dan Sikhem, anak Hemor, menyetujui usul mereka.
\verse Dan orang muda itu tidak bertangguh melakukannya, sebab ia suka kepada anak Yakub, lagipula ia seorang yang paling dihormati di antara seluruh kaum keluarganya.
\verse Lalu pergilah Hemor dan Sikhem, anaknya itu, ke pintu gerbang kota mereka dan mereka berbicara kepada penduduk kota itu:
\verse "Orang-orang itu mau hidup damai dengan kita, biarlah mereka tinggal di negeri ini dan menjalaninya dengan bebas; bukankah negeri ini cukup luas untuk mereka? Maka kita dapat mengambil gadis-gadis mereka menjadi isteri kita dan kita dapat memberikan gadis-gadis kita kepada mereka.
\verse Namun hanya dengan syarat ini orang-orang itu setuju tinggal bersama-sama dengan kita, sehingga kita menjadi satu bangsa, yaitu setiap laki-laki di antara kita harus disunat seperti mereka bersunat.
\verse Ternak mereka, harta benda mereka dan segala hewan mereka, bukankah semuanya itu akan menjadi milik kita? Hanya biarlah kita menyetujui permintaan mereka, sehingga mereka tetap tinggal pada kita."
\verse Maka usul Hemor dan Sikhem, anaknya itu, didengarkan oleh semua orang yang datang berkumpul di pintu gerbang kota itu, lalu disunatlah setiap laki-laki, yakni setiap orang dewasa di kota itu.
\verse Pada hari ketiga, ketika mereka sedang menderita kesakitan, datanglah dua orang anak Yakub, yaitu Simeon dan Lewi, kakak-kakak Dina, setelah masing-masing mengambil pedangnya, menyerang kota itu dengan tidak takut-takut serta membunuh setiap laki-laki.
\verse Juga Hemor dan Sikhem, anaknya, dibunuh mereka dengan mata pedang, dan mereka mengambil Dina dari rumah Sikhem, lalu pergi.
\verse Kemudian datanglah anak-anak Yakub merampasi orang-orang yang terbunuh itu, lalu menjarah kota itu, karena adik mereka telah dicemari.
\verse Kambing dombanya dan lembu sapinya, keledainya dan segala yang di dalam dan di luar kota itu dibawa mereka;
\verse segala kekayaannya, semua anaknya dan perempuannya ditawan dan dijarah mereka, juga seluruhnya yang ada di rumah-rumah.
\verse Yakub berkata kepada Simeon dan Lewi: "Kamu telah mencelakakan aku dengan membusukkan namaku kepada penduduk negeri ini, kepada orang Kanaan dan orang Feris, padahal kita ini hanya sedikit jumlahnya; apabila mereka bersekutu melawan kita, tentulah mereka akan memukul kita kalah, dan kita akan dipunahkan, aku beserta seisi rumahku."
\verse Tetapi jawab mereka: "Mengapa adik kita diperlakukannya sebagai seorang perempuan sundal!"
\end{biblechapter}

\begin{biblechapter} % Kejadian 35
\verseWithHeading{Yakub di Betel untuk kedua kalinya} Allah berfirman kepada Yakub: "Bersiaplah, pergilah ke Betel, tinggallah di situ, dan buatlah di situ mezbah bagi Allah, yang telah menampakkan diri kepadamu, ketika engkau lari dari Esau, kakakmu."
\verse Lalu berkatalah Yakub kepada seisi rumahnya dan kepada semua orang yang bersama-sama dengan dia: "Jauhkanlah dewa-dewa asing yang ada di tengah-tengah kamu, tahirkanlah dirimu dan tukarlah pakaianmu.
\verse Marilah kita bersiap dan pergi ke Betel; aku akan membuat mezbah di situ bagi Allah, yang telah menjawab aku pada masa kesesakanku dan yang telah menyertai aku di jalan yang kutempuh."
\verse Mereka menyerahkan kepada Yakub segala dewa asing yang dipunyai mereka dan anting-anting yang ada pada telinga mereka, lalu Yakub menanamnya di bawah pohon besar yang dekat Sikhem.
\verse Sesudah itu berangkatlah mereka. Dan kedahsyatan yang dari Allah meliputi kota-kota sekeliling mereka, sehingga anak-anak Yakub tidak dikejar.
\verse Lalu sampailah Yakub ke Lus yang di tanah Kanaan -- yaitu Betel --, ia dan semua orang yang bersama-sama dengan dia.
\verse Didirikannyalah mezbah di situ, dan dinamainyalah tempat itu El-Betel, karena Allah telah menyatakan diri kepadanya di situ, ketika ia lari terhadap kakaknya.
\verse Ketika Debora, inang pengasuh Ribka, mati, dikuburkanlah ia di sebelah hilir Betel di bawah pohon besar, yang dinamai orang: Pohon Besar Penangisan.
\verse Setelah Yakub datang dari Padan-Aram, maka Allah menampakkan diri pula kepadanya dan memberkati dia.
\verse Firman Allah kepadanya: "Namamu Yakub; dari sekarang namamu bukan lagi Yakub, melainkan Israel, itulah yang akan menjadi namamu." Maka Allah menamai dia Israel.
\verse Lagi firman Allah kepadanya: "Akulah Allah Yang Mahakuasa. Beranakcuculah dan bertambah banyak; satu bangsa, bahkan sekumpulan bangsa-bangsa, akan terjadi dari padamu dan raja-raja akan berasal dari padamu.
\verse Dan negeri ini yang telah Kuberikan kepada Abraham dan kepada Ishak, akan Kuberikan kepadamu dan juga kepada keturunanmu."
\verse Lalu naiklah Allah meninggalkan Yakub dari tempat Ia berfirman kepadanya.
\verse Kemudian Yakub mendirikan tugu di tempat itu, yakni tugu batu; ia mempersembahkan korban curahan dan menuangkan minyak di atasnya.
\verse Yakub menamai tempat di mana Allah telah berfirman kepadanya "Betel".
\verseWithSubheading{Kelahiran Benyamin Rahel mati} Sesudah itu berangkatlah mereka dari Betel. Ketika mereka tidak berapa jauh lagi dari Efrata, bersalinlah Rahel, dan bersalinnya itu sangat sukar.
\verse Sedang ia sangat sukar bersalin, berkatalah bidan kepadanya: "Janganlah takut, sekali ini pun anak laki-laki yang kaudapat."
\verse Dan ketika ia hendak menghembuskan nafas -- sebab ia mati kemudian -- diberikannyalah nama Ben-oni kepada anak itu, tetapi ayahnya menamainya Benyamin.
\verse Demikianlah Rahel mati, lalu ia dikuburkan di sisi jalan ke Efrata, yaitu Betlehem.
\verse Yakub mendirikan tugu di atas kuburnya; itulah tugu kubur Rahel sampai sekarang.
\verse Sesudah itu berangkatlah Israel, lalu ia memasang kemahnya di seberang Migdal-Eder. 22a Ketika Israel diam di negeri ini, terjadilah bahwa Ruben sampai tidur dengan Bilha, gundik ayahnya, dan kedengaranlah hal itu kepada Israel. | 22b Adapun anak-anak lelaki Yakub dua belas orang jumlahnya. 23 Anak-anak Lea ialah Ruben, anak sulung Yakub, kemudian Simeon, Lewi, Yehuda, Isakhar dan Zebulon. 24 Anak-anak Rahel ialah Yusuf dan Benyamin. 25 Dan anak-anak Bilha, budak perempuan Rahel ialah Dan serta Naftali. 26 Dan anak-anak Zilpa, budak perempuan Lea ialah Gad dan Asyer. Itulah anak-anak lelaki Yakub, yang dilahirkan baginya di Padan-Aram. 27 Lalu sampailah Yakub kepada Ishak, ayahnya, di Mamre dekat Kiryat-Arba -- itulah Hebron -- tempat Abraham dan Ishak tinggal sebagai orang asing. 28 Adapun umur Ishak seratus delapan puluh tahun. 29 Lalu meninggallah Ishak, ia mati dan dikumpulkan kepada kaum leluhurnya; ia tua dan suntuk umur, maka Esau dan Yakub, anak-anaknya itu, menguburkan dia.
\end{biblechapter}

\begin{biblechapter} % Kejadian 36
1 Inilah keturunan Esau, yaitu Edom. 2 Esau mengambil perempuan-perempuan Kanaan menjadi isterinya, yakni Ada, anak Elon orang Het, dan Oholibama, anak Ana anak Zibeon orang Hewi, 3 dan Basmat, anak Ismael, adik Nebayot. 4 Ada melahirkan Elifas bagi Esau, dan Basmat melahirkan Rehuel, 5 dan Oholibama melahirkan Yeush, Yaelam dan Korah. Itulah anak-anak Esau, yang lahir baginya di tanah Kanaan. 6 Esau membawa isteri-isterinya, anak-anaknya lelaki dan perempuan dan semua orang yang ada di rumahnya, ternaknya, segala hewannya dan segala harta bendanya yang telah diperolehnya di tanah Kanaan, lalu pergilah ia ke negeri lain dan ia meninggalkan Yakub, adiknya itu. 7 Sebab harta milik mereka terlalu banyak, sehingga mereka tidak dapat tinggal bersama-sama, dan negeri penumpangan mereka tidak dapat memuat mereka karena banyaknya ternak mereka itu. 8 Maka menetaplah Esau di pegunungan Seir; Esau itulah Edom. 9 Inilah keturunan Esau, bapa orang Edom, di pegunungan Seir. 10 Nama anak-anaknya ialah: Elifas, anak Ada isteri Esau; Rehuel, anak Basmat isteri Esau. 11 Anak-anak Elifas ialah Téman, Omar, Zefo, Gaetam dan Kenas. 12 Timna adalah gundik Elifas anak Esau; ia melahirkan Amalek bagi Elifas. Itulah cucu-cucu Ada isteri Esau. 13 Inilah anak-anak Rehuel: Nahat, Zerah, Syama dan Miza. Itulah cucu-cucu Basmat isteri Esau. 14 Inilah anak-anak Oholibama, isteri Esau itu, anak Ana anak Zibeon; ia melahirkan bagi Esau: Yeush, Yaelam dan Korah. 15 Inilah kepala-kepala kaum bani Esau: keturunan Elifas anak sulung Esau, ialah kepala kaum Téman, kepala kaum Omar, kepala kaum Zefo, kepala kaum Kenas, 16 kepala kaum Korah, kepala kaum Gaetam dan kepala kaum Amalek; itulah kepala-kepala kaum Elifas di tanah Edom; itulah keturunan Ada. 17 Inilah keturunan Rehuel anak Esau: kepala kaum Nahat, kepala kaum Zerah, kepala kaum Syama dan kepala kaum Miza; itulah kepala-kepala kaum Rehuel di tanah Edom; itulah keturunan Basmat isteri Esau. 18 Inilah keturunan Oholibama isteri Esau: kepala kaum Yeush, kepala kaum Yaelam, kepala kaum Korah; itulah kepala-kepala kaum Oholibama, isteri Esau, anak Ana. 19 Itulah bani Esau, yakni Edom, dan itulah kepala-kepala kaum mereka. | 20 Inilah anak-anak Seir, orang Hori, penduduk negeri itu: Lotan, Syobal, Zibeon, Ana, 21 Disyon, Ezer, Disyan; itulah kepala-kepala kaum orang Hori, anak-anak Seir, di tanah Edom. 22 Anak-anak Lotan ialah Hori dan Heman, dan saudara perempuan Lotan ialah Timna. 23 Inilah anak-anak Syobal: Alwan, Manahat, Ebal, Syefo dan Onam. 24 Inilah anak-anak Zibeon: Aya dan Ana; Ana inilah yang menemui mata-mata air panas di padang gurun, ketika ia sedang menggembalakan keledai Zibeon ayahnya itu. 25 Inilah anak-anak Ana: Disyon dan Oholibama anak perempuan Ana. 26 Inilah anak-anak Disyon: Hemdan, Esyban, Yitran dan Keran. 27 Inilah anak-anak Ezer: Bilhan, Zaawan dan Akan. 28 Inilah anak-anak Disyan: Us dan Aran. 29 Itulah kepala-kepala kaum orang Hori: kepala kaum Lotan, kepala kaum Syobal, kepala kaum Zibeon, kepala kaum Ana, 30 kepala kaum Disyon, kepala kaum Ezer dan kepala kaum Disyan; itulah kepala-kepala kaum orang Hori, kaum demi kaum, di tanah Seir. | 31 Inilah raja-raja yang memerintah di tanah Edom, sebelum ada seorang raja memerintah atas orang Israel. 32 Di Edom yang memerintah ialah Bela bin Beor dan kotanya bernama Dinhaba. 33 Setelah Bela mati, Yobab bin Zerah dari Bozra menjadi raja menggantikan dia. 34 Setelah Yobab mati, Husyam, dari negeri orang Téman, menjadi raja menggantikan dia. 35 Setelah Husyam mati, Hadad bin Bedad menjadi raja menggantikan dia; dialah yang memukul kalah orang Midian di daerah Moab, dan kotanya bernama Awit. 36 Setelah Hadad mati, Samla dari Masreka menjadi raja menggantikan dia. 37 Setelah Samla mati, Saul, dari Rehobot yang di pinggir sungai, menjadi raja menggantikan dia. 38 Setelah Saul mati, Baal-Hanan bin Akhbor menjadi raja menggantikan dia. 39 Setelah Baal-Hanan bin Akhbor mati, Hadar menjadi raja menggantikan dia; kotanya bernama Pahu dan isterinya bernama Mehetabeel binti Matred binti Mezahab. 40 Inilah nama kepala-kepala kaum Esau menurut kaum dan tempat mereka, dengan nama mereka masing-masing: kepala kaum Timna, kepala kaum Alwa, kepala kaum Yetet, 41 kepala kaum Oholibama, kepala kaum Ela, kepala kaum Pinon, 42 kepala kaum Kenas, kepala kaum Téman, kepala kaum Mibzar, 43 kepala kaum Magdiel dan kepala kaum Iram; itulah kepala-kepala kaum Edom, menurut tempat kediaman mereka di tanah milik mereka; Edom ialah Esau, bapa orang Edom.
\end{biblechapter}

\begin{biblechapter} % Kejadian 37
1 Adapun Yakub, ia diam di negeri penumpangan ayahnya, yakni di tanah Kanaan. 2 Inilah riwayat keturunan Yakub. Yusuf, tatkala berumur tujuh belas tahun -- jadi masih muda -- biasa menggembalakan kambing domba, bersama-sama dengan saudara-saudaranya, anak-anak Bilha dan Zilpa, kedua isteri ayahnya. Dan Yusuf menyampaikan kepada ayahnya kabar tentang kejahatan saudara-saudaranya. 3 Israel lebih mengasihi Yusuf dari semua anaknya yang lain, sebab Yusuf itulah anaknya yang lahir pada masa tuanya; dan ia menyuruh membuat jubah yang maha indah bagi dia. 4 Setelah dilihat oleh saudara-saudaranya, bahwa ayahnya lebih mengasihi Yusuf dari semua saudaranya, maka bencilah mereka itu kepadanya dan tidak mau menyapanya dengan ramah. 5 Pada suatu kali bermimpilah Yusuf, lalu mimpinya itu diceritakannya kepada saudara-saudaranya; sebab itulah mereka lebih benci lagi kepadanya. 6 Karena katanya kepada mereka: "Coba dengarkan mimpi yang kumimpikan ini: 7 Tampak kita sedang di ladang mengikat berkas-berkas gandum, lalu bangkitlah berkasku dan tegak berdiri; kemudian datanglah berkas-berkas kamu sekalian mengelilingi dan sujud menyembah kepada berkasku itu." 8 Lalu saudara-saudaranya berkata kepadanya: "Apakah engkau ingin menjadi raja atas kami? Apakah engkau ingin berkuasa atas kami?" Jadi makin bencilah mereka kepadanya karena mimpinya dan karena perkataannya itu. 9 Lalu ia memimpikan pula mimpi yang lain, yang diceritakannya kepada saudara-saudaranya. Katanya: "Aku bermimpi pula: Tampak matahari, bulan dan sebelas bintang sujud menyembah kepadaku." 10 Setelah hal ini diceritakannya kepada ayah dan saudara-saudaranya, maka ia ditegor oleh ayahnya: "Mimpi apa mimpimu itu? Masakan aku dan ibumu serta saudara-saudaramu sujud menyembah kepadamu sampai ke tanah?" 11 Maka iri hatilah saudara-saudaranya kepadanya, tetapi ayahnya menyimpan hal itu dalam hatinya. | 12 Pada suatu kali pergilah saudara-saudaranya menggembalakan kambing domba ayahnya dekat Sikhem. 13 Lalu Israel berkata kepada Yusuf: "Bukankah saudara-saudaramu menggembalakan kambing domba dekat Sikhem? Marilah engkau kusuruh kepada mereka." Sahut Yusuf: "Ya bapa." 14 Kata Israel kepadanya: "Pergilah engkau melihat apakah baik keadaan saudara-saudaramu dan keadaan kambing domba; dan bawalah kabar tentang itu kepadaku." Lalu Yakub menyuruh dia dari lembah Hebron, dan Yusuf pun sampailah ke Sikhem. 15 Ketika Yusuf berjalan ke sana ke mari di padang, bertemulah ia dengan seorang laki-laki, yang bertanya kepadanya: "Apakah yang kaucari?" 16 Sahutnya: "Aku mencari saudara-saudaraku. Tolonglah katakan kepadaku di mana mereka menggembalakan kambing domba?" 17 Lalu kata orang itu: "Mereka telah berangkat dari sini, sebab telah kudengar mereka berkata: Marilah kita pergi ke Dotan." Maka Yusuf menyusul saudara-saudaranya itu dan didapatinyalah mereka di Dotan. 18 Dari jauh ia telah kelihatan kepada mereka. Tetapi sebelum ia dekat pada mereka, mereka telah bermufakat mencari daya upaya untuk membunuhnya. 19 Kata mereka seorang kepada yang lain: "Lihat, tukang mimpi kita itu datang! 20 Sekarang, marilah kita bunuh dia dan kita lemparkan ke dalam salah satu sumur ini, lalu kita katakan: seekor binatang buas telah menerkamnya. Dan kita akan lihat nanti, bagaimana jadinya mimpinya itu!" 21 Ketika Ruben mendengar hal ini, ia ingin melepaskan Yusuf dari tangan mereka, sebab itu katanya: "Janganlah kita bunuh dia!" 22 Lagi kata Ruben kepada mereka: "Janganlah tumpahkan darah, lemparkanlah dia ke dalam sumur yang ada di padang gurun ini, tetapi janganlah apa-apakan dia" -- maksudnya hendak melepaskan Yusuf dari tangan mereka dan membawanya kembali kepada ayahnya. 23 Baru saja Yusuf sampai kepada saudara-saudaranya, mereka pun menanggalkan jubah Yusuf, jubah maha indah yang dipakainya itu. 24 Dan mereka membawa dia dan melemparkan dia ke dalam sumur. Sumur itu kosong, tidak berair. 25 Kemudian duduklah mereka untuk makan. Ketika mereka mengangkat muka, kelihatanlah kepada mereka suatu kafilah orang Ismael datang dari Gilead dengan untanya yang membawa damar, balsam dan damar ladan, dalam perjalanannya mengangkut barang-barang itu ke Mesir. 26 Lalu kata Yehuda kepada saudara-saudaranya itu: "Apakah untungnya kalau kita membunuh adik kita itu dan menyembunyikan darahnya? 27 Marilah kita jual dia kepada orang Ismael ini, tetapi janganlah kita apa-apakan dia, karena ia saudara kita, darah daging kita." Dan saudara-saudaranya mendengarkan perkataannya itu. 28 Ketika ada saudagar-saudagar Midian lewat, Yusuf diangkat ke atas dari dalam sumur itu, kemudian dijual kepada orang Ismael itu dengan harga dua puluh syikal perak. Lalu Yusuf dibawa mereka ke Mesir. 29 Ketika Ruben kembali ke sumur itu, ternyata Yusuf tidak ada lagi di dalamnya. Lalu dikoyakkannyalah bajunya, 30 dan kembalilah ia kepada saudara-saudaranya, katanya: "Anak itu tidak ada lagi, ke manakah aku ini?" 31 Kemudian mereka mengambil jubah Yusuf, dan menyembelih seekor kambing, lalu mencelupkan jubah itu ke dalam darahnya. 32 Jubah maha indah itu mereka suruh antarkan kepada ayah mereka dengan pesan: "Ini kami dapati. Silakanlah bapa periksa apakah jubah ini milik anak bapa atau tidak?" 33 Ketika Yakub memeriksa jubah itu, ia berkata: "Ini jubah anakku; binatang buas telah memakannya; tentulah Yusuf telah diterkam." 34 Dan Yakub mengoyakkan jubahnya, lalu mengenakan kain kabung pada pinggangnya dan berkabunglah ia berhari-hari lamanya karena anaknya itu. 35 Sekalian anaknya laki-laki dan perempuan berusaha menghiburkan dia, tetapi ia menolak dihiburkan, serta katanya: "Tidak! Aku akan berkabung, sampai aku turun mendapatkan anakku, ke dalam dunia orang mati!" Demikianlah Yusuf ditangisi oleh ayahnya. 36 Adapun Yusuf, ia dijual oleh orang Midian itu ke Mesir, kepada Potifar, seorang pegawai istana Firaun, kepala pengawal raja.
\end{biblechapter}

\begin{biblechapter} % Kejadian 38
1 Pada waktu itu Yehuda meninggalkan saudara-saudaranya dan menumpang pada seorang Adulam, yang namanya Hira. 2 Di situ Yehuda melihat anak perempuan seorang Kanaan; nama orang itu ialah Syua. Lalu Yehuda kawin dengan perempuan itu dan menghampirinya. 3 Perempuan itu mengandung, lalu melahirkan seorang anak laki-laki dan menamai anak itu Er. 4 Sesudah itu perempuan itu mengandung lagi, lalu melahirkan seorang anak laki-laki dan menamai anak itu Onan. 5 Kemudian perempuan itu melahirkan seorang anak laki-laki sekali lagi, dan menamai anak itu Syela. Yehuda sedang berada di Kezib, ketika anak itu dilahirkan. 6 Sesudah itu Yehuda mengambil bagi Er, anak sulungnya, seorang isteri, yang bernama Tamar. 7 Tetapi Er, anak sulung Yehuda itu, adalah jahat di mata TUHAN, maka TUHAN membunuh dia. 8 Lalu berkatalah Yehuda kepada Onan: "Hampirilah isteri kakakmu itu, kawinlah dengan dia sebagai ganti kakakmu dan bangkitkanlah keturunan bagi kakakmu." 9 Tetapi Onan tahu, bahwa bukan ia yang empunya keturunannya nanti, sebab itu setiap kali ia menghampiri isteri kakaknya itu, ia membiarkan maninya terbuang, supaya ia jangan memberi keturunan kepada kakaknya. 10 Tetapi yang dilakukannya itu adalah jahat di mata TUHAN, maka TUHAN membunuh dia juga. 11 Lalu berkatalah Yehuda kepada Tamar, menantunya itu: "Tinggallah sebagai janda di rumah ayahmu, sampai anakku Syela itu besar," sebab pikirnya: "Jangan-jangan ia mati seperti kedua kakaknya itu." Maka pergilah Tamar dan tinggal di rumah ayahnya. 12 Setelah beberapa lama matilah anak Syua, isteri Yehuda. Habis berkabung pergilah Yehuda ke Timna, kepada orang-orang yang menggunting bulu domba-dombanya, bersama dengan Hira, sahabatnya, orang Adulam itu. 13 Ketika dikabarkan kepada Tamar: "Bapa mertuamu sedang di jalan ke Timna untuk menggunting bulu domba-dombanya," 14 maka ditanggalkannyalah pakaian kejandaannya, ia bertelekung dan berselubung, lalu pergi duduk di pintu masuk ke Enaim yang di jalan ke Timna, karena dilihatnya, bahwa Syela telah menjadi besar, dan dia tidak diberikan juga kepada Syela itu untuk menjadi isterinya. 15 Ketika Yehuda melihat dia, disangkanyalah dia seorang perempuan sundal, karena ia menutupi mukanya. 16 Lalu berpalinglah Yehuda mendapatkan perempuan yang di pinggir jalan itu serta berkata: "Marilah, aku mau menghampiri engkau," sebab ia tidak tahu, bahwa perempuan itu menantunya. Tanya perempuan itu: "Apakah yang akan kauberikan kepadaku, jika engkau menghampiri aku?" 17 Jawabnya: "Aku akan mengirimkan kepadamu seekor anak kambing dari kambing dombaku." Kata perempuan itu: "Asal engkau memberikan tanggungannya, sampai engkau mengirimkannya kepadaku." 18 Tanyanya: "Apakah tanggungan yang harus kuberikan kepadamu?" Jawab perempuan itu: "Cap meteraimu serta kalungmu dan tongkat yang ada di tanganmu itu." Lalu diberikannyalah semuanya itu kepadanya, maka ia menghampirinya. Perempuan itu mengandung dari padanya. 19 Bangunlah perempuan itu, lalu pergi, ditanggalkannya telekungnya dan dikenakannya pula pakaian kejandaannya. 20 Adapun Yehuda, ia mengirimkan anak kambing itu dengan perantaraan sahabatnya, orang Adulam itu, untuk mengambil kembali tanggungannya dari tangan perempuan itu, tetapi perempuan itu tidak dijumpainya lagi. 21 Ia bertanya-tanya di tempat tinggal perempuan itu: "Di manakah perempuan jalang, yang duduk tadinya di pinggir jalan di Enaim itu?" Jawab mereka: "Tidak ada di sini perempuan jalang." 22 Kembalilah ia kepada Yehuda dan berkata: "Tidak ada kujumpai dia; dan juga orang-orang di tempat itu berkata: Tidak ada perempuan jalang di sini." 23 Lalu berkatalah Yehuda: "Biarlah barang-barang itu dipegangnya, supaya kita jangan menjadi buah olok-olok orang; sungguhlah anak kambing itu telah kukirimkan, tetapi engkau tidak menjumpai perempuan itu." 24 Sesudah kira-kira tiga bulan dikabarkanlah kepada Yehuda: "Tamar, menantumu, bersundal, bahkan telah mengandung dari persundalannya itu." Lalu kata Yehuda: "Bawalah perempuan itu, supaya dibakar." 25 Waktu dibawa, perempuan itu menyuruh orang kepada mertuanya mengatakan: "Dari laki-laki yang empunya barang-barang inilah aku mengandung." Juga dikatakannya: "Periksalah, siapa yang empunya cap meterai serta kalung dan tongkat ini?" 26 Yehuda memeriksa barang-barang itu, lalu berkata: "Bukan aku, tetapi perempuan itulah yang benar, karena memang aku tidak memberikan dia kepada Syela, anakku." Dan ia tidak bersetubuh lagi dengan perempuan itu. 27 Pada waktu perempuan itu hendak bersalin, nyatalah ada anak kembar dalam kandungannya. 28 Dan ketika ia bersalin, seorang dari anak itu mengeluarkan tangannya, lalu dipegang oleh bidan, diikatnya dengan benang kirmizi serta berkata: "Inilah yang lebih dahulu keluar." 29 Ketika anak itu menarik tangannya kembali, keluarlah saudaranya laki-laki, dan bidan itu berkata: "Alangkah kuatnya engkau menembus ke luar," maka anak itu dinamai Peres. 30 Sesudah itu keluarlah saudaranya laki-laki yang tangannya telah berikat benang kirmizi itu, lalu kepadanya diberi nama Zerah.
\end{biblechapter}

\begin{biblechapter} % Kejadian 39
1 Adapun Yusuf telah dibawa ke Mesir; dan Potifar, seorang Mesir, pegawai istana Firaun, kepala pengawal raja, membeli dia dari tangan orang Ismael yang telah membawa dia ke situ. 2 Tetapi TUHAN menyertai Yusuf, sehingga ia menjadi seorang yang selalu berhasil dalam pekerjaannya; maka tinggallah ia di rumah tuannya, orang Mesir itu. 3 Setelah dilihat oleh tuannya, bahwa Yusuf disertai TUHAN dan bahwa TUHAN membuat berhasil segala sesuatu yang dikerjakannya, 4 maka Yusuf mendapat kasih tuannya, dan ia boleh melayani dia; kepada Yusuf diberikannya kuasa atas rumahnya dan segala miliknya diserahkannya pada kekuasaan Yusuf. 5 Sejak ia memberikan kuasa dalam rumahnya dan atas segala miliknya kepada Yusuf, TUHAN memberkati rumah orang Mesir itu karena Yusuf, sehingga berkat TUHAN ada atas segala miliknya, baik yang di rumah maupun yang di ladang. 6 Segala miliknya diserahkannya pada kekuasaan Yusuf, dan dengan bantuan Yusuf ia tidak usah lagi mengatur apa-apa pun selain dari makanannya sendiri. Adapun Yusuf itu manis sikapnya dan elok parasnya. 7 Selang beberapa waktu isteri tuannya memandang Yusuf dengan berahi, lalu katanya: "Marilah tidur dengan aku." 8 Tetapi Yusuf menolak dan berkata kepada isteri tuannya itu: "Dengan bantuanku tuanku itu tidak lagi mengatur apa yang ada di rumah ini dan ia telah menyerahkan segala miliknya pada kekuasaanku, 9 bahkan di rumah ini ia tidak lebih besar kuasanya dari padaku, dan tiada yang tidak diserahkannya kepadaku selain dari pada engkau, sebab engkau isterinya. Bagaimanakah mungkin aku melakukan kejahatan yang besar ini dan berbuat dosa terhadap Allah?" 10 Walaupun dari hari ke hari perempuan itu membujuk Yusuf, Yusuf tidak mendengarkan bujukannya itu untuk tidur di sisinya dan bersetubuh dengan dia. 11 Pada suatu hari masuklah Yusuf ke dalam rumah untuk melakukan pekerjaannya, sedang dari seisi rumah itu seorang pun tidak ada di rumah. 12 Lalu perempuan itu memegang baju Yusuf sambil berkata: "Marilah tidur dengan aku." Tetapi Yusuf meninggalkan bajunya di tangan perempuan itu dan lari ke luar. 13 Ketika dilihat perempuan itu, bahwa Yusuf meninggalkan bajunya dalam tangannya dan telah lari ke luar, 14 dipanggilnyalah seisi rumah itu, lalu katanya kepada mereka: "Lihat, dibawanya ke mari seorang Ibrani, supaya orang ini dapat mempermainkan kita. Orang ini mendekati aku untuk tidur dengan aku, tetapi aku berteriak-teriak dengan suara keras. 15 Dan ketika didengarnya bahwa aku berteriak sekeras-kerasnya, ditinggalkannyalah bajunya padaku, lalu ia lari ke luar." 16 Juga ditaruhnya baju Yusuf itu di sisinya, sampai tuan rumah pulang. 17 Perkataan itu jugalah yang diceritakan perempuan itu kepada Potifar, katanya: "Hamba orang Ibrani yang kaubawa ke mari itu datang kepadaku untuk mempermainkan aku. 18 Tetapi ketika aku berteriak sekeras-kerasnya, ditinggalkannya bajunya padaku, lalu ia lari ke luar." 19 Baru saja didengar oleh tuannya perkataan yang diceritakan isterinya kepadanya: begini begitulah aku diperlakukan oleh hambamu itu, maka bangkitlah amarahnya. 20 Lalu Yusuf ditangkap oleh tuannya dan dimasukkan ke dalam penjara, tempat tahanan-tahanan raja dikurung. Demikianlah Yusuf dipenjarakan di sana. 21 Tetapi TUHAN menyertai Yusuf dan melimpahkan kasih setia-Nya kepadanya, dan membuat Yusuf kesayangan bagi kepala penjara itu. 22 Sebab itu kepala penjara mempercayakan semua tahanan dalam penjara itu kepada Yusuf, dan segala pekerjaan yang harus dilakukan di situ, dialah yang mengurusnya. 23 Dan kepala penjara tidak mencampuri segala yang dipercayakannya kepada Yusuf, karena TUHAN menyertai dia dan apa yang dikerjakannya dibuat TUHAN berhasil.
\end{biblechapter}

\begin{biblechapter} % Kejadian 40
1 Sesudah semuanya itu terjadilah, bahwa juru minuman raja Mesir dan juru rotinya membuat kesalahan terhadap tuannya, raja Mesir itu, 2 maka murkalah Firaun kepada kedua pegawai istananya, kepala juru minuman dan kepala juru roti itu. 3 Ia menahan mereka dalam rumah kepala pengawal raja, dalam penjara tempat Yusuf dikurung. 4 Kepala pengawal raja menempatkan Yusuf bersama-sama dengan mereka untuk melayani mereka. Demikianlah mereka ditahan beberapa waktu lamanya. 5 Pada suatu kali bermimpilah mereka keduanya -- baik juru minuman maupun juru roti raja Mesir, yang ditahan dalam penjara itu -- masing-masing ada mimpinya, pada satu malam juga, dan mimpi masing-masing itu ada artinya sendiri. 6 Ketika pada waktu pagi Yusuf datang kepada mereka, segera dilihatnya, bahwa mereka bersusah hati. 7 Lalu ia bertanya kepada pegawai-pegawai istana Firaun yang ditahan bersama-sama dengan dia dalam rumah tuannya itu: "Mengapakah hari ini mukamu semuram itu?" 8 Jawab mereka kepadanya: "Kami bermimpi, tetapi tidak ada orang yang dapat mengartikannya." Lalu kata Yusuf kepada mereka: "Bukankah Allah yang menerangkan arti mimpi? Ceritakanlah kiranya mimpimu itu kepadaku." 9 Kemudian juru minuman itu menceritakan mimpinya kepada Yusuf, katanya: "Dalam mimpiku itu tampak ada pohon anggur di depanku. 10 Pohon anggur itu ada tiga carangnya dan baru saja pohon itu bertunas, bunganya sudah keluar dan tandan-tandannya penuh buah anggur yang ranum. 11 Dan di tanganku ada piala Firaun. Buah anggur itu kuambil, lalu kuperas ke dalam piala Firaun, kemudian kusampaikan piala itu ke tangan Firaun." 12 Kata Yusuf kepadanya: "Beginilah arti mimpi itu: ketiga carang itu artinya tiga hari; 13 dalam tiga hari ini Firaun akan meninggikan engkau dan mengembalikan engkau ke dalam pangkatmu yang dahulu dan engkau akan menyampaikan piala ke tangan Firaun seperti dahulu kala, ketika engkau jadi juru minumannya. 14 Tetapi, ingatlah kepadaku, apabila keadaanmu telah baik nanti, tunjukkanlah terima kasihmu kepadaku dengan menceritakan hal ihwalku kepada Firaun dan tolonglah keluarkan aku dari rumah ini. 15 Sebab aku dicuri diculik begitu saja dari negeri orang Ibrani dan di sini pun aku tidak pernah melakukan apa-apa yang menyebabkan aku layak dimasukkan ke dalam liang tutupan ini." 16 Setelah dilihat oleh kepala juru roti, betapa baik arti mimpi itu, berkatalah ia kepadanya: "Aku pun bermimpi juga. Tampak aku menjunjung tiga bakul berisi penganan. 17 Dalam bakul atas ada berbagai-bagai makanan untuk Firaun, buatan juru roti, tetapi burung-burung memakannya dari dalam bakul yang di atas kepalaku." 18 Yusuf menjawab: "Beginilah arti mimpi itu: ketiga bakul itu artinya tiga hari; 19 dalam tiga hari ini Firaun akan meninggikan engkau, tinggi ke atas, dan menggantung engkau pada sebuah tiang, dan burung-burung akan memakan dagingmu dari tubuhmu." 20 Dan terjadilah pada hari ketiga, hari kelahiran Firaun, maka Firaun mengadakan perjamuan untuk semua pegawainya. Ia meninggikan kepala juru minuman dan kepala juru roti itu di tengah-tengah para pegawainya: 21 kepala juru minuman itu dikembalikannya ke dalam jabatannya, sehingga ia menyampaikan pula piala ke tangan Firaun; 22 tetapi kepala juru roti itu digantungnya, seperti yang ditakbirkan Yusuf kepada mereka. 23 Tetapi Yusuf tidaklah diingat oleh kepala juru minuman itu, melainkan dilupakannya.
\end{biblechapter}

\begin{biblechapter} % Kejadian 41
1 Setelah lewat dua tahun lamanya, bermimpilah Firaun, bahwa ia berdiri di tepi sungai Nil. 2 Tampaklah dari sungai Nil itu keluar tujuh ekor lembu yang indah bangunnya dan gemuk badannya; lalu memakan rumput yang di tepi sungai itu. 3 Kemudian tampaklah juga tujuh ekor lembu yang lain, yang keluar dari dalam sungai Nil itu, buruk bangunnya dan kurus badannya, lalu berdiri di samping lembu-lembu yang tadi, di tepi sungai itu. 4 Lembu-lembu yang buruk bangunnya dan kurus badannya itu memakan ketujuh ekor lembu yang indah bangunnya dan gemuk itu. Lalu terjagalah Firaun. 5 Setelah itu tertidur pulalah ia dan bermimpi kedua kalinya: Tampak timbul dari satu tangkai tujuh bulir gandum yang bernas dan baik. 6 Tetapi kemudian tampaklah juga tumbuh tujuh bulir gandum yang kurus dan layu oleh angin timur. 7 Bulir yang kurus itu menelan ketujuh bulir yang bernas dan berisi tadi. Lalu terjagalah Firaun. Agaknya ia bermimpi! 8 Pada waktu pagi gelisahlah hatinya, lalu disuruhnyalah memanggil semua ahli dan semua orang berilmu di Mesir. Firaun menceritakan mimpinya kepada mereka, tetapi seorang pun tidak ada yang dapat mengartikannya kepadanya. 9 Lalu berkatalah kepala juru minuman kepada Firaun: "Hari ini aku merasa perlu menyebutkan kesalahanku yang dahulu. 10 Waktu itu tuanku Firaun murka kepada pegawai-pegawainya, dan menahan aku dalam rumah pengawal istana, beserta dengan kepala juru roti. 11 Pada satu malam juga kami bermimpi, aku dan kepala juru roti itu; masing-masing mempunyai mimpi dengan artinya sendiri. 12 Bersama-sama dengan kami ada di sana seorang muda Ibrani, hamba kepala pengawal istana itu; kami menceritakan mimpi kami kepadanya, lalu diartikannya kepada kami mimpi kami masing-masing. 13 Dan seperti yang diartikannya itu kepada kami, demikianlah pula terjadi: aku dikembalikan ke dalam pangkatku, dan kepala juru roti itu digantung." 14 Kemudian Firaun menyuruh memanggil Yusuf. Segeralah ia dikeluarkan dari tutupan; ia bercukur dan berganti pakaian, lalu pergi menghadap Firaun. 15 Berkatalah Firaun kepada Yusuf: "Aku telah bermimpi, dan seorang pun tidak ada yang dapat mengartikannya, tetapi telah kudengar tentang engkau: hanya dengan mendengar mimpi saja engkau dapat mengartikannya." 16 Yusuf menyahut Firaun: "Bukan sekali-kali aku, melainkan Allah juga yang akan memberitakan kesejahteraan kepada tuanku Firaun." 17 Lalu berkatalah Firaun kepada Yusuf: "Dalam mimpiku itu, aku berdiri di tepi sungai Nil; 18 lalu tampaklah dari sungai Nil itu keluar tujuh ekor lembu yang gemuk badannya dan indah bentuknya, dan makan rumput yang di tepi sungai itu. 19 Tetapi kemudian tampaklah juga keluar tujuh ekor lembu yang lain, kulit pemalut tulang, sangat buruk bangunnya dan kurus badannya; tidak pernah kulihat yang seburuk itu di seluruh tanah Mesir. 20 Lembu yang kurus dan buruk itu memakan ketujuh ekor lembu gemuk yang mula-mula. 21 Lembu-lembu ini masuk ke dalam perutnya, tetapi walaupun telah masuk ke dalam perutnya, tidaklah kelihatan sedikit pun tandanya: bangunnya tetap sama buruknya seperti semula. Lalu terjagalah aku. 22 Selanjutnya dalam mimpiku itu kulihat timbul dari satu tangkai tujuh bulir gandum yang berisi dan baik. 23 Tetapi kemudian tampaklah juga tumbuh tujuh bulir yang kering, kurus dan layu oleh angin timur. 24 Bulir yang kurus itu memakan ketujuh bulir yang baik tadi. Telah kuceritakan hal ini kepada semua ahli, tetapi seorang pun tidak ada yang dapat menerangkannya kepadaku." 25 Lalu kata Yusuf kepada Firaun: "Kedua mimpi tuanku Firaun itu sama. Allah telah memberitahukan kepada tuanku Firaun apa yang hendak dilakukan-Nya. 26 Ketujuh ekor lembu yang baik itu ialah tujuh tahun, dan ketujuh bulir gandum yang baik itu ialah tujuh tahun juga; kedua mimpi itu sama. 27 Ketujuh ekor lembu yang kurus dan buruk, yang keluar kemudian, maksudnya tujuh tahun, demikian pula ketujuh bulir gandum yang hampa dan layu oleh angin timur itu; maksudnya akan ada tujuh tahun kelaparan. 28 Inilah maksud perkataanku, ketika aku berkata kepada tuanku Firaun: Allah telah memperlihatkan kepada tuanku Firaun apa yang hendak dilakukan-Nya. 29 Ketahuilah tuanku, akan datang tujuh tahun kelimpahan di seluruh tanah Mesir. 30 Kemudian akan timbul tujuh tahun kelaparan; maka akan dilupakan segala kelimpahan itu di tanah Mesir, karena kelaparan itu menguruskeringkan negeri ini. 31 Sesudah itu akan tidak kelihatan lagi bekas-bekas kelimpahan di negeri ini karena kelaparan itu, sebab sangat hebatnya kelaparan itu. 32 Sampai dua kali mimpi itu diulangi bagi tuanku Firaun berarti: hal itu telah ditetapkan oleh Allah dan Allah akan segera melakukannya. 33 Oleh sebab itu, baiklah tuanku Firaun mencari seorang yang berakal budi dan bijaksana, dan mengangkatnya menjadi kuasa atas tanah Mesir. 34 Baiklah juga tuanku Firaun berbuat begini, yakni menempatkan penilik-penilik atas negeri ini dan dalam ketujuh tahun kelimpahan itu memungut seperlima dari hasil tanah Mesir. 35 Mereka harus mengumpulkan segala bahan makanan dalam tahun-tahun baik yang akan datang ini dan, di bawah kuasa tuanku Firaun, menimbun gandum di kota-kota sebagai bahan makanan, serta menyimpannya. 36 Demikianlah segala bahan makanan itu menjadi persediaan untuk negeri ini dalam ketujuh tahun kelaparan yang akan terjadi di tanah Mesir, supaya negeri ini jangan binasa karena kelaparan itu." | 37 Usul itu dipandang baik oleh Firaun dan oleh semua pegawainya. 38 Lalu berkatalah Firaun kepada para pegawainya: "Mungkinkah kita mendapat orang seperti ini, seorang yang penuh dengan Roh Allah?" 39 Kata Firaun kepada Yusuf: "Oleh karena Allah telah memberitahukan semuanya ini kepadamu, tidaklah ada orang yang demikian berakal budi dan bijaksana seperti engkau. 40 Engkaulah menjadi kuasa atas istanaku, dan kepada perintahmu seluruh rakyatku akan taat; hanya takhta inilah kelebihanku dari padamu." 41 Selanjutnya Firaun berkata kepada Yusuf: "Dengan ini aku melantik engkau menjadi kuasa atas seluruh tanah Mesir." 42 Sesudah itu Firaun menanggalkan cincin meterainya dari jarinya dan mengenakannya pada jari Yusuf; dipakaikannyalah kepada Yusuf pakaian dari pada kain halus dan digantungkannya kalung emas pada lehernya. 43 Lalu Firaun menyuruh menaikkan Yusuf dalam keretanya yang kedua, dan berserulah orang di hadapan Yusuf: "Hormat!" Demikianlah Yusuf dilantik oleh Firaun menjadi kuasa atas seluruh tanah Mesir. 44 Berkatalah Firaun kepada Yusuf: "Akulah Firaun, tetapi dengan tidak setahumu, seorang pun tidak boleh bergerak di seluruh tanah Mesir." 45 Lalu Firaun menamai Yusuf: Zafnat-Paaneah, serta memberikan Asnat, anak Potifera, imam di On, kepadanya menjadi isterinya. Demikianlah Yusuf muncul sebagai kuasa atas seluruh tanah Mesir. 46 Yusuf berumur tiga puluh tahun ketika ia menghadap Firaun, raja Mesir itu. Maka pergilah Yusuf dari depan Firaun, lalu dikelilinginya seluruh tanah Mesir. 47 Tanah itu mengeluarkan hasil bertumpuk-tumpuk dalam ketujuh tahun kelimpahan itu, 48 maka Yusuf mengumpulkan segala bahan makanan ketujuh tahun kelimpahan yang ada di tanah Mesir, lalu disimpannya di kota-kota; hasil daerah sekitar tiap-tiap kota disimpan di dalam kota itu. 49 Demikianlah Yusuf menimbun gandum seperti pasir di laut, sangat banyak, sehingga orang berhenti menghitungnya, karena memang tidak terhitung. 50 Sebelum datang tahun kelaparan itu, lahirlah bagi Yusuf dua orang anak laki-laki, yang dilahirkan oleh Asnat, anak Potifera, imam di On. 51 Yusuf memberi nama Manasye kepada anak sulungnya itu, sebab katanya: "Allah telah membuat aku lupa sama sekali kepada kesukaranku dan kepada rumah bapaku." 52 Dan kepada anaknya yang kedua diberinya nama Efraim, sebab katanya: "Allah membuat aku mendapat anak dalam negeri kesengsaraanku." 53 Setelah lewat ketujuh tahun kelimpahan yang ada di tanah Mesir itu, 54 mulailah datang tujuh tahun kelaparan, seperti yang telah dikatakan Yusuf; dalam segala negeri ada kelaparan, tetapi di seluruh negeri Mesir ada roti. 55 Ketika seluruh negeri Mesir menderita kelaparan, dan rakyat berteriak meminta roti kepada Firaun, berkatalah Firaun kepada semua orang Mesir: "Pergilah kepada Yusuf, perbuatlah apa yang akan dikatakannya kepadamu." 56 Kelaparan itu merajalela di seluruh bumi. Maka Yusuf membuka segala lumbung dan menjual gandum kepada orang Mesir, sebab makin hebat kelaparan itu di tanah Mesir. 57 Juga dari seluruh bumi datanglah orang ke Mesir untuk membeli gandum dari Yusuf, sebab hebat kelaparan itu di seluruh bumi.
\end{biblechapter}

\begin{biblechapter} % Kejadian 46
1 Jadi berangkatlah Israel dengan segala miliknya dan ia tiba di Bersyeba, lalu dipersembahkannya korban sembelihan kepada Allah Ishak ayahnya. 2 Berfirmanlah Allah kepada Israel dalam penglihatan waktu malam: "Yakub, Yakub!" Sahutnya: "Ya, Tuhan." 3 Lalu firman-Nya: "Akulah Allah, Allah ayahmu, janganlah takut pergi ke Mesir, sebab Aku akan membuat engkau menjadi bangsa yang besar di sana. 4 Aku sendiri akan menyertai engkau pergi ke Mesir dan tentulah Aku juga akan membawa engkau kembali; dan tangan Yusuflah yang akan mengatupkan kelopak matamu nanti." 5 Lalu berangkatlah Yakub dari Bersyeba, dan anak-anak Israel membawa Yakub, ayah mereka, beserta anak dan isteri mereka, dan mereka menaiki kereta yang dikirim Firaun untuk menjemputnya. 6 Mereka membawa juga ternaknya dan harta bendanya, yang telah diperoleh mereka di tanah Kanaan, lalu tibalah mereka di Mesir, yakni Yakub dan seluruh keturunannya bersama-sama dengan dia. 7 Anak-anak dan cucu-cucunya laki-laki dan perempuan, seluruh keturunannya dibawanyalah ke Mesir. 8 Inilah nama-nama bani Israel yang datang ke Mesir, yakni Yakub beserta keturunannya. Anak sulung Yakub ialah Ruben. 9 Anak-anak Ruben ialah Henokh, Palu, Hezron dan Karmi. 10 Anak-anak Simeon ialah Yemuel, Yamin, Ohad, Yakhin dan Zohar serta Saul, anak seorang perempuan Kanaan. 11 Anak-anak Lewi ialah Gerson, Kehat dan Merari. 12 Anak-anak Yehuda ialah Er, Onan, Syela, Peres dan Zerah; tetapi Er dan Onan mati di tanah Kanaan; dan anak-anak Peres ialah Hezron dan Hamul. 13 Anak-anak Isakhar ialah Tola, Pua, Ayub dan Simron. 14 Anak-anak Zebulon ialah Sered, Elon dan Yahleel. 15 Itulah keturunan Lea, yang melahirkan bagi Yakub di Padan-Aram anak-anak lelaki serta Dina juga, anaknya yang perempuan. Jadi seluruhnya, laki-laki dan perempuan, berjumlah tiga puluh tiga jiwa. 16 Anak-anak Gad ialah Zifyon, Hagi, Syuni, Ezbon, Eri, Arodi dan Areli. 17 Anak-anak Asyer ialah Yimna, Yiswa, Yiswi dan Beria; Serah ialah saudara perempuan mereka; dan anak-anak Beria ialah Heber dan Malkiel. 18 Itulah keturunan Zilpa, yakni hamba perempuan yang telah diberikan Laban kepada Lea, anaknya perempuan, dan yang melahirkan anak-anak bagi Yakub; seluruhnya enam belas jiwa. 19 Anak-anak Rahel, isteri Yakub, ialah Yusuf dan Benyamin. 20 Bagi Yusuf lahir Manasye dan Efraim di tanah Mesir, yang dilahirkan baginya oleh Asnat, anak perempuan Potifera, imam di On. 21 Anak-anak Benyamin ialah Bela, Bekher, Asybel, Gera, Naaman, Ehi, Rosh, Mupim, Hupim dan Ared. 22 Itulah keturunan Rahel, yang telah lahir bagi Yakub, seluruhnya berjumlah empat belas jiwa. 23 Anak Dan ialah Husim. 24 Anak-anak Naftali ialah Yahzeel, Guni, Yezer dan Syilem. 25 Itulah keturunan Bilha, yakni hamba perempuan yang diberikan Laban kepada Rahel, anaknya yang perempuan dan yang melahirkan anak-anak itu bagi Yakub -- seluruhnya berjumlah tujuh jiwa. 26 Semua orang yang tiba di Mesir bersama-sama dengan Yakub, yakni anak-anak kandungnya, dengan tidak terhitung isteri anak-anaknya, seluruhnya berjumlah enam puluh enam jiwa. 27 Anak-anak Yusuf yang lahir baginya di Mesir ada dua orang. Jadi keluarga Yakub yang tiba di Mesir, seluruhnya berjumlah tujuh puluh jiwa. 28 Yakub menyuruh Yehuda berjalan lebih dahulu mendapatkan Yusuf, supaya Yusuf datang ke Gosyen menemui ayahnya. Sementara itu sampailah mereka ke tanah Gosyen. 29 Lalu Yusuf memasang keretanya dan pergi ke Gosyen, mendapatkan Israel, ayahnya. Ketika ia bertemu dengan dia, dipeluknyalah leher ayahnya dan lama menangis pada bahunya. 30 Berkatalah Israel kepada Yusuf: "Sekarang bolehlah aku mati, setelah aku melihat mukamu dan mengetahui bahwa engkau masih hidup." 31 Kemudian berkatalah Yusuf kepada saudara-saudaranya dan kepada keluarga ayahnya itu: "Aku mau menghadap Firaun dan memberitahukan kepadanya: Saudara-saudaraku dan keluarga ayahku, yang tinggal di tanah Kanaan, telah datang kepadaku; 32 orang-orang itu gembala kambing domba, sebab mereka itu pemelihara ternak, dan kambing dombanya, lembu sapinya dan segala miliknya telah dibawa mereka. 33 Apabila Firaun memanggil kamu dan bertanya: Apakah pekerjaanmu? 34 maka jawablah: Hamba-hambamu ini pemelihara ternak, sejak dari kecil sampai sekarang, baik kami maupun nenek moyang kami -- dengan maksud supaya kamu boleh diam di tanah Gosyen." -- Sebab segala gembala kambing domba adalah suatu kekejian bagi orang Mesir.
\end{biblechapter}

\begin{biblechapter} % Kejadian 50
1 Lalu Yusuf merebahkan dirinya mendekap muka ayahnya serta menangisi dan mencium dia. 2 Dan Yusuf memerintahkan kepada tabib-tabib, yaitu hamba-hambanya, untuk merempah-rempahi mayat ayahnya; maka tabib-tabib itu merempah-rempahi mayat Israel. 3 Hal itu memerlukan empat puluh hari lamanya, sebab demikianlah lamanya waktu yang diperlukan untuk merempah-rempahi, dan orang Mesir menangisi dia tujuh puluh hari lamanya. 4 Setelah lewat hari-hari penangisan itu, berkatalah Yusuf kepada seisi istana Firaun: "Jika kiranya aku mendapat kasihmu, katakanlah kepada Firaun, 5 bahwa ayahku telah menyuruh aku bersumpah, katanya: Tidak lama lagi aku akan mati; dalam kuburku yang telah kugali di tanah Kanaan, di situlah kaukuburkan aku. Oleh sebab itu, izinkanlah aku pergi ke sana, supaya aku menguburkan ayahku; kemudian aku akan kembali." 6 Lalu berkatalah Firaun: "Pergilah ke sana dan kuburkanlah ayahmu itu, seperti yang telah disuruhnya engkau bersumpah." 7 Lalu berjalanlah Yusuf ke sana untuk menguburkan ayahnya, dan bersama-sama dengan dia berjalanlah semua pegawai Firaun, para tua-tua dari istananya, dan semua tua-tua dari tanah Mesir, 8 serta seisi rumah Yusuf juga, saudara-saudaranya dan seisi rumah ayahnya; hanya anak-anaknya serta kambing domba dan lembu sapinya ditinggalkan mereka di tanah Gosyen. 9 Baik kereta maupun orang-orang berkuda turut pergi ke sana bersama-sama dengan dia, sehingga iring-iringan itu sangat besar. 10 Setelah mereka sampai ke Goren-Haatad, yang di seberang sungai Yordan, maka mereka mengadakan di situ ratapan yang sangat sedih dan riuh; dan Yusuf mengadakan perkabungan tujuh hari lamanya karena ayahnya itu. 11 Ketika penduduk negeri itu, orang-orang Kanaan, melihat perkabungan di Goren-Haatad itu, berkatalah mereka: "Inilah perkabungan orang Mesir yang amat riuh." Itulah sebabnya tempat itu dinamai Abel-Mizraim, yang letaknya di seberang Yordan. 12 Anak-anak Yakub melakukan kepadanya, seperti yang dipesankannya kepada mereka. 13 Anak-anaknya mengangkut dia ke tanah Kanaan, dan mereka menguburkan dia dalam gua di ladang Makhpela yang telah dibeli Abraham dari Efron, orang Het itu, untuk menjadi kuburan milik, yaitu ladang yang di sebelah timur Mamre. 14 Setelah ayahnya dikuburkan, pulanglah Yusuf ke Mesir, dia dan saudara-saudaranya dan semua orang yang turut pergi ke sana bersama-sama dengan dia untuk menguburkan ayahnya itu. | 15 Ketika saudara-saudara Yusuf melihat, bahwa ayah mereka telah mati, berkatalah mereka: "Boleh jadi Yusuf akan mendendam kita dan membalaskan sepenuhnya kepada kita segala kejahatan yang telah kita lakukan kepadanya." 16 Sebab itu mereka menyuruh menyampaikan pesan ini kepada Yusuf: "Sebelum ayahmu mati, ia telah berpesan: 17 Beginilah harus kamu katakan kepada Yusuf: Ampunilah kiranya kesalahan saudara-saudaramu dan dosa mereka, sebab mereka telah berbuat jahat kepadamu. Maka sekarang, ampunilah kiranya kesalahan yang dibuat hamba-hamba Allah ayahmu." Lalu menangislah Yusuf, ketika orang berkata demikian kepadanya. 18 Juga saudara-saudaranya datang sendiri dan sujud di depannya serta berkata: "Kami datang untuk menjadi budakmu." 19 Tetapi Yusuf berkata kepada mereka: "Janganlah takut, sebab aku inikah pengganti Allah? 20 Memang kamu telah mereka-rekakan yang jahat terhadap aku, tetapi Allah telah mereka-rekakannya untuk kebaikan, dengan maksud melakukan seperti yang terjadi sekarang ini, yakni memelihara hidup suatu bangsa yang besar. 21 Jadi janganlah takut, aku akan menanggung makanmu dan makan anak-anakmu juga." Demikianlah ia menghiburkan mereka dan menenangkan hati mereka dengan perkataannya. | 22 Adapun Yusuf, ia tetap tinggal di Mesir beserta kaum keluarganya; dan Yusuf hidup seratus sepuluh tahun. 23 Jadi Yusuf sempat melihat anak cucu Efraim sampai keturunan yang ketiga; juga anak-anak Makhir, anak Manasye, lahir di pangkuan Yusuf. 24 Berkatalah Yusuf kepada saudara-saudaranya: "Tidak lama lagi aku akan mati; tentu Allah akan memperhatikan kamu dan membawa kamu keluar dari negeri ini, ke negeri yang telah dijanjikan-Nya dengan sumpah kepada Abraham, Ishak dan Yakub." 25 Lalu Yusuf menyuruh anak-anak Israel bersumpah, katanya: "Tentu Allah akan memperhatikan kamu; pada waktu itu kamu harus membawa tulang-tulangku dari sini." 26 Kemudian matilah Yusuf, berumur seratus sepuluh tahun. Mayatnya dirempah-rempahi, dan ditaruh dalam peti mati di Mesir.
\end{biblechapter}
