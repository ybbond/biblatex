\book{Kejadian}

% Bab 1
\begin{biblechapter}
\verse Pada mulanya Allah menciptakan langit dan bumi.
\verse Bumi belum berbentuk dan kosong; gelap gulita menutupi samudera raya, dan Roh Allah melayang-layang di atas permukaan air.
\verse Berfirmanlah Allah: "Jadilah terang." Lalu terang itu jadi.
\verse Allah melihat bahwa terang itu baik, lalu dipisahkan-Nyalah terang itu dari gelap.
\verse Dan Allah menamai terang itu siang, dan gelap itu malam. Jadilah petang dan jadilah pagi, itulah hari pertama.
\verse Berfirmanlah Allah: "Jadilah cakrawala di tengah segala air untuk memisahkan air dari air."
\verse Maka Allah menjadikan cakrawala dan Ia memisahkan air yang ada di bawah cakrawala itu dari air yang ada di atasnya. Dan jadilah demikian.
\verse Lalu Allah menamai cakrawala itu langit. Jadilah petang dan jadilah pagi, itulah hari kedua.
\verse Berfirmanlah Allah: "Hendaklah segala air yang di bawah langit berkumpul pada satu tempat, sehingga kelihatan yang kering." Dan jadilah demikian.
\verse Lalu Allah menamai yang kering itu darat, dan kumpulan air itu dinamai-Nya laut. Allah melihat bahwa semuanya itu baik.
\verse Berfirmanlah Allah: "Hendaklah tanah menumbuhkan tunas-tunas muda, tumbuh-tumbuhan yang berbiji, segala jenis pohon buah-buahan yang menghasilkan buah yang berbiji, supaya ada tumbuh-tumbuhan di bumi." Dan jadilah demikian.
\verse Tanah itu menumbuhkan tunas-tunas muda, segala jenis tumbuh-tumbuhan yang berbiji dan segala jenis pohon-pohonan yang menghasilkan buah yang berbiji. Allah melihat bahwa semuanya itu baik.
\verse Jadilah petang dan jadilah pagi, itulah hari ketiga.
\verse Berfirmanlah Allah: "Jadilah benda-benda penerang pada cakrawala untuk memisahkan siang dari malam. Biarlah benda-benda penerang itu menjadi tanda yang menunjukkan masa-masa yang tetap dan hari-hari dan tahun-tahun,
\verse dan sebagai penerang pada cakrawala biarlah benda-benda itu menerangi bumi." Dan jadilah demikian.
\verse Maka Allah menjadikan kedua benda penerang yang besar itu, yakni yang lebih besar untuk menguasai siang dan yang lebih kecil untuk menguasai malam, dan menjadikan juga bintang-bintang.
\verse Allah menaruh semuanya itu di cakrawala untuk menerangi bumi,
\verse dan untuk menguasai siang dan malam, dan untuk memisahkan terang dari gelap. Allah melihat bahwa semuanya itu baik.
\verse Jadilah petang dan jadilah pagi, itulah hari keempat.
\verse Berfirmanlah Allah: "Hendaklah dalam air berkeriapan makhluk yang hidup, dan hendaklah burung beterbangan di atas bumi melintasi cakrawala."
\verse Maka Allah menciptakan binatang-binatang laut yang besar dan segala jenis makhluk hidup yang bergerak, yang berkeriapan dalam air, dan segala jenis burung yang bersayap. Allah melihat bahwa semuanya itu baik.
\verse Lalu Allah memberkati semuanya itu, firman-Nya: "Berkembangbiaklah dan bertambah banyaklah serta penuhilah air dalam laut, dan hendaklah burung-burung di bumi bertambah banyak."
\verse Jadilah petang dan jadilah pagi, itulah hari kelima.
\verse Berfirmanlah Allah: "Hendaklah bumi mengeluarkan segala jenis makhluk yang hidup, ternak dan binatang melata dan segala jenis binatang liar." Dan jadilah demikian.
\verse Allah menjadikan segala jenis binatang liar dan segala jenis ternak dan segala jenis binatang melata di muka bumi. Allah melihat bahwa semuanya itu baik.
\verse Berfirmanlah Allah: "Baiklah Kita menjadikan manusia menurut gambar dan rupa Kita, supaya mereka berkuasa atas ikan-ikan di laut dan burung-burung di udara dan atas ternak dan atas seluruh bumi dan atas segala binatang melata yang merayap di bumi."
\verse Maka Allah menciptakan manusia itu menurut gambar-Nya, menurut gambar Allah diciptakan-Nya dia; laki-laki dan perempuan diciptakan-Nya mereka.
\verse Allah memberkati mereka, lalu Allah berfirman kepada mereka: "Beranakcuculah dan bertambah banyak; penuhilah bumi dan taklukkanlah itu, berkuasalah atas ikan-ikan di laut dan burung-burung di udara dan atas segala binatang yang merayap di bumi."
\verse Berfirmanlah Allah: "Lihatlah, Aku memberikan kepadamu segala tumbuh-tumbuhan yang berbiji di seluruh bumi dan segala pohon-pohonan yang buahnya berbiji; itulah akan menjadi makananmu.
\verse Tetapi kepada segala binatang di bumi dan segala burung di udara dan segala yang merayap di bumi, yang bernyawa, Kuberikan segala tumbuh-tumbuhan hijau menjadi makanannya." Dan jadilah demikian.
\verse Maka Allah melihat segala yang dijadikan-Nya itu, sungguh amat baik. Jadilah petang dan jadilah pagi, itulah hari keenam.
\end{biblechapter}

% Bab 2
\begin{biblechapter}
\verse Demikianlah diselesaikan langit dan bumi dan segala isinya.
\verse Ketika Allah pada hari ketujuh telah menyelesaikan pekerjaan yang dibuat-Nya itu, berhentilah Ia pada hari ketujuh dari segala pekerjaan yang telah dibuat-Nya itu.
\verse Lalu Allah memberkati hari ketujuh itu dan menguduskannya, karena pada hari itulah Ia berhenti dari segala pekerjaan penciptaan yang telah dibuat-Nya itu.
\verse Demikianlah riwayat langit dan bumi pada waktu diciptakan. Ketika TUHAN Allah menjadikan bumi dan langit, ?
\verse belum ada semak apapun di bumi, belum timbul tumbuh-tumbuhan apapun di padang, sebab TUHAN Allah belum menurunkan hujan ke bumi, dan belum ada orang untuk mengusahakan tanah itu;
\verse tetapi ada kabut naik ke atas dari bumi dan membasahi seluruh permukaan bumi itu?
\verse ketika itulah TUHAN Allah membentuk manusia itu dari debu tanah dan menghembuskan nafas hidup ke dalam hidungnya; demikianlah manusia itu menjadi makhluk yang hidup.
\verse Selanjutnya TUHAN Allah membuat taman di Eden, di sebelah timur; disitulah ditempatkan-Nya manusia yang dibentuk-Nya itu.
\verse Lalu TUHAN Allah menumbuhkan berbagai-bagai pohon dari bumi, yang menarik dan yang baik untuk dimakan buahnya; dan pohon kehidupan di tengah-tengah taman itu, serta pohon pengetahuan tentang yang baik dan yang jahat.
\verse Ada suatu sungai mengalir dari Eden untuk membasahi taman itu, dan dari situ sungai itu terbagi menjadi empat cabang.
\verse Yang pertama, namanya Pison, yakni yang mengalir mengelilingi seluruh tanah Hawila, tempat emas ada.
\verse Dan emas dari negeri itu baik; di sana ada damar bedolah dan batu krisopras.
\verse Nama sungai yang kedua ialah Gihon, yakni yang mengalir mengelilingi seluruh tanah Kush.
\verse Nama sungai yang ketiga ialah Tigris, yakni yang mengalir di sebelah timur Asyur. Dan sungai yang keempat ialah Efrat.
\verse TUHAN Allah mengambil manusia itu dan menempatkannya dalam taman Eden untuk mengusahakan dan memelihara taman itu.
\verse Lalu TUHAN Allah memberi perintah ini kepada manusia: "Semua pohon dalam taman ini boleh kaumakan buahnya dengan bebas,
\verse tetapi pohon pengetahuan tentang yang baik dan yang jahat itu, janganlah kaumakan buahnya, sebab pada hari engkau memakannya, pastilah engkau mati."
\verse TUHAN Allah berfirman: "Tidak baik, kalau manusia itu seorang diri saja. Aku akan menjadikan penolong baginya, yang sepadan dengan dia."
\verse Lalu TUHAN Allah membentuk dari tanah segala binatang hutan dan segala burung di udara. Dibawa-Nyalah semuanya kepada manusia itu untuk melihat, bagaimana ia menamainya; dan seperti nama yang diberikan manusia itu kepada tiap-tiap makhluk yang hidup, demikianlah nanti nama makhluk itu.
\verse Manusia itu memberi nama kepada segala ternak, kepada burung-burung di udara dan kepada segala binatang hutan, tetapi baginya sendiri ia tidak menjumpai penolong yang sepadan dengan dia.
\verse Lalu TUHAN Allah membuat manusia itu tidur nyenyak; ketika ia tidur, TUHAN Allah mengambil salah satu rusuk dari padanya, lalu menutup tempat itu dengan daging.
\verse Dan dari rusuk yang diambil TUHAN Allah dari manusia itu, dibangun-Nyalah seorang perempuan, lalu dibawa-Nya kepada manusia itu.
\verse Lalu berkatalah manusia itu: "Inilah dia, tulang dari tulangku dan daging dari dagingku. Ia akan dinamai perempuan, sebab ia diambil dari laki-laki."
\verse Sebab itu seorang laki-laki akan meninggalkan ayahnya dan ibunya dan bersatu dengan isterinya, sehingga keduanya menjadi satu daging.
\verse Mereka keduanya telanjang, manusia dan isterinya itu, tetapi mereka tidak merasa malu.
\end{biblechapter}
